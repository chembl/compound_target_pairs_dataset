%% Generated by Sphinx.
\def\sphinxdocclass{report}
\documentclass[letterpaper,10pt,english]{sphinxmanual}
\ifdefined\pdfpxdimen
   \let\sphinxpxdimen\pdfpxdimen\else\newdimen\sphinxpxdimen
\fi \sphinxpxdimen=.75bp\relax
\ifdefined\pdfimageresolution
    \pdfimageresolution= \numexpr \dimexpr1in\relax/\sphinxpxdimen\relax
\fi
%% let collapsible pdf bookmarks panel have high depth per default
\PassOptionsToPackage{bookmarksdepth=5}{hyperref}

\PassOptionsToPackage{booktabs}{sphinx}
\PassOptionsToPackage{colorrows}{sphinx}

\PassOptionsToPackage{warn}{textcomp}
\usepackage[utf8]{inputenc}
\ifdefined\DeclareUnicodeCharacter
% support both utf8 and utf8x syntaxes
  \ifdefined\DeclareUnicodeCharacterAsOptional
    \def\sphinxDUC#1{\DeclareUnicodeCharacter{"#1}}
  \else
    \let\sphinxDUC\DeclareUnicodeCharacter
  \fi
  \sphinxDUC{00A0}{\nobreakspace}
  \sphinxDUC{2500}{\sphinxunichar{2500}}
  \sphinxDUC{2502}{\sphinxunichar{2502}}
  \sphinxDUC{2514}{\sphinxunichar{2514}}
  \sphinxDUC{251C}{\sphinxunichar{251C}}
  \sphinxDUC{2572}{\textbackslash}
\fi
\usepackage{cmap}
\usepackage[T1]{fontenc}
\usepackage{amsmath,amssymb,amstext}
\usepackage{babel}



\usepackage{tgtermes}
\usepackage{tgheros}
\renewcommand{\ttdefault}{txtt}



\usepackage[Bjarne]{fncychap}
\usepackage{sphinx}

\fvset{fontsize=auto}
\usepackage{geometry}


% Include hyperref last.
\usepackage{hyperref}
% Fix anchor placement for figures with captions.
\usepackage{hypcap}% it must be loaded after hyperref.
% Set up styles of URL: it should be placed after hyperref.
\urlstyle{same}

\addto\captionsenglish{\renewcommand{\contentsname}{Contents:}}

\usepackage{sphinxmessages}
\setcounter{tocdepth}{1}



\title{compound\_target\_pairs\_dataset}
\date{Feb 13, 2024}
\release{0.0.1}
\author{Lina Heinzke, Barbara Zdrazil}
\newcommand{\sphinxlogo}{\vbox{}}
\renewcommand{\releasename}{Release}
\makeindex
\begin{document}

\ifdefined\shorthandoff
  \ifnum\catcode`\=\string=\active\shorthandoff{=}\fi
  \ifnum\catcode`\"=\active\shorthandoff{"}\fi
\fi

\pagestyle{empty}
\sphinxmaketitle
\pagestyle{plain}
\sphinxtableofcontents
\pagestyle{normal}
\phantomsection\label{\detokenize{index::doc}}


\sphinxstepscope


\chapter{Introduction}
\label{\detokenize{introduction:introduction}}\label{\detokenize{introduction::doc}}
\sphinxAtStartPar
This code extract a dataset of compound\sphinxhyphen{}target pairs from the open\sphinxhyphen{}source bioactivity database \sphinxhref{https://www.ebi.ac.uk/chembl/}{ChEMBL} \sphinxcite{introduction:zdrazil2023}.

\sphinxAtStartPar
The compound\sphinxhyphen{}target pairs are known to interact because
\begin{itemize}
\item {} 
\sphinxAtStartPar
they have at least one corresponding measured activity values in ChEMBL or

\item {} 
\sphinxAtStartPar
they are part of a set of manually curated known interactions in ChEMBL.

\end{itemize}

\sphinxAtStartPar
Furthermore, the dataset contains a number of compounds and target annotations to enable future analyses.

\sphinxAtStartPar
Previously, a similar dataset has been curated manually and has been used to investigate target\sphinxhyphen{}based differences in drug\sphinxhyphen{}like properties and ligand efficiencies \sphinxcite{introduction:leeson2021}.
This code can generate an extended version of the previous dataset for every ChEMBL version from ChEMBL 26 onwards.


\section{Dataset Documentation}
\label{\detokenize{introduction:dataset-documentation}}
\sphinxAtStartPar
If you are interested in understanding the fields in the resulting dataset, see {\hyperref[\detokenize{columns_docs::doc}]{\sphinxcrossref{\DUrole{doc}{Columns in the Final Dataset}}}}


\section{User Guide}
\label{\detokenize{introduction:user-guide}}
\sphinxAtStartPar
If you are interested in using the code, see {\hyperref[\detokenize{user_guide::doc}]{\sphinxcrossref{\DUrole{doc}{User Guide}}}}


\section{Code Documentation}
\label{\detokenize{introduction:code-documentation}}
\sphinxAtStartPar
If you are interested in understanding the code, see {\hyperref[\detokenize{modules::doc}]{\sphinxcrossref{\DUrole{doc}{src}}}}

\sphinxstepscope


\chapter{Columns in the Final Dataset}
\label{\detokenize{columns_docs:columns-in-the-final-dataset}}\label{\detokenize{columns_docs::doc}}
\sphinxAtStartPar
This page provides explanations for all columns available in the final dataset.

\sphinxAtStartPar
More information on ChEMBL\sphinxhyphen{}based columns can be found in the respective \sphinxhref{https://ftp.ebi.ac.uk/pub/databases/chembl/ChEMBLdb/releases/}{ChEMBL schema documentation}.
The information on this page mostly corresponds to the \sphinxhref{https://ftp.ebi.ac.uk/pub/databases/chembl/ChEMBLdb/releases/chembl\_32/schema\_documentation.html}{ChEMBL 32 schema documentation}.


\section{Initial Query}
\label{\detokenize{columns_docs:initial-query}}

\subsection{PChEMBL Values}
\label{\detokenize{columns_docs:pchembl-values}}
\sphinxAtStartPar
The pchembl\_value is later aggregated into mean, max and median per compound\sphinxhyphen{}target pair and dropped.


\begin{savenotes}\sphinxattablestart
\sphinxthistablewithglobalstyle
\centering
\begin{tabular}[t]{\X{20}{100}\X{10}{100}\X{15}{100}\X{20}{100}\X{35}{100}}
\sphinxtoprule
\sphinxstyletheadfamily 
\sphinxAtStartPar
Column Name
&\sphinxstyletheadfamily 
\sphinxAtStartPar
Type
&\sphinxstyletheadfamily 
\sphinxAtStartPar
Info Re.
&\sphinxstyletheadfamily 
\sphinxAtStartPar
Based On
&\sphinxstyletheadfamily 
\sphinxAtStartPar
Description / Notes
\\
\sphinxmidrule
\sphinxtableatstartofbodyhook
\sphinxAtStartPar
pchembl\_value
&
\sphinxAtStartPar
Float
&
\sphinxAtStartPar
Compound\sphinxhyphen{}Target Pair
&
\sphinxAtStartPar
ChEMBL: activities
&
\sphinxAtStartPar
Negative log of selected concentration\sphinxhyphen{}response activity values (IC50 / EC50 / XC50 / AC50 / Ki / Kd / Potency).
\\
\sphinxbottomrule
\end{tabular}
\sphinxtableafterendhook\par
\sphinxattableend\end{savenotes}


\subsection{Compound Information}
\label{\detokenize{columns_docs:compound-information}}

\begin{savenotes}\sphinxattablestart
\sphinxthistablewithglobalstyle
\centering
\begin{tabular}[t]{\X{20}{100}\X{10}{100}\X{15}{100}\X{20}{100}\X{35}{100}}
\sphinxtoprule
\sphinxstyletheadfamily 
\sphinxAtStartPar
Column Name
&\sphinxstyletheadfamily 
\sphinxAtStartPar
Type
&\sphinxstyletheadfamily 
\sphinxAtStartPar
Info Re.
&\sphinxstyletheadfamily 
\sphinxAtStartPar
Based On
&\sphinxstyletheadfamily 
\sphinxAtStartPar
Description / Notes
\\
\sphinxmidrule
\sphinxtableatstartofbodyhook
\sphinxAtStartPar
parent\_molregno
&
\sphinxAtStartPar
Int
&
\sphinxAtStartPar
Compound
&
\sphinxAtStartPar
ChEMBL: molecule\_dictionary
&
\sphinxAtStartPar
Internal Primary Key for the molecule
\\
\sphinxhline
\sphinxAtStartPar
parent\_chemblid
&
\sphinxAtStartPar
String
&
\sphinxAtStartPar
Compound
&
\sphinxAtStartPar
“
&
\sphinxAtStartPar
ChEMBL identifier for this compound (for use on web interface etc)
\\
\sphinxhline
\sphinxAtStartPar
parent\_pref\_name
&
\sphinxAtStartPar
String
&
\sphinxAtStartPar
Compound
&
\sphinxAtStartPar
“
&
\sphinxAtStartPar
Preferred name for the molecule
\\
\sphinxhline
\sphinxAtStartPar
max\_phase
&
\sphinxAtStartPar
Float
&
\sphinxAtStartPar
Compound
&
\sphinxAtStartPar
“
&
\sphinxAtStartPar
Maximum phase of development reached for the compound across all indications %
\begin{footnote}[1]\sphinxAtStartFootnote
There have been changes to the max\_phase field in ChEMBL with \sphinxhref{https://ftp.ebi.ac.uk/pub/databases/chembl/ChEMBLdb/releases/chembl\_32/chembl\_32\_release\_notes.txt}{version 32}. See {\hyperref[\detokenize{columns_docs:max-phase-in-chembl}]{\sphinxcrossref{MAX\_PHASE in ChEMBL}}}.
%
\end{footnote}
\\
\sphinxhline
\sphinxAtStartPar
first\_approval
&
\sphinxAtStartPar
Int
&
\sphinxAtStartPar
Compound
&
\sphinxAtStartPar
“
&
\sphinxAtStartPar
Earliest known approval year for the drug (NULL is the default value)
\\
\sphinxhline
\sphinxAtStartPar
usan\_year
&
\sphinxAtStartPar
Int
&
\sphinxAtStartPar
Compound
&
\sphinxAtStartPar
“
&
\sphinxAtStartPar
The year in which the application for a USAN/INN name was granted. (NULL is the default value)
\\
\sphinxhline
\sphinxAtStartPar
black\_box\_warning
&
\sphinxAtStartPar
Int
&
\sphinxAtStartPar
Compound
&
\sphinxAtStartPar
“
&
\sphinxAtStartPar
Indicates that the drug has a black box warning (1 = yes, 0 = default value)
\\
\sphinxhline
\sphinxAtStartPar
prodrug
&
\sphinxAtStartPar
Int
&
\sphinxAtStartPar
Compound
&
\sphinxAtStartPar
“
&
\sphinxAtStartPar
Indicates that the drug is a pro\sphinxhyphen{}drug (1 = yes, 0 = no, \sphinxhyphen{}1 = preclinical compound ie not a drug)
\\
\sphinxhline
\sphinxAtStartPar
oral
&
\sphinxAtStartPar
Int
&
\sphinxAtStartPar
Compound
&
\sphinxAtStartPar
“
&
\sphinxAtStartPar
Indicates whether the drug is known to be administered orally (1 = yes, 0 = default value)
\\
\sphinxhline
\sphinxAtStartPar
parenteral
&
\sphinxAtStartPar
Int
&
\sphinxAtStartPar
Compound
&
\sphinxAtStartPar
“
&
\sphinxAtStartPar
Indicates whether the drug is known to be administered parenterally (1 = yes, 0 = default value)
\\
\sphinxhline
\sphinxAtStartPar
topical
&
\sphinxAtStartPar
Int
&
\sphinxAtStartPar
Compound
&
\sphinxAtStartPar
“
&
\sphinxAtStartPar
Indicates whether the drug is known to be administered topically (1 = yes, 0 = default value)
\\
\sphinxbottomrule
\end{tabular}
\sphinxtableafterendhook\par
\sphinxattableend\end{savenotes}


\subsection{Target Information}
\label{\detokenize{columns_docs:target-information}}

\begin{savenotes}\sphinxattablestart
\sphinxthistablewithglobalstyle
\centering
\begin{tabular}[t]{\X{20}{100}\X{10}{100}\X{15}{100}\X{20}{100}\X{35}{100}}
\sphinxtoprule
\sphinxstyletheadfamily 
\sphinxAtStartPar
Column Name
&\sphinxstyletheadfamily 
\sphinxAtStartPar
Type
&\sphinxstyletheadfamily 
\sphinxAtStartPar
Info Re.
&\sphinxstyletheadfamily 
\sphinxAtStartPar
Based On
&\sphinxstyletheadfamily 
\sphinxAtStartPar
Description / Notes
\\
\sphinxmidrule
\sphinxtableatstartofbodyhook
\sphinxAtStartPar
tid
&
\sphinxAtStartPar
Int
&
\sphinxAtStartPar
Target
&
\sphinxAtStartPar
ChEMBL: assays
&
\sphinxAtStartPar
Unique ID for the target
\\
\sphinxhline
\sphinxAtStartPar
mutation
&
\sphinxAtStartPar
String
&
\sphinxAtStartPar
Target
&
\sphinxAtStartPar
ChEMBL: variant\_sequences
&
\sphinxAtStartPar
Details of variant(s) used, with residue positions adjusted to match provided sequence.
\\
\sphinxhline
\sphinxAtStartPar
target\_chembl\_id
&
\sphinxAtStartPar
String
&
\sphinxAtStartPar
Target
&
\sphinxAtStartPar
ChEMBL: target\_dictionary
&
\sphinxAtStartPar
ChEMBL identifier for this target (for use on web interface etc)
\\
\sphinxhline
\sphinxAtStartPar
target\_pref\_name
&
\sphinxAtStartPar
String
&
\sphinxAtStartPar
Target
&
\sphinxAtStartPar
“
&
\sphinxAtStartPar
Preferred target name: manually curated
\\
\sphinxhline
\sphinxAtStartPar
target\_type
&
\sphinxAtStartPar
String
&
\sphinxAtStartPar
Target
&
\sphinxAtStartPar
“
&
\sphinxAtStartPar
Describes whether target is a protein, an organism, a tissue etc.
\\
\sphinxhline
\sphinxAtStartPar
organism
&
\sphinxAtStartPar
String
&
\sphinxAtStartPar
Target
&
\sphinxAtStartPar
“
&
\sphinxAtStartPar
Source organism of molecuar target or tissue, or the target organism if compound activity is reported in an organism rather than a protein or tissue
\\
\sphinxbottomrule
\end{tabular}
\sphinxtableafterendhook\par
\sphinxattableend\end{savenotes}


\subsection{Helper Columns}
\label{\detokenize{columns_docs:helper-columns}}
\sphinxAtStartPar
These columns are combination of other columns, used for easier processing of the dataset.


\begin{savenotes}\sphinxattablestart
\sphinxthistablewithglobalstyle
\centering
\begin{tabular}[t]{\X{20}{100}\X{10}{100}\X{15}{100}\X{20}{100}\X{35}{100}}
\sphinxtoprule
\sphinxstyletheadfamily 
\sphinxAtStartPar
Column Name
&\sphinxstyletheadfamily 
\sphinxAtStartPar
Type
&\sphinxstyletheadfamily 
\sphinxAtStartPar
Info Re.
&\sphinxstyletheadfamily 
\sphinxAtStartPar
Based On
&\sphinxstyletheadfamily 
\sphinxAtStartPar
Description / Notes
\\
\sphinxmidrule
\sphinxtableatstartofbodyhook
\sphinxAtStartPar
tid\_mutation
&
\sphinxAtStartPar
String
&
\sphinxAtStartPar
Target
&
\sphinxAtStartPar
tid + ‘\_’ + mutation
&
\sphinxAtStartPar
Helper column
\\
\sphinxhline
\sphinxAtStartPar
cpd\_target\_pair
&
\sphinxAtStartPar
String
&
\sphinxAtStartPar
Compound\sphinxhyphen{}Target Pair
&
\sphinxAtStartPar
parent\_molregno + ‘\_’ +  tid
&
\sphinxAtStartPar
Helper column
\\
\sphinxhline
\sphinxAtStartPar
cpd\_target\_pair\_ mutation
&
\sphinxAtStartPar
String
&
\sphinxAtStartPar
Compound\sphinxhyphen{}Target Pair
&
\sphinxAtStartPar
parent\_molregno + ‘\_’ +  tid\_mutation
&
\sphinxAtStartPar
Helper column
\\
\sphinxbottomrule
\end{tabular}
\sphinxtableafterendhook\par
\sphinxattableend\end{savenotes}


\section{Aggregated Values}
\label{\detokenize{columns_docs:aggregated-values}}
\sphinxAtStartPar
Aggregated per compound\sphinxhyphen{}target pair using parent\_molregno and tid\_mutation.


\begin{savenotes}\sphinxattablestart
\sphinxthistablewithglobalstyle
\centering
\begin{tabular}[t]{\X{30}{100}\X{10}{100}\X{15}{100}\X{45}{100}}
\sphinxtoprule
\sphinxstyletheadfamily 
\sphinxAtStartPar
Column Name
&\sphinxstyletheadfamily 
\sphinxAtStartPar
Type
&\sphinxstyletheadfamily 
\sphinxAtStartPar
Info Re.
&\sphinxstyletheadfamily 
\sphinxAtStartPar
Description / Notes
\\
\sphinxmidrule
\sphinxtableatstartofbodyhook
\sphinxAtStartPar
pchembl\_value\_mean\_BF / \_B
&
\sphinxAtStartPar
Float
&
\sphinxAtStartPar
Compound\sphinxhyphen{}Target Pair
&
\sphinxAtStartPar
Mean pchemb\_value for the compound\sphinxhyphen{}target pair
\\
\sphinxhline
\sphinxAtStartPar
pchembl\_value\_max\_BF / \_B
&
\sphinxAtStartPar
Float
&
\sphinxAtStartPar
Compound\sphinxhyphen{}Target Pair
&
\sphinxAtStartPar
Maximum pchemb\_value for the compound\sphinxhyphen{}target pair
\\
\sphinxhline
\sphinxAtStartPar
pchembl\_value\_median\_BF / \_B
&
\sphinxAtStartPar
Float
&
\sphinxAtStartPar
Compound\sphinxhyphen{}Target Pair
&
\sphinxAtStartPar
Median pchemb\_value for the compound\sphinxhyphen{}target pair
\\
\sphinxhline
\sphinxAtStartPar
first\_publication\_ cpd\_target\_pair\_BF /\_B
&
\sphinxAtStartPar
Int
&
\sphinxAtStartPar
Compound\sphinxhyphen{}Target Pair
&
\sphinxAtStartPar
First publication in ChEMBL with this compound\sphinxhyphen{}target pair
\\
\sphinxhline
\sphinxAtStartPar
first\_publication\_ cpd\_target\_pair\_ w\_pchembl\_BF / \_B
&
\sphinxAtStartPar
Int
&
\sphinxAtStartPar
Compound\sphinxhyphen{}Target Pair
&
\sphinxAtStartPar
First publication in ChEMBL with this compound\sphinxhyphen{}target pair and an associated pchembl value
\\
\sphinxbottomrule
\end{tabular}
\sphinxtableafterendhook\par
\sphinxattableend\end{savenotes}


\subsection{Naming Convention: B vs. BF}
\label{\detokenize{columns_docs:naming-convention-b-vs-bf}}
\sphinxAtStartPar
These values are aggregated based on different subsets of the full dataset.
The corresponding columns in the final dataset have a suffix that corresponds to the assay types the value is based on:
\begin{itemize}
\item {} 
\sphinxAtStartPar
\_BF: based on binding + functional assays

\item {} 
\sphinxAtStartPar
\_B: based on binding assays

\end{itemize}


\section{DTI (Drug\sphinxhyphen{}Target Interaction) Annotations}
\label{\detokenize{columns_docs:dti-drug-target-interaction-annotations}}
\sphinxAtStartPar
Based on cpd\_target\_pair, does not include mutation information.


\begin{savenotes}\sphinxattablestart
\sphinxthistablewithglobalstyle
\centering
\begin{tabular}[t]{\X{20}{100}\X{10}{100}\X{15}{100}\X{20}{100}\X{35}{100}}
\sphinxtoprule
\sphinxstyletheadfamily 
\sphinxAtStartPar
Column Name
&\sphinxstyletheadfamily 
\sphinxAtStartPar
Type
&\sphinxstyletheadfamily 
\sphinxAtStartPar
Info Re.
&\sphinxstyletheadfamily 
\sphinxAtStartPar
Based On
&\sphinxstyletheadfamily 
\sphinxAtStartPar
Description / Notes
\\
\sphinxmidrule
\sphinxtableatstartofbodyhook
\sphinxAtStartPar
therapeutic\_target
&
\sphinxAtStartPar
Bool
&
\sphinxAtStartPar
Target
&
\sphinxAtStartPar
ChEMBL: drug\_mechanism table
&
\sphinxAtStartPar
Is the target in the drug mechanism table?
\\
\sphinxhline
\sphinxAtStartPar
DTI
&
\sphinxAtStartPar
String
&
\sphinxAtStartPar
Compound\sphinxhyphen{}Target Pair
&
\sphinxAtStartPar
Assigned as below
&
\sphinxAtStartPar
Drug target interaction (DTI) annotation
\\
\sphinxbottomrule
\end{tabular}
\sphinxtableafterendhook\par
\sphinxattableend\end{savenotes}


\subsection{Mechanism to Assign DTI}
\label{\detokenize{columns_docs:mechanism-to-assign-dti}}

\begin{savenotes}\sphinxattablestart
\sphinxthistablewithglobalstyle
\centering
\begin{tabular}[t]{\X{15}{100}\X{15}{100}\X{15}{100}\X{10}{100}\X{45}{100}}
\sphinxtoprule
\sphinxstyletheadfamily 
\sphinxAtStartPar
In DM Table? \sphinxfootnotemark[2]
&\sphinxstyletheadfamily 
\sphinxAtStartPar
max\_phase? \sphinxfootnotemark[3]
&\sphinxstyletheadfamily 
\sphinxAtStartPar
Th. Target? \sphinxfootnotemark[4]
&\sphinxstyletheadfamily 
\sphinxAtStartPar
DTI
&\sphinxstyletheadfamily 
\sphinxAtStartPar
Explanation
\\
\sphinxmidrule
\sphinxtableatstartofbodyhook%
\begin{footnotetext}[2]\sphinxAtStartFootnote
Is the compound\sphinxhyphen{}target pair in the drug\_mechanisms table? = Is it a known relevant compound\sphinxhyphen{}target interaction?
%
\end{footnotetext}\ignorespaces %
\begin{footnotetext}[3]\sphinxAtStartFootnote
What is the max\_phase of the compound? = Is it a drug / clinical compound?
%
\end{footnotetext}\ignorespaces %
\begin{footnotetext}[4]\sphinxAtStartFootnote
Is the target in the drug\_mechanisms table? = Is it a therapeutic target?
%
\end{footnotetext}\ignorespaces 
\sphinxAtStartPar
Yes
&
\sphinxAtStartPar
4
&
\sphinxAtStartPar
\textendash{}
&
\sphinxAtStartPar
D\_DT
&
\sphinxAtStartPar
Drug \sphinxhyphen{} drug target
\\
\sphinxhline
\sphinxAtStartPar
Yes
&
\sphinxAtStartPar
3
&
\sphinxAtStartPar
\textendash{}
&
\sphinxAtStartPar
C3\_DT
&
\sphinxAtStartPar
Clinical candidate in phase 3 \sphinxhyphen{} drug target
\\
\sphinxhline
\sphinxAtStartPar
Yes
&
\sphinxAtStartPar
2
&
\sphinxAtStartPar
\textendash{}
&
\sphinxAtStartPar
C2\_DT
&
\sphinxAtStartPar
Clinical candidate in phase 2 \sphinxhyphen{} drug target
\\
\sphinxhline
\sphinxAtStartPar
Yes
&
\sphinxAtStartPar
1
&
\sphinxAtStartPar
\textendash{}
&
\sphinxAtStartPar
C1\_DT
&
\sphinxAtStartPar
Clinical candidate in phase 1 \sphinxhyphen{} drug target
\\
\sphinxhline
\sphinxAtStartPar
Yes
&
\sphinxAtStartPar
\textless{} 1
&
\sphinxAtStartPar
\textendash{}
&
\sphinxAtStartPar
C0\_DT
&
\sphinxAtStartPar
Compound in unknown clinical phase %
\begin{footnote}[5]\sphinxAtStartFootnote
There have been changes to the max\_phase field in ChEMBL with \sphinxhref{https://ftp.ebi.ac.uk/pub/databases/chembl/ChEMBLdb/releases/chembl\_32/chembl\_32\_release\_notes.txt}{version 32}.
C0\_DT groups together all compounds with a max\_phase not between 1 and 4. See {\hyperref[\detokenize{columns_docs:max-phase-in-chembl}]{\sphinxcrossref{MAX\_PHASE in ChEMBL}}}
%
\end{footnote}  \sphinxhyphen{} drug target
\\
\sphinxhline
\sphinxAtStartPar
No
&
\sphinxAtStartPar
\textendash{}
&
\sphinxAtStartPar
Yes
&
\sphinxAtStartPar
DT
&
\sphinxAtStartPar
Drug target
\\
\sphinxhline
\sphinxAtStartPar
No
&
\sphinxAtStartPar
\textendash{}
&
\sphinxAtStartPar
No
&
\sphinxAtStartPar
NDT
&
\sphinxAtStartPar
Not drug target
\\
\sphinxbottomrule
\end{tabular}
\sphinxtableafterendhook\par
\sphinxattableend\end{savenotes}


\subsection{MAX\_PHASE in ChEMBL}
\label{\detokenize{columns_docs:max-phase-in-chembl}}
\sphinxAtStartPar
Before ChEMBL 32, compounds with a max\_phase not between 1 and 4 were assigned a max\_phase of 0.

\begin{DUlineblock}{0em}
\item[] From ChEMBL 32 onwards, compounds with a max\_phase not between 1 and 4 can have three possible values:
\item[]
\begin{DUlineblock}{\DUlineblockindent}
\item[] \sphinxhyphen{} 0.5 = early phase 1 clinical trials
\item[] \sphinxhyphen{} \sphinxhyphen{}1 = clinical phase unknown for drug or clinical candidate drug, i.e., where ChEMBL cannot assign a clinical phase
\item[] \sphinxhyphen{} NULL = preclinical compounds with bioactivity data
\end{DUlineblock}
\end{DUlineblock}


\section{Compound and Target Properties Based on ChEMBL Data}
\label{\detokenize{columns_docs:compound-and-target-properties-based-on-chembl-data}}

\subsection{First publication}
\label{\detokenize{columns_docs:first-publication}}
\sphinxAtStartPar
In contrast to the aggregated time\sphinxhyphen{}related fields,
this field takes all of ChEMBL and not just the time\sphinxhyphen{}related data within the dataset into account.


\begin{savenotes}\sphinxattablestart
\sphinxthistablewithglobalstyle
\centering
\begin{tabular}[t]{\X{20}{100}\X{10}{100}\X{15}{100}\X{20}{100}\X{35}{100}}
\sphinxtoprule
\sphinxstyletheadfamily 
\sphinxAtStartPar
Column Name
&\sphinxstyletheadfamily 
\sphinxAtStartPar
Type
&\sphinxstyletheadfamily 
\sphinxAtStartPar
Info Re.
&\sphinxstyletheadfamily 
\sphinxAtStartPar
Based On
&\sphinxstyletheadfamily 
\sphinxAtStartPar
Description / Notes
\\
\sphinxmidrule
\sphinxtableatstartofbodyhook
\sphinxAtStartPar
first\_publication\_cpd
&
\sphinxAtStartPar
Int
&
\sphinxAtStartPar
Compound
&
\sphinxAtStartPar
ChEMBL: docs
&
\sphinxAtStartPar
First appearance of the compound in the literature
\\
\sphinxbottomrule
\end{tabular}
\sphinxtableafterendhook\par
\sphinxattableend\end{savenotes}


\subsection{Compound Properties}
\label{\detokenize{columns_docs:compound-properties}}

\begin{savenotes}\sphinxattablestart
\sphinxthistablewithglobalstyle
\centering
\begin{tabular}[t]{\X{20}{100}\X{10}{100}\X{15}{100}\X{20}{100}\X{35}{100}}
\sphinxtoprule
\sphinxstyletheadfamily 
\sphinxAtStartPar
Column Name
&\sphinxstyletheadfamily 
\sphinxAtStartPar
Type
&\sphinxstyletheadfamily 
\sphinxAtStartPar
Info Re.
&\sphinxstyletheadfamily 
\sphinxAtStartPar
Based On
&\sphinxstyletheadfamily 
\sphinxAtStartPar
Description / Notes
\\
\sphinxmidrule
\sphinxtableatstartofbodyhook
\sphinxAtStartPar
mw\_freebase
&
\sphinxAtStartPar
Float
&
\sphinxAtStartPar
Compound
&
\sphinxAtStartPar
ChEMBL: compound\_properties
&
\sphinxAtStartPar
Molecular weight of parent compound
\\
\sphinxhline
\sphinxAtStartPar
alogp
&
\sphinxAtStartPar
Float
&
\sphinxAtStartPar
Compound
&
\sphinxAtStartPar
“
&
\sphinxAtStartPar
Calculated ALogP
\\
\sphinxhline
\sphinxAtStartPar
hba
&
\sphinxAtStartPar
Int
&
\sphinxAtStartPar
Compound
&
\sphinxAtStartPar
“
&
\sphinxAtStartPar
Number hydrogen bond acceptors
\\
\sphinxhline
\sphinxAtStartPar
hbd
&
\sphinxAtStartPar
Int
&
\sphinxAtStartPar
Compound
&
\sphinxAtStartPar
“
&
\sphinxAtStartPar
Number hydrogen bond donors
\\
\sphinxhline
\sphinxAtStartPar
psa
&
\sphinxAtStartPar
Float
&
\sphinxAtStartPar
Compound
&
\sphinxAtStartPar
“
&
\sphinxAtStartPar
Polar surface area
\\
\sphinxhline
\sphinxAtStartPar
rtb
&
\sphinxAtStartPar
Int
&
\sphinxAtStartPar
Compound
&
\sphinxAtStartPar
“
&
\sphinxAtStartPar
Number rotatable bonds
\\
\sphinxhline
\sphinxAtStartPar
ro3\_pass
&
\sphinxAtStartPar
String
&
\sphinxAtStartPar
Compound
&
\sphinxAtStartPar
“
&
\sphinxAtStartPar
Indicates whether the compound passes the rule\sphinxhyphen{}of\sphinxhyphen{}three (mw \textless{} 300, logP \textless{} 3 etc)
\\
\sphinxhline
\sphinxAtStartPar
num\_ro5\_violations
&
\sphinxAtStartPar
Int
&
\sphinxAtStartPar
Compound
&
\sphinxAtStartPar
“
&
\sphinxAtStartPar
Number of violations of Lipinski’s rule\sphinxhyphen{}of\sphinxhyphen{}five, using HBA and HBD definitions
\\
\sphinxhline
\sphinxAtStartPar
cx\_most\_apka
&
\sphinxAtStartPar
Float
&
\sphinxAtStartPar
Compound
&
\sphinxAtStartPar
“
&
\sphinxAtStartPar
The most acidic pKa calculated using ChemAxon
\\
\sphinxhline
\sphinxAtStartPar
cx\_most\_bpka
&
\sphinxAtStartPar
Float
&
\sphinxAtStartPar
Compound
&
\sphinxAtStartPar
“
&
\sphinxAtStartPar
The most basic pKa calculated using ChemAxon
\\
\sphinxhline
\sphinxAtStartPar
cx\_logp
&
\sphinxAtStartPar
Float
&
\sphinxAtStartPar
Compound
&
\sphinxAtStartPar
“
&
\sphinxAtStartPar
The calculated octanol/water partition coefficient using ChemAxon
\\
\sphinxhline
\sphinxAtStartPar
cx\_logd
&
\sphinxAtStartPar
Float
&
\sphinxAtStartPar
Compound
&
\sphinxAtStartPar
“
&
\sphinxAtStartPar
The calculated octanol/water distribution coefficient at pH7.4 using ChemAxon
\\
\sphinxhline
\sphinxAtStartPar
molecular\_species
&
\sphinxAtStartPar
String
&
\sphinxAtStartPar
Compound
&
\sphinxAtStartPar
“
&
\sphinxAtStartPar
Indicates whether the compound is an acid/base/neutral
\\
\sphinxhline
\sphinxAtStartPar
full\_mwt
&
\sphinxAtStartPar
Float
&
\sphinxAtStartPar
Compound
&
\sphinxAtStartPar
“
&
\sphinxAtStartPar
Molecular weight of the full compound including any salts
\\
\sphinxhline
\sphinxAtStartPar
aromatic\_rings
&
\sphinxAtStartPar
Int
&
\sphinxAtStartPar
Compound
&
\sphinxAtStartPar
“
&
\sphinxAtStartPar
Number of aromatic rings
\\
\sphinxhline
\sphinxAtStartPar
heavy\_atoms
&
\sphinxAtStartPar
Int
&
\sphinxAtStartPar
Compound
&
\sphinxAtStartPar
“
&
\sphinxAtStartPar
Number of heavy (non\sphinxhyphen{}hydrogen) atoms
\\
\sphinxhline
\sphinxAtStartPar
qed\_weighted
&
\sphinxAtStartPar
Float
&
\sphinxAtStartPar
Compound
&
\sphinxAtStartPar
“
&
\sphinxAtStartPar
Weighted quantitative estimate of drug likeness (as defined by Bickerton et al., Nature Chem 2012)
\\
\sphinxhline
\sphinxAtStartPar
mw\_monoisotopic
&
\sphinxAtStartPar
Float
&
\sphinxAtStartPar
Compound
&
\sphinxAtStartPar
“
&
\sphinxAtStartPar
Monoisotopic parent molecular weight
\\
\sphinxhline
\sphinxAtStartPar
full\_molformula
&
\sphinxAtStartPar
String
&
\sphinxAtStartPar
Compound
&
\sphinxAtStartPar
“
&
\sphinxAtStartPar
Molecular formula for the full compound (including any salt)
\\
\sphinxhline
\sphinxAtStartPar
hba\_lipinski
&
\sphinxAtStartPar
Int
&
\sphinxAtStartPar
Compound
&
\sphinxAtStartPar
“
&
\sphinxAtStartPar
Number of hydrogen bond acceptors calculated according to Lipinski’s original rules (i.e., N + O count))
\\
\sphinxhline
\sphinxAtStartPar
hbd\_lipinski
&
\sphinxAtStartPar
Int
&
\sphinxAtStartPar
Compound
&
\sphinxAtStartPar
“
&
\sphinxAtStartPar
Number of hydrogen bond donors calculated according to Lipinski’s original rules (i.e., NH + OH count)
\\
\sphinxhline
\sphinxAtStartPar
num\_lipinski\_ ro5\_violations
&
\sphinxAtStartPar
Int
&
\sphinxAtStartPar
Compound
&
\sphinxAtStartPar
“
&
\sphinxAtStartPar
Number of violations of Lipinski’s rule of five using HBA\_LIPINSKI and HBD\_LIPINSKI counts
\\
\sphinxbottomrule
\end{tabular}
\sphinxtableafterendhook\par
\sphinxattableend\end{savenotes}


\subsection{Compound Structures}
\label{\detokenize{columns_docs:compound-structures}}

\begin{savenotes}\sphinxattablestart
\sphinxthistablewithglobalstyle
\centering
\begin{tabular}[t]{\X{20}{100}\X{10}{100}\X{15}{100}\X{20}{100}\X{35}{100}}
\sphinxtoprule
\sphinxstyletheadfamily 
\sphinxAtStartPar
Column Name
&\sphinxstyletheadfamily 
\sphinxAtStartPar
Type
&\sphinxstyletheadfamily 
\sphinxAtStartPar
Info Re.
&\sphinxstyletheadfamily 
\sphinxAtStartPar
Based On
&\sphinxstyletheadfamily 
\sphinxAtStartPar
Description / Notes
\\
\sphinxmidrule
\sphinxtableatstartofbodyhook
\sphinxAtStartPar
standard\_inchi
&
\sphinxAtStartPar
String
&
\sphinxAtStartPar
Compound
&
\sphinxAtStartPar
ChEMBL: compound\_structures
&
\sphinxAtStartPar
IUPAC standard InChI for the compound
\\
\sphinxhline
\sphinxAtStartPar
standard\_inchi\_key
&
\sphinxAtStartPar
String
&
\sphinxAtStartPar
Compound
&
\sphinxAtStartPar
“
&
\sphinxAtStartPar
IUPAC standard InChI key for the compound
\\
\sphinxhline
\sphinxAtStartPar
canonical\_smiles
&
\sphinxAtStartPar
String
&
\sphinxAtStartPar
Compound
&
\sphinxAtStartPar
“
&
\sphinxAtStartPar
Canonical smiles, generated using RDKit
\\
\sphinxbottomrule
\end{tabular}
\sphinxtableafterendhook\par
\sphinxattableend\end{savenotes}


\subsection{ATC and Target Class}
\label{\detokenize{columns_docs:atc-and-target-class}}

\begin{savenotes}\sphinxattablestart
\sphinxthistablewithglobalstyle
\centering
\begin{tabular}[t]{\X{20}{100}\X{10}{100}\X{15}{100}\X{20}{100}\X{35}{100}}
\sphinxtoprule
\sphinxstyletheadfamily 
\sphinxAtStartPar
Column Name
&\sphinxstyletheadfamily 
\sphinxAtStartPar
Type
&\sphinxstyletheadfamily 
\sphinxAtStartPar
Info Re.
&\sphinxstyletheadfamily 
\sphinxAtStartPar
Based On
&\sphinxstyletheadfamily 
\sphinxAtStartPar
Description / Notes
\\
\sphinxmidrule
\sphinxtableatstartofbodyhook
\sphinxAtStartPar
atc\_level1
&
\sphinxAtStartPar
String
&
\sphinxAtStartPar
Compound
&
\sphinxAtStartPar
ChEMBL: atc\_classification, molecule\_atc\_ classification
&
\sphinxAtStartPar
Anatomical Therapeutic Chemical (ATC) classification, level 1
\\
\sphinxhline
\sphinxAtStartPar
target\_class\_l1
&
\sphinxAtStartPar
String
&
\sphinxAtStartPar
Target
&
\sphinxAtStartPar
ChEMBL: protein\_classification, protein\_family\_ classification
&
\sphinxAtStartPar
Target class, level 1 (more general)
\\
\sphinxhline
\sphinxAtStartPar
target\_class\_l2
&
\sphinxAtStartPar
String
&
\sphinxAtStartPar
Target
&
\sphinxAtStartPar
“
&
\sphinxAtStartPar
Target class, level 2 (more detailed)
\\
\sphinxbottomrule
\end{tabular}
\sphinxtableafterendhook\par
\sphinxattableend\end{savenotes}


\section{Ligand Efficiency Metrics}
\label{\detokenize{columns_docs:ligand-efficiency-metrics}}
\sphinxAtStartPar
Calculated based on pchembl\_value\_mean.

\sphinxAtStartPar
Since LE metrics are based on pChEMBL values, they are calculated twice.
Once for the pChEMBL values based on binding and functional assays (suffix \_BF)
and once for the pChEMBL values based on binding assays only (suffix \_B).


\begin{savenotes}\sphinxattablestart
\sphinxthistablewithglobalstyle
\centering
\begin{tabular}[t]{\X{20}{100}\X{20}{100}\X{20}{100}\X{40}{100}}
\sphinxtoprule
\sphinxstyletheadfamily 
\sphinxAtStartPar
Column Name
&\sphinxstyletheadfamily 
\sphinxAtStartPar
Type
&\sphinxstyletheadfamily 
\sphinxAtStartPar
Info Re.
&\sphinxstyletheadfamily 
\sphinxAtStartPar
Description / Notes
\\
\sphinxmidrule
\sphinxtableatstartofbodyhook
\sphinxAtStartPar
LE\_BF  / LE\_B
&
\sphinxAtStartPar
Float
&
\sphinxAtStartPar
Compound
&
\sphinxAtStartPar
Ligand efficiency
\\
\sphinxhline
\sphinxAtStartPar
BEI\_BF / BEI\_B
&
\sphinxAtStartPar
Float
&
\sphinxAtStartPar
Compound
&
\sphinxAtStartPar
Binding efficiency index
\\
\sphinxhline
\sphinxAtStartPar
SEI\_BF / SEI\_B
&
\sphinxAtStartPar
Float
&
\sphinxAtStartPar
Compound
&
\sphinxAtStartPar
Surface efficiency index
\\
\sphinxhline
\sphinxAtStartPar
LLE\_BF / LLE\_B
&
\sphinxAtStartPar
Float
&
\sphinxAtStartPar
Compound
&
\sphinxAtStartPar
Lipophilic ligand efficiency
\\
\sphinxbottomrule
\end{tabular}
\sphinxtableafterendhook\par
\sphinxattableend\end{savenotes}


\subsection{Equations}
\label{\detokenize{columns_docs:equations}}\begin{flalign*}
LE &= \frac{2.303 \cdot 298 \cdot 0.00199 \cdot pchembl\_value} {heavy\_atoms} \\
BEI  &= \frac{pchembl\_mean \cdot 1000}{mw\_freebase} \\
SEI &= \frac{pchembl\_mean \cdot 100}{PSA} \\
LLE &= pchembl\_mean - ALogP \\
\end{flalign*}

\section{RDKit\sphinxhyphen{}Based Compound Descriptors}
\label{\detokenize{columns_docs:rdkit-based-compound-descriptors}}

\subsection{Built\sphinxhyphen{}in Methods}
\label{\detokenize{columns_docs:built-in-methods}}
\sphinxAtStartPar
These compound descriptors are calculated using built\sphinxhyphen{}in RDKit methods from \sphinxhref{https://www.rdkit.org/docs/source/rdkit.Chem.Descriptors.html}{Descriptors} and \sphinxhref{https://www.rdkit.org/docs/source/rdkit.Chem.rdMolDescriptors.html}{rdMolDescriptors}.


\begin{savenotes}\sphinxattablestart
\sphinxthistablewithglobalstyle
\centering
\begin{tabular}[t]{\X{20}{100}\X{10}{100}\X{15}{100}\X{20}{100}\X{35}{100}}
\sphinxtoprule
\sphinxstyletheadfamily 
\sphinxAtStartPar
Column Name
&\sphinxstyletheadfamily 
\sphinxAtStartPar
Type
&\sphinxstyletheadfamily 
\sphinxAtStartPar
Info Re.
&\sphinxstyletheadfamily 
\sphinxAtStartPar
Based On
&\sphinxstyletheadfamily 
\sphinxAtStartPar
Description / Notes
\\
\sphinxmidrule
\sphinxtableatstartofbodyhook
\sphinxAtStartPar
fraction\_csp3
&
\sphinxAtStartPar
Float
&
\sphinxAtStartPar
Compound
&
\sphinxAtStartPar
canonical\_smiles + built\sphinxhyphen{}in RDKit methods
&
\sphinxAtStartPar
Fraction of C atoms that are SP3 hybridized (rdkit.Chem.Descriptors. FractionCSP3)
\\
\sphinxhline
\sphinxAtStartPar
ring\_count
&
\sphinxAtStartPar
Int
&
\sphinxAtStartPar
Compound
&
\sphinxAtStartPar
“
&
\sphinxAtStartPar
(rdkit.Chem.Descriptors. RingCount)
\\
\sphinxhline
\sphinxAtStartPar
num\_aliphatic\_ rings
&
\sphinxAtStartPar
Int
&
\sphinxAtStartPar
Compound
&
\sphinxAtStartPar
“
&
\sphinxAtStartPar
Number of aliphatic (containing at least one non\sphinxhyphen{}aromatic bond) rings (rdkit.Chem.Descriptors. NumAliphaticRings)
\\
\sphinxhline
\sphinxAtStartPar
num\_aliphatic\_ carbocycles
&
\sphinxAtStartPar
Int
&
\sphinxAtStartPar
Compound
&
\sphinxAtStartPar
“
&
\sphinxAtStartPar
Number of aliphatic (containing at least one non\sphinxhyphen{}aromatic bond) carbocycles (rdkit.Chem.Descriptors. NumAliphaticCarbocycles)
\\
\sphinxhline
\sphinxAtStartPar
num\_aliphatic\_ heterocycles
&
\sphinxAtStartPar
Int
&
\sphinxAtStartPar
Compound
&
\sphinxAtStartPar
“
&
\sphinxAtStartPar
Number of aliphatic (containing at least one non\sphinxhyphen{}aromatic bond) heterocycles (rdkit.Chem.Descriptors. NumAliphaticHeterocycles)
\\
\sphinxhline
\sphinxAtStartPar
num\_aromatic\_ rings
&
\sphinxAtStartPar
Int
&
\sphinxAtStartPar
Compound
&
\sphinxAtStartPar
“
&
\sphinxAtStartPar
Number of aromatic rings (rdkit.Chem.Descriptors. NumAromaticRings)
\\
\sphinxhline
\sphinxAtStartPar
num\_aromatic\_ carbocycles
&
\sphinxAtStartPar
Int
&
\sphinxAtStartPar
Compound
&
\sphinxAtStartPar
“
&
\sphinxAtStartPar
Number of aromatic carbocycles (rdkit.Chem.Descriptors. NumAromaticCarbocycles)
\\
\sphinxhline
\sphinxAtStartPar
num\_aromatic\_ heterocycles
&
\sphinxAtStartPar
Int
&
\sphinxAtStartPar
Compound
&
\sphinxAtStartPar
“
&
\sphinxAtStartPar
Number of aromatic heterocycles (rdkit.Chem.Descriptors. NumAromaticHeterocycles)
\\
\sphinxhline
\sphinxAtStartPar
num\_saturated\_ rings
&
\sphinxAtStartPar
Int
&
\sphinxAtStartPar
Compound
&
\sphinxAtStartPar
“
&
\sphinxAtStartPar
Number of saturated rings (rdkit.Chem.Descriptors. NumSaturatedRings)
\\
\sphinxhline
\sphinxAtStartPar
num\_saturated\_ carbocycles
&
\sphinxAtStartPar
Int
&
\sphinxAtStartPar
Compound
&
\sphinxAtStartPar
“
&
\sphinxAtStartPar
Number of saturated carbocycles (rdkit.Chem.Descriptors. NumSaturatedCarbocycles)
\\
\sphinxhline
\sphinxAtStartPar
num\_saturated\_ heterocycles
&
\sphinxAtStartPar
Int
&
\sphinxAtStartPar
Compound
&
\sphinxAtStartPar
“
&
\sphinxAtStartPar
Number of saturated heterocycles (rdkit.Chem.Descriptors. NumSaturatedHeterocycles)
\\
\sphinxhline
\sphinxAtStartPar
num\_stereocentres
&
\sphinxAtStartPar
Int
&
\sphinxAtStartPar
Compound
&
\sphinxAtStartPar
“
&
\sphinxAtStartPar
Number of atomic stereocenters (specified and unspecified) (rdkit.Chem.rdMolDescriptors. CalcNumAtomStereoCenters)
\\
\sphinxhline
\sphinxAtStartPar
num\_heteroatoms
&
\sphinxAtStartPar
Int
&
\sphinxAtStartPar
Compound
&
\sphinxAtStartPar
“
&
\sphinxAtStartPar
Number of heteroatoms (rdkit.Chem.Descriptors. NumHeteroatoms)
\\
\sphinxbottomrule
\end{tabular}
\sphinxtableafterendhook\par
\sphinxattableend\end{savenotes}


\subsection{Bespoke Methods}
\label{\detokenize{columns_docs:bespoke-methods}}
\sphinxAtStartPar
These compound descriptors are calculated using custom RDKit\sphinxhyphen{}based methods.


\begin{savenotes}\sphinxattablestart
\sphinxthistablewithglobalstyle
\centering
\begin{tabular}[t]{\X{20}{100}\X{10}{100}\X{15}{100}\X{20}{100}\X{35}{100}}
\sphinxtoprule
\sphinxstyletheadfamily 
\sphinxAtStartPar
Column Name
&\sphinxstyletheadfamily 
\sphinxAtStartPar
Type
&\sphinxstyletheadfamily 
\sphinxAtStartPar
Info Re.
&\sphinxstyletheadfamily 
\sphinxAtStartPar
Based On
&\sphinxstyletheadfamily 
\sphinxAtStartPar
Description / Notes
\\
\sphinxmidrule
\sphinxtableatstartofbodyhook
\sphinxAtStartPar
aromatic\_atoms
&
\sphinxAtStartPar
Int
&
\sphinxAtStartPar
Compound
&
\sphinxAtStartPar
canonical\_smiles + RDKit\sphinxhyphen{}based methods
&
\sphinxAtStartPar
Number of aromatic atoms
\\
\sphinxhline
\sphinxAtStartPar
aromatic\_c
&
\sphinxAtStartPar
Int
&
\sphinxAtStartPar
Compound
&
\sphinxAtStartPar
“
&
\sphinxAtStartPar
Number of aromatic C
\\
\sphinxhline
\sphinxAtStartPar
aromatic\_n
&
\sphinxAtStartPar
Int
&
\sphinxAtStartPar
Compound
&
\sphinxAtStartPar
“
&
\sphinxAtStartPar
Number of aromatic N
\\
\sphinxhline
\sphinxAtStartPar
aromatic\_hetero
&
\sphinxAtStartPar
Int
&
\sphinxAtStartPar
Compound
&
\sphinxAtStartPar
“
&
\sphinxAtStartPar
Number of aromatic hetero atoms
\\
\sphinxhline
\sphinxAtStartPar
scaffold\_ w\_stereo
&
\sphinxAtStartPar
String
&
\sphinxAtStartPar
Compound
&
\sphinxAtStartPar
“
&
\sphinxAtStartPar
Scaffold SMILES, including stereochemistry information
\\
\sphinxhline
\sphinxAtStartPar
scaffold\_ wo\_stereo
&
\sphinxAtStartPar
String
&
\sphinxAtStartPar
Compound
&
\sphinxAtStartPar
“
&
\sphinxAtStartPar
Scaffold SMILES of the molecule after removing stereochemistry information
\\
\sphinxbottomrule
\end{tabular}
\sphinxtableafterendhook\par
\sphinxattableend\end{savenotes}


\section{Annotations for Filtering}
\label{\detokenize{columns_docs:annotations-for-filtering}}
\sphinxAtStartPar
Columns are only available for the full dataset to facilitate the filtering into subsets.


\subsection{Helper Columns}
\label{\detokenize{columns_docs:id11}}
\sphinxAtStartPar
pair\_mutation\_in\_dm\_table and pair\_in\_dm\_table are similar fields.
They differ in whether mutation information is taken into account,
reflecting that mutation information is only sometimes taken into account
when calculating fields and adding rows to the dataset.
\begin{itemize}
\item {} \begin{description}
\sphinxlineitem{pair\_mutation\_in\_dm\_table:}
\sphinxAtStartPar
Is the compound\sphinxhyphen{}target pair in the drug\_mechanism table
when taking mutation information into account?
Mutation information IS taken into account when adding pairs to the dataset
because they appear in the drug\_mechanism table.
(cpd A, target B without mutation) will be added to the set of existing
compound\sphinxhyphen{}target pairs with pChEMBL values
if there is a pair with a pChEMBL value for (cpd A, target B with mutation C)
but there is no pair with a pChEMBL value for (cpd A, target B without mutation).
It is used to determine keep\_for\_binding which in turn is used
to determine the B subset of data based on binding assays.

\end{description}

\item {} \begin{description}
\sphinxlineitem{pair\_in\_dm\_table:}
\sphinxAtStartPar
Is the compound\sphinxhyphen{}target pair in the drug\_mechanism table
when ignoring mutation information?
Mutation information is NOT taken into account when assigning DTI values.

\end{description}

\end{itemize}


\begin{savenotes}\sphinxattablestart
\sphinxthistablewithglobalstyle
\centering
\begin{tabular}[t]{\X{20}{100}\X{10}{100}\X{15}{100}\X{55}{100}}
\sphinxtoprule
\sphinxstyletheadfamily 
\sphinxAtStartPar
Column Name
&\sphinxstyletheadfamily 
\sphinxAtStartPar
Type
&\sphinxstyletheadfamily 
\sphinxAtStartPar
Info Re.
&\sphinxstyletheadfamily 
\sphinxAtStartPar
Description / Notes
\\
\sphinxmidrule
\sphinxtableatstartofbodyhook
\sphinxAtStartPar
pair\_mutation\_in\_dm\_table
&
\sphinxAtStartPar
Bool
&
\sphinxAtStartPar
Compound\sphinxhyphen{}Target Pair
&
\sphinxAtStartPar
Is the compound\sphinxhyphen{}target pair (taking mutation annotation into account) in the drug mechanism table?
\\
\sphinxhline
\sphinxAtStartPar
pair\_in\_dm\_table
&
\sphinxAtStartPar
Bool
&
\sphinxAtStartPar
Compound\sphinxhyphen{}Target Pair
&
\sphinxAtStartPar
Is the compound\sphinxhyphen{}target pair (ignoring mutation annotation) in the drug mechanism table?
\\
\sphinxhline
\sphinxAtStartPar
keep\_for\_binding
&
\sphinxAtStartPar
Bool
&
\sphinxAtStartPar
Compound\sphinxhyphen{}Target Pair
&
\sphinxAtStartPar
Rows to keep if interested in information based only on binding assays + the drug\_mechanism table. True if pchembl\_value\_mean\_B (based on binding assays) exists or if pair\_mutation\_in\_dm\_table == True, i.e., the pair (including mutation information) is in the drug mechanism table.
\\
\sphinxbottomrule
\end{tabular}
\sphinxtableafterendhook\par
\sphinxattableend\end{savenotes}


\subsection{Filtering Columns}
\label{\detokenize{columns_docs:filtering-columns}}

\begin{savenotes}\sphinxattablestart
\sphinxthistablewithglobalstyle
\centering
\begin{tabular}[t]{\X{20}{100}\X{10}{100}\X{15}{100}\X{15}{100}\X{15}{100}\X{25}{100}}
\sphinxtoprule
\sphinxstyletheadfamily 
\sphinxAtStartPar
Column Name
&\sphinxstyletheadfamily 
\sphinxAtStartPar
Type
&\sphinxstyletheadfamily 
\sphinxAtStartPar
Info Re.
&\sphinxstyletheadfamily 
\sphinxAtStartPar
Assays
&\sphinxstyletheadfamily 
\sphinxAtStartPar
\#Comparators \sphinxfootnotemark[6]
&\sphinxstyletheadfamily 
\sphinxAtStartPar
Other
\\
\sphinxmidrule
\sphinxtableatstartofbodyhook%
\begin{footnotetext}[6]\sphinxAtStartFootnote
Comparator compounds in this context are all compounds with a pchembl\_value\_mean\_BF / \_B.
I.e., this includes compounds with a DTI of D\_DT or C\textless{}p\textgreater{}\_DT.
%
\end{footnotetext}\ignorespaces 
\sphinxAtStartPar
BF\_100
&
\sphinxAtStartPar
Bool
&
\sphinxAtStartPar
Compound\sphinxhyphen{}Target Pair
&
\sphinxAtStartPar
binding + functional
&
\sphinxAtStartPar
\textgreater{}= 100
&\\
\sphinxhline
\sphinxAtStartPar
BF\_100\_c\_dt\_d\_dt
&
\sphinxAtStartPar
Bool
&
\sphinxAtStartPar
Compound\sphinxhyphen{}Target Pair
&
\sphinxAtStartPar
binding + functional
&
\sphinxAtStartPar
\textgreater{}= 100
&
\sphinxAtStartPar
at least one compound with an annotation of D\_DT or C\textless{}p\textgreater{}\_DT (C0\_DT, C1\_DT, C2\_DT, C3\_DT) per target
\\
\sphinxhline
\sphinxAtStartPar
BF\_100\_d\_dt
&
\sphinxAtStartPar
Bool
&
\sphinxAtStartPar
Compound\sphinxhyphen{}Target Pair
&
\sphinxAtStartPar
binding + functional
&
\sphinxAtStartPar
\textgreater{}= 100
&
\sphinxAtStartPar
at least one compound with an annotation of D\_DT per target
\\
\sphinxhline
\sphinxAtStartPar
B\_100
&
\sphinxAtStartPar
Bool
&
\sphinxAtStartPar
Compound\sphinxhyphen{}Target Pair
&
\sphinxAtStartPar
binding
&
\sphinxAtStartPar
\textgreater{}= 100
&\\
\sphinxhline
\sphinxAtStartPar
B\_100\_c\_dt\_d\_dt
&
\sphinxAtStartPar
Bool
&
\sphinxAtStartPar
Compound\sphinxhyphen{}Target Pair
&
\sphinxAtStartPar
binding
&
\sphinxAtStartPar
\textgreater{}= 100
&
\sphinxAtStartPar
at least one compound with an annotation of D\_DT or C\textless{}p\textgreater{}\_DT (C0\_DT, C1\_DT, C2\_DT, C3\_DT) per target
\\
\sphinxhline
\sphinxAtStartPar
B\_100\_d\_dt
&
\sphinxAtStartPar
Bool
&
\sphinxAtStartPar
Compound\sphinxhyphen{}Target Pair
&
\sphinxAtStartPar
binding
&
\sphinxAtStartPar
\textgreater{}= 100
&
\sphinxAtStartPar
at least one compound with an annotation of D\_DT per target
\\
\sphinxbottomrule
\end{tabular}
\sphinxtableafterendhook\par
\sphinxattableend\end{savenotes}

\sphinxstepscope


\chapter{User Guide}
\label{\detokenize{user_guide:user-guide}}\label{\detokenize{user_guide::doc}}
\sphinxAtStartPar
The default version of the dataset (the full dataset as a CSV file based on the newest ChEMBL version) can be generated by calling

\begin{sphinxVerbatim}[commandchars=\\\{\}]
\PYG{n}{python} \PYG{n}{main}\PYG{o}{.}\PYG{n}{py} \PYG{o}{\PYGZhy{}}\PYG{n}{o} \PYG{o}{\PYGZlt{}}\PYG{n}{output\PYGZus{}path}\PYG{o}{\PYGZgt{}}
\end{sphinxVerbatim}

\sphinxAtStartPar
with further options explained in {\hyperref[\detokenize{user_guide:arguments}]{\sphinxcrossref{Arguments}}}.

\sphinxAtStartPar
An overview of the available arguments is also available by calling

\begin{sphinxVerbatim}[commandchars=\\\{\}]
\PYG{n}{python} \PYG{n}{main}\PYG{o}{.}\PYG{n}{py} \PYG{o}{\PYGZhy{}}\PYG{o}{\PYGZhy{}}\PYG{n}{help}
\end{sphinxVerbatim}

\sphinxAtStartPar
The output will always contain the full dataset as a CSV file.
The arguments only allow for the output of additional files or modify how the full dataset is extracted.


\section{Arguments}
\label{\detokenize{user_guide:arguments}}

\begin{savenotes}\sphinxattablestart
\sphinxthistablewithglobalstyle
\centering
\begin{tabular}[t]{\X{20}{100}\X{10}{100}\X{10}{100}\X{10}{100}\X{50}{100}}
\sphinxtoprule
\sphinxstyletheadfamily 
\sphinxAtStartPar
Parameter
&\sphinxstyletheadfamily 
\sphinxAtStartPar
Required
&\sphinxstyletheadfamily 
\sphinxAtStartPar
Flag
&\sphinxstyletheadfamily 
\sphinxAtStartPar
Default
&\sphinxstyletheadfamily 
\sphinxAtStartPar
Explanation
\\
\sphinxmidrule
\sphinxtableatstartofbodyhook
\sphinxAtStartPar
\sphinxhyphen{}\sphinxhyphen{}chembl, \sphinxhyphen{}c
&
\sphinxAtStartPar
No
&
\sphinxAtStartPar
No
&
\sphinxAtStartPar
None
&
\sphinxAtStartPar
ChEMBL version. The latest available ChEMBL version is used if this is not set.
\\
\sphinxhline
\sphinxAtStartPar
\sphinxhyphen{}\sphinxhyphen{}sqlite, \sphinxhyphen{}s
&
\sphinxAtStartPar
No
&
\sphinxAtStartPar
No
&
\sphinxAtStartPar
None
&
\sphinxAtStartPar
Path to SQLite database. If this is not set, ChEMBL is downloaded as an SQLite database and handled using the chembl\_downloader package.
\\
\sphinxhline
\sphinxAtStartPar
\sphinxhyphen{}\sphinxhyphen{}output, \sphinxhyphen{}o
&
\sphinxAtStartPar
Yes
&
\sphinxAtStartPar
No
&
\sphinxAtStartPar
None
&
\sphinxAtStartPar
Path to write the output file(s) to.
\\
\sphinxhline
\sphinxAtStartPar
\sphinxhyphen{}\sphinxhyphen{}delimiter, \sphinxhyphen{}d
&
\sphinxAtStartPar
No
&
\sphinxAtStartPar
No
&
\sphinxAtStartPar
;
&
\sphinxAtStartPar
Delimiter in output csv\sphinxhyphen{}files.
\\
\sphinxhline
\sphinxAtStartPar
\sphinxhyphen{}\sphinxhyphen{}all\_sources
&
\sphinxAtStartPar
No
&
\sphinxAtStartPar
Yes
&
\sphinxAtStartPar
n/a
&
\sphinxAtStartPar
Include all sources if this is set. By default, this is not set, and the dataset is calculated based on only literature sources.
\\
\sphinxhline
\sphinxAtStartPar
\sphinxhyphen{}\sphinxhyphen{}rdkit
&
\sphinxAtStartPar
No
&
\sphinxAtStartPar
Yes
&
\sphinxAtStartPar
n/a
&
\sphinxAtStartPar
Calculate RDKit\sphinxhyphen{}based compound properties if this is set.
\\
\sphinxhline
\sphinxAtStartPar
\sphinxhyphen{}\sphinxhyphen{}excel
&
\sphinxAtStartPar
No
&
\sphinxAtStartPar
Yes
&
\sphinxAtStartPar
n/a
&
\sphinxAtStartPar
Write the results to excel. Note: this may fail if the output is too large. The results will always be written to csv.
\\
\sphinxhline
\sphinxAtStartPar
\sphinxhyphen{}\sphinxhyphen{}BF
&
\sphinxAtStartPar
No
&
\sphinxAtStartPar
Yes
&
\sphinxAtStartPar
n/a
&
\sphinxAtStartPar
Write the subsets based on binding and functional assays.
\\
\sphinxhline
\sphinxAtStartPar
\sphinxhyphen{}\sphinxhyphen{}B
&
\sphinxAtStartPar
No
&
\sphinxAtStartPar
Yes
&
\sphinxAtStartPar
n/a
&
\sphinxAtStartPar
Write the subsets based on binding assays.
\\
\sphinxhline
\sphinxAtStartPar
\sphinxhyphen{}\sphinxhyphen{}debug
&
\sphinxAtStartPar
No
&
\sphinxAtStartPar
Yes
&
\sphinxAtStartPar
n/a
&
\sphinxAtStartPar
Log additional debugging information.
\\
\sphinxbottomrule
\end{tabular}
\sphinxtableafterendhook\par
\sphinxattableend\end{savenotes}


\section{Accessing ChEMBL}
\label{\detokenize{user_guide:accessing-chembl}}
\sphinxAtStartPar
ChEMBL is accessed either through a given path to an SQLite database download or through the \sphinxhref{https://github.com/cthoyt/chembl-downloader}{chembl\_downloader package}.
In both cases, SQLite is used to query ChEMBL.
Some of the earlier ChEMBL versions are missing tables or fields required to calculate the dataset.
Therefore, the earliest ChEMBL version for which the dataset can be calculated is ChEMBL 26.

\sphinxstepscope


\chapter{src}
\label{\detokenize{modules:src}}\label{\detokenize{modules::doc}}
\sphinxstepscope


\section{add\_chembl\_compound\_properties module}
\label{\detokenize{add_chembl_compound_properties:module-add_chembl_compound_properties}}\label{\detokenize{add_chembl_compound_properties:add-chembl-compound-properties-module}}\label{\detokenize{add_chembl_compound_properties::doc}}\index{module@\spxentry{module}!add\_chembl\_compound\_properties@\spxentry{add\_chembl\_compound\_properties}}\index{add\_chembl\_compound\_properties@\spxentry{add\_chembl\_compound\_properties}!module@\spxentry{module}}\index{add\_all\_chembl\_compound\_properties() (in module add\_chembl\_compound\_properties)@\spxentry{add\_all\_chembl\_compound\_properties()}\spxextra{in module add\_chembl\_compound\_properties}}

\begin{fulllineitems}
\phantomsection\label{\detokenize{add_chembl_compound_properties:add_chembl_compound_properties.add_all_chembl_compound_properties}}
\pysigstartsignatures
\pysiglinewithargsret{\sphinxcode{\sphinxupquote{add\_chembl\_compound\_properties.}}\sphinxbfcode{\sphinxupquote{add\_all\_chembl\_compound\_properties}}}{\sphinxparam{\DUrole{n}{df\_combined}\DUrole{p}{:}\DUrole{w}{ }\DUrole{n}{DataFrame}}\sphinxparamcomma \sphinxparam{\DUrole{n}{chembl\_con}\DUrole{p}{:}\DUrole{w}{ }\DUrole{n}{Connection}}\sphinxparamcomma \sphinxparam{\DUrole{n}{limit\_to\_literature}\DUrole{p}{:}\DUrole{w}{ }\DUrole{n}{bool}}}{{ $\rightarrow$ tuple\DUrole{p}{{[}}DataFrame\DUrole{p}{,}\DUrole{w}{ }DataFrame\DUrole{p}{,}\DUrole{w}{ }DataFrame\DUrole{p}{{]}}}}
\pysigstopsignatures
\sphinxAtStartPar
Add ChEMBL\sphinxhyphen{}based compound properties to the given compound\sphinxhyphen{}target pairs, specifically:
\begin{itemize}
\item {} 
\sphinxAtStartPar
the first publication date of a compound (first\_publication\_cpd)

\item {} 
\sphinxAtStartPar
ChEMBL compound properties

\item {} 
\sphinxAtStartPar
InChI, InChI key and canonical smiles

\item {} 
\sphinxAtStartPar
ligand efficiency metrics

\item {} 
\sphinxAtStartPar
ATC classifications

\end{itemize}
\begin{quote}\begin{description}
\sphinxlineitem{Parameters}\begin{itemize}
\item {} 
\sphinxAtStartPar
\sphinxstyleliteralstrong{\sphinxupquote{df\_combined}} (\sphinxstyleliteralemphasis{\sphinxupquote{pd.DataFrame}}) \textendash{} Pandas DataFrame with compound\sphinxhyphen{}target pairs

\item {} 
\sphinxAtStartPar
\sphinxstyleliteralstrong{\sphinxupquote{chembl\_con}} (\sphinxstyleliteralemphasis{\sphinxupquote{sqlite3.Connection}}) \textendash{} Sqlite3 connection to ChEMBL database.

\item {} 
\sphinxAtStartPar
\sphinxstyleliteralstrong{\sphinxupquote{limit\_to\_literature}} (\sphinxstyleliteralemphasis{\sphinxupquote{bool}}) \textendash{} Base first\_publication\_cpd on literature sources only if True.
Base it on all available sources otherwise.

\end{itemize}

\sphinxlineitem{Returns}
\sphinxAtStartPar
\begin{itemize}
\item {} 
\sphinxAtStartPar
Pandas DataFrame with added compound properties 

\item {} 
\sphinxAtStartPar
Pandas DataFrame with compound properties and structures for all compound ids in ChEMBL 

\item {} 
\sphinxAtStartPar
Pandas DataFrame with ATC annotations in ChEMBL

\end{itemize}


\sphinxlineitem{Return type}
\sphinxAtStartPar
(pd.DataFrame, pd.DataFrame, pd.DataFrame)

\end{description}\end{quote}

\end{fulllineitems}

\index{add\_atc\_classification() (in module add\_chembl\_compound\_properties)@\spxentry{add\_atc\_classification()}\spxextra{in module add\_chembl\_compound\_properties}}

\begin{fulllineitems}
\phantomsection\label{\detokenize{add_chembl_compound_properties:add_chembl_compound_properties.add_atc_classification}}
\pysigstartsignatures
\pysiglinewithargsret{\sphinxcode{\sphinxupquote{add\_chembl\_compound\_properties.}}\sphinxbfcode{\sphinxupquote{add\_atc\_classification}}}{\sphinxparam{\DUrole{n}{df\_combined}\DUrole{p}{:}\DUrole{w}{ }\DUrole{n}{DataFrame}}\sphinxparamcomma \sphinxparam{\DUrole{n}{chembl\_con}\DUrole{p}{:}\DUrole{w}{ }\DUrole{n}{Connection}}}{{ $\rightarrow$ tuple\DUrole{p}{{[}}DataFrame\DUrole{p}{,}\DUrole{w}{ }DataFrame\DUrole{p}{{]}}}}
\pysigstopsignatures
\sphinxAtStartPar
Query and add ATC classifications (level 1) from the atc\_classification and
molecule\_atc\_classification tables.
ATC level annotations for the same parent\_molregno are combined into one description
that concatenates all descriptions sorted alphabetically
into one string with ‘ | ‘ as a separator.
\begin{quote}\begin{description}
\sphinxlineitem{Parameters}\begin{itemize}
\item {} 
\sphinxAtStartPar
\sphinxstyleliteralstrong{\sphinxupquote{df\_combined}} (\sphinxstyleliteralemphasis{\sphinxupquote{pd.DataFrame}}) \textendash{} Pandas DataFrame with compound\sphinxhyphen{}target pairs

\item {} 
\sphinxAtStartPar
\sphinxstyleliteralstrong{\sphinxupquote{chembl\_con}} (\sphinxstyleliteralemphasis{\sphinxupquote{sqlite3.Connection}}) \textendash{} Sqlite3 connection to ChEMBL database.

\end{itemize}

\sphinxlineitem{Returns}
\sphinxAtStartPar
\begin{itemize}
\item {} 
\sphinxAtStartPar
Pandas DataFrame with added ATC classifications 

\item {} 
\sphinxAtStartPar
Pandas DataFrame with ATC annotations in ChEMBL

\end{itemize}


\sphinxlineitem{Return type}
\sphinxAtStartPar
(pd.DataFrame, pd.DataFrame)

\end{description}\end{quote}

\end{fulllineitems}

\index{add\_chembl\_properties\_and\_structures() (in module add\_chembl\_compound\_properties)@\spxentry{add\_chembl\_properties\_and\_structures()}\spxextra{in module add\_chembl\_compound\_properties}}

\begin{fulllineitems}
\phantomsection\label{\detokenize{add_chembl_compound_properties:add_chembl_compound_properties.add_chembl_properties_and_structures}}
\pysigstartsignatures
\pysiglinewithargsret{\sphinxcode{\sphinxupquote{add\_chembl\_compound\_properties.}}\sphinxbfcode{\sphinxupquote{add\_chembl\_properties\_and\_structures}}}{\sphinxparam{\DUrole{n}{df\_combined}\DUrole{p}{:}\DUrole{w}{ }\DUrole{n}{DataFrame}}\sphinxparamcomma \sphinxparam{\DUrole{n}{chembl\_con}\DUrole{p}{:}\DUrole{w}{ }\DUrole{n}{Connection}}}{{ $\rightarrow$ tuple\DUrole{p}{{[}}DataFrame\DUrole{p}{,}\DUrole{w}{ }DataFrame\DUrole{p}{{]}}}}
\pysigstopsignatures
\sphinxAtStartPar
Add compound properties from the compound\_properties table
(e.g., alogp, \#hydrogen bond acceptors / donors, etc.).
Add InChI, InChI key and canonical smiles.
\begin{quote}\begin{description}
\sphinxlineitem{Parameters}\begin{itemize}
\item {} 
\sphinxAtStartPar
\sphinxstyleliteralstrong{\sphinxupquote{df\_combined}} (\sphinxstyleliteralemphasis{\sphinxupquote{pd.DataFrame}}) \textendash{} Pandas DataFrame with compound\sphinxhyphen{}target pairs

\item {} 
\sphinxAtStartPar
\sphinxstyleliteralstrong{\sphinxupquote{chembl\_con}} (\sphinxstyleliteralemphasis{\sphinxupquote{sqlite3.Connection}}) \textendash{} Sqlite3 connection to ChEMBL database.

\end{itemize}

\sphinxlineitem{Returns}
\sphinxAtStartPar
\begin{itemize}
\item {} 
\sphinxAtStartPar
Pandas DataFrame with added compound properties and structures. 

\item {} 
\sphinxAtStartPar
Pandas DataFrame with compound properties and structures for all compound ids in ChEMBL.

\end{itemize}


\sphinxlineitem{Return type}
\sphinxAtStartPar
(pd.DataFrame, pd.DataFrame)

\end{description}\end{quote}

\end{fulllineitems}

\index{add\_first\_publication\_date() (in module add\_chembl\_compound\_properties)@\spxentry{add\_first\_publication\_date()}\spxextra{in module add\_chembl\_compound\_properties}}

\begin{fulllineitems}
\phantomsection\label{\detokenize{add_chembl_compound_properties:add_chembl_compound_properties.add_first_publication_date}}
\pysigstartsignatures
\pysiglinewithargsret{\sphinxcode{\sphinxupquote{add\_chembl\_compound\_properties.}}\sphinxbfcode{\sphinxupquote{add\_first\_publication\_date}}}{\sphinxparam{\DUrole{n}{df\_combined}\DUrole{p}{:}\DUrole{w}{ }\DUrole{n}{DataFrame}}\sphinxparamcomma \sphinxparam{\DUrole{n}{chembl\_con}\DUrole{p}{:}\DUrole{w}{ }\DUrole{n}{Connection}}\sphinxparamcomma \sphinxparam{\DUrole{n}{limit\_to\_literature}\DUrole{p}{:}\DUrole{w}{ }\DUrole{n}{bool}}}{{ $\rightarrow$ DataFrame}}
\pysigstopsignatures
\sphinxAtStartPar
Query and calculate the first publication of a compound
based on ChEMBL data (column name: first\_publication\_cpd).
If limit\_to\_literature is True, this corresponds to the first appearance
of the compound in the literature according to ChEMBL.
Otherwise this is the first appearance in any source in ChEMBL.
\begin{quote}\begin{description}
\sphinxlineitem{Parameters}\begin{itemize}
\item {} 
\sphinxAtStartPar
\sphinxstyleliteralstrong{\sphinxupquote{df\_combined}} (\sphinxstyleliteralemphasis{\sphinxupquote{pd.DataFrame}}) \textendash{} Pandas DataFrame with compound\sphinxhyphen{}target pairs

\item {} 
\sphinxAtStartPar
\sphinxstyleliteralstrong{\sphinxupquote{chembl\_con}} (\sphinxstyleliteralemphasis{\sphinxupquote{sqlite3.Connection}}) \textendash{} Sqlite3 connection to ChEMBL database.

\item {} 
\sphinxAtStartPar
\sphinxstyleliteralstrong{\sphinxupquote{limit\_to\_literature}} (\sphinxstyleliteralemphasis{\sphinxupquote{bool}}) \textendash{} Base first\_publication\_cpd on literature sources only if True.

\end{itemize}

\sphinxlineitem{Returns}
\sphinxAtStartPar
Pandas DataFrame with added first\_publication\_cpd.

\sphinxlineitem{Return type}
\sphinxAtStartPar
pd.DataFrame

\end{description}\end{quote}

\end{fulllineitems}

\index{add\_ligand\_efficiency\_metrics() (in module add\_chembl\_compound\_properties)@\spxentry{add\_ligand\_efficiency\_metrics()}\spxextra{in module add\_chembl\_compound\_properties}}

\begin{fulllineitems}
\phantomsection\label{\detokenize{add_chembl_compound_properties:add_chembl_compound_properties.add_ligand_efficiency_metrics}}
\pysigstartsignatures
\pysiglinewithargsret{\sphinxcode{\sphinxupquote{add\_chembl\_compound\_properties.}}\sphinxbfcode{\sphinxupquote{add\_ligand\_efficiency\_metrics}}}{\sphinxparam{\DUrole{n}{df\_combined}\DUrole{p}{:}\DUrole{w}{ }\DUrole{n}{DataFrame}}}{{ $\rightarrow$ DataFrame}}
\pysigstopsignatures
\sphinxAtStartPar
Calculate the ligand efficiency metrics for the compounds
based on the mean pchembl values for a compound\sphinxhyphen{}target pair and
the following ligand efficiency (LE) formulas:
\begin{align*}\!\begin{aligned}
LE &= \frac{\Delta G}{HA}
    \qquad \qquad \text{where } \Delta G = - RT \ln(K_d)
    \text{, } - RT\ln(K_i)
    \text{,  or} - RT\ln(IC_{50})\\
LE &= \frac{2.303 \cdot 298 \cdot 0.00199 \cdot pchembl \_ value} {heavy \_ atoms}\\
BEI &= \frac{pchembl \_ mean \cdot 1000}{mw \_ freebase}\\
SEI &= \frac{pchembl \_ mean \cdot 100}{PSA}\\
LLE &= pchembl \_ mean - ALOGP\\
\end{aligned}\end{align*}
\sphinxAtStartPar
Since LE metrics are based on pchembl values, they are calculated twice.
Once for the pchembl values based on binding + functional assays (BF)
and once for the pchembl values based on binding assays only (B).
\begin{quote}\begin{description}
\sphinxlineitem{Parameters}
\sphinxAtStartPar
\sphinxstyleliteralstrong{\sphinxupquote{df\_combined}} (\sphinxstyleliteralemphasis{\sphinxupquote{pd.DataFrame}}) \textendash{} Pandas DataFrame with compound\sphinxhyphen{}target pairs

\sphinxlineitem{Returns}
\sphinxAtStartPar
Pandas DataFrame with added ligand efficiency metrics

\sphinxlineitem{Return type}
\sphinxAtStartPar
pd.DataFrame

\end{description}\end{quote}

\end{fulllineitems}


\sphinxstepscope


\section{add\_chembl\_target\_class\_annotations module}
\label{\detokenize{add_chembl_target_class_annotations:module-add_chembl_target_class_annotations}}\label{\detokenize{add_chembl_target_class_annotations:add-chembl-target-class-annotations-module}}\label{\detokenize{add_chembl_target_class_annotations::doc}}\index{module@\spxentry{module}!add\_chembl\_target\_class\_annotations@\spxentry{add\_chembl\_target\_class\_annotations}}\index{add\_chembl\_target\_class\_annotations@\spxentry{add\_chembl\_target\_class\_annotations}!module@\spxentry{module}}\index{add\_chembl\_target\_class\_annotations() (in module add\_chembl\_target\_class\_annotations)@\spxentry{add\_chembl\_target\_class\_annotations()}\spxextra{in module add\_chembl\_target\_class\_annotations}}

\begin{fulllineitems}
\phantomsection\label{\detokenize{add_chembl_target_class_annotations:add_chembl_target_class_annotations.add_chembl_target_class_annotations}}
\pysigstartsignatures
\pysiglinewithargsret{\sphinxcode{\sphinxupquote{add\_chembl\_target\_class\_annotations.}}\sphinxbfcode{\sphinxupquote{add\_chembl\_target\_class\_annotations}}}{\sphinxparam{\DUrole{n}{df\_combined}\DUrole{p}{:}\DUrole{w}{ }\DUrole{n}{DataFrame}}\sphinxparamcomma \sphinxparam{\DUrole{n}{chembl\_con}\DUrole{p}{:}\DUrole{w}{ }\DUrole{n}{Connection}}\sphinxparamcomma \sphinxparam{\DUrole{n}{output\_path}\DUrole{p}{:}\DUrole{w}{ }\DUrole{n}{str}}\sphinxparamcomma \sphinxparam{\DUrole{n}{write\_to\_csv}\DUrole{p}{:}\DUrole{w}{ }\DUrole{n}{bool}}\sphinxparamcomma \sphinxparam{\DUrole{n}{write\_to\_excel}\DUrole{p}{:}\DUrole{w}{ }\DUrole{n}{bool}}\sphinxparamcomma \sphinxparam{\DUrole{n}{delimiter}\DUrole{p}{:}\DUrole{w}{ }\DUrole{n}{str}}\sphinxparamcomma \sphinxparam{\DUrole{n}{chembl\_version}\DUrole{p}{:}\DUrole{w}{ }\DUrole{n}{str}}\sphinxparamcomma \sphinxparam{\DUrole{n}{limited\_flag}\DUrole{p}{:}\DUrole{w}{ }\DUrole{n}{str}}}{{ $\rightarrow$ tuple\DUrole{p}{{[}}DataFrame\DUrole{p}{,}\DUrole{w}{ }DataFrame\DUrole{p}{,}\DUrole{w}{ }DataFrame\DUrole{p}{{]}}}}
\pysigstopsignatures
\sphinxAtStartPar
Add level 1 and 2 target class annotations.
Assignments for target IDs with more than one target class assignment per level
are summarised into one string with ‘|’ as a separator
between the different target class annotations.

\sphinxAtStartPar
Targets with more than one level 1 / level 2 target class assignment are written to a file.
These could be reassigned by hand if a single target class is preferable.
\begin{quote}\begin{description}
\sphinxlineitem{Parameters}\begin{itemize}
\item {} 
\sphinxAtStartPar
\sphinxstyleliteralstrong{\sphinxupquote{df\_combined}} (\sphinxstyleliteralemphasis{\sphinxupquote{pd.DataFrame}}) \textendash{} Pandas DataFrame with compound\sphinxhyphen{}target pairs

\item {} 
\sphinxAtStartPar
\sphinxstyleliteralstrong{\sphinxupquote{chembl\_con}} (\sphinxstyleliteralemphasis{\sphinxupquote{sqlite3.Connection}}) \textendash{} Sqlite3 connection to ChEMBL database.

\item {} 
\sphinxAtStartPar
\sphinxstyleliteralstrong{\sphinxupquote{output\_path}} (\sphinxstyleliteralemphasis{\sphinxupquote{str}}) \textendash{} Path to write the targets with more than one target class assignment to

\item {} 
\sphinxAtStartPar
\sphinxstyleliteralstrong{\sphinxupquote{write\_to\_csv}} (\sphinxstyleliteralemphasis{\sphinxupquote{bool}}) \textendash{} True if output should be written to csv

\item {} 
\sphinxAtStartPar
\sphinxstyleliteralstrong{\sphinxupquote{write\_to\_excel}} (\sphinxstyleliteralemphasis{\sphinxupquote{bool}}) \textendash{} True if output should be written to excel

\item {} 
\sphinxAtStartPar
\sphinxstyleliteralstrong{\sphinxupquote{delimiter}} (\sphinxstyleliteralemphasis{\sphinxupquote{str}}) \textendash{} Delimiter in csv\sphinxhyphen{}output

\item {} 
\sphinxAtStartPar
\sphinxstyleliteralstrong{\sphinxupquote{chembl\_version}} (\sphinxstyleliteralemphasis{\sphinxupquote{str}}) \textendash{} Version of ChEMBL for output files

\item {} 
\sphinxAtStartPar
\sphinxstyleliteralstrong{\sphinxupquote{limited\_flag}} (\sphinxstyleliteralemphasis{\sphinxupquote{str}}) \textendash{} Document suffix indicating
whether the dataset was limited to literature sources

\end{itemize}

\sphinxlineitem{Returns}
\sphinxAtStartPar
\begin{itemize}
\item {} 
\sphinxAtStartPar
Pandas DataFrame with added target class annotations 

\item {} 
\sphinxAtStartPar
Pandas DataFrame with mapping from target id to level 1 target class 

\item {} 
\sphinxAtStartPar
Pandas DataFrame with mapping from target id to level 2 target class

\end{itemize}


\sphinxlineitem{Return type}
\sphinxAtStartPar
(pd.DataFrame, pd.DataFrame, pd.DataFrame)

\end{description}\end{quote}

\end{fulllineitems}

\index{get\_target\_class\_table() (in module add\_chembl\_target\_class\_annotations)@\spxentry{get\_target\_class\_table()}\spxextra{in module add\_chembl\_target\_class\_annotations}}

\begin{fulllineitems}
\phantomsection\label{\detokenize{add_chembl_target_class_annotations:add_chembl_target_class_annotations.get_target_class_table}}
\pysigstartsignatures
\pysiglinewithargsret{\sphinxcode{\sphinxupquote{add\_chembl\_target\_class\_annotations.}}\sphinxbfcode{\sphinxupquote{get\_target\_class\_table}}}{\sphinxparam{\DUrole{n}{chembl\_con}\DUrole{p}{:}\DUrole{w}{ }\DUrole{n}{Connection}}\sphinxparamcomma \sphinxparam{\DUrole{n}{current\_tids}\DUrole{p}{:}\DUrole{w}{ }\DUrole{n}{set\DUrole{p}{{[}}int\DUrole{p}{{]}}}}}{{ $\rightarrow$ DataFrame}}
\pysigstopsignatures
\sphinxAtStartPar
Get level 1 and level 2 target class annotations in ChEMBL.
\begin{quote}\begin{description}
\sphinxlineitem{Parameters}\begin{itemize}
\item {} 
\sphinxAtStartPar
\sphinxstyleliteralstrong{\sphinxupquote{chembl\_con}} (\sphinxstyleliteralemphasis{\sphinxupquote{sqlite3.Connection}}) \textendash{} Sqlite3 connection to ChEMBL database.

\item {} 
\sphinxAtStartPar
\sphinxstyleliteralstrong{\sphinxupquote{current\_tids}} (\sphinxstyleliteralemphasis{\sphinxupquote{set}}\sphinxstyleliteralemphasis{\sphinxupquote{{[}}}\sphinxstyleliteralemphasis{\sphinxupquote{int}}\sphinxstyleliteralemphasis{\sphinxupquote{{]}}}) \textendash{} Set of target ids to take into account

\end{itemize}

\sphinxlineitem{Returns}
\sphinxAtStartPar
Pandas DataFrame with target class information

\sphinxlineitem{Return type}
\sphinxAtStartPar
pd.DataFrame

\end{description}\end{quote}

\end{fulllineitems}


\sphinxstepscope


\section{add\_dti\_annotations module}
\label{\detokenize{add_dti_annotations:module-add_dti_annotations}}\label{\detokenize{add_dti_annotations:add-dti-annotations-module}}\label{\detokenize{add_dti_annotations::doc}}\index{module@\spxentry{module}!add\_dti\_annotations@\spxentry{add\_dti\_annotations}}\index{add\_dti\_annotations@\spxentry{add\_dti\_annotations}!module@\spxentry{module}}\index{add\_dti\_annotations() (in module add\_dti\_annotations)@\spxentry{add\_dti\_annotations()}\spxextra{in module add\_dti\_annotations}}

\begin{fulllineitems}
\phantomsection\label{\detokenize{add_dti_annotations:add_dti_annotations.add_dti_annotations}}
\pysigstartsignatures
\pysiglinewithargsret{\sphinxcode{\sphinxupquote{add\_dti\_annotations.}}\sphinxbfcode{\sphinxupquote{add\_dti\_annotations}}}{\sphinxparam{\DUrole{n}{df\_combined}\DUrole{p}{:}\DUrole{w}{ }\DUrole{n}{DataFrame}}\sphinxparamcomma \sphinxparam{\DUrole{n}{drug\_mechanism\_pairs\_set}\DUrole{p}{:}\DUrole{w}{ }\DUrole{n}{set}}\sphinxparamcomma \sphinxparam{\DUrole{n}{drug\_mechanism\_targets\_set}\DUrole{p}{:}\DUrole{w}{ }\DUrole{n}{set}}}{{ $\rightarrow$ DataFrame}}
\pysigstopsignatures
\sphinxAtStartPar
Every compound\sphinxhyphen{}target pair is assigned a DTI (drug target interaction) annotation.

\sphinxAtStartPar
The assignment is based on three questions:
\begin{itemize}
\item {} \begin{description}
\sphinxlineitem{Is the compound\sphinxhyphen{}target pair in the drug\_mechanisms table? =}
\sphinxAtStartPar
Is it a known relevant compound\sphinxhyphen{}target interaction?

\end{description}

\item {} 
\sphinxAtStartPar
What is the max\_phase of the compound? = Is it a drug / clinical compound?

\item {} 
\sphinxAtStartPar
Is the target in the drug\_mechanisms table = Is it a therapeutic target?

\end{itemize}

\sphinxAtStartPar
The assigments are based on the following table:


\begin{savenotes}\sphinxattablestart
\sphinxthistablewithglobalstyle
\centering
\begin{tabulary}{\linewidth}[t]{TTTTT}
\sphinxtoprule
\sphinxstyletheadfamily 
\sphinxAtStartPar
in DM table?
&\sphinxstyletheadfamily 
\sphinxAtStartPar
max\_phase?
&\sphinxstyletheadfamily 
\sphinxAtStartPar
th. target?
&\sphinxstyletheadfamily 
\sphinxAtStartPar
DTI
&\sphinxstyletheadfamily 
\sphinxAtStartPar
explanation
\\
\sphinxmidrule
\sphinxtableatstartofbodyhook
\sphinxAtStartPar
yes
&
\sphinxAtStartPar
4
&
\sphinxAtStartPar
\textendash{}
&
\sphinxAtStartPar
D\_DT %
\begin{footnote}[1]\sphinxAtStartFootnote
The annotation D\_DT instead of C4\_DT was chosen to be consistent     with the annotations in a previous version of the dataset.     For the same reason the column is named DTI (drug\sphinxhyphen{}target interaction)     instead of CTI (compound\sphinxhyphen{}target interaction)     despite having specific annotations for clinical canidates.
%
\end{footnote}
&
\sphinxAtStartPar
drug \sphinxhyphen{} drug target
\\
\sphinxhline
\sphinxAtStartPar
yes
&
\sphinxAtStartPar
3
&
\sphinxAtStartPar
\textendash{}
&
\sphinxAtStartPar
C3\_DT
&
\sphinxAtStartPar
clinical candidate in phase 3 \sphinxhyphen{} drug target
\\
\sphinxhline
\sphinxAtStartPar
yes
&
\sphinxAtStartPar
2
&
\sphinxAtStartPar
\textendash{}
&
\sphinxAtStartPar
C2\_DT
&
\sphinxAtStartPar
clinical candidate in phase 2 \sphinxhyphen{} drug target
\\
\sphinxhline
\sphinxAtStartPar
yes
&
\sphinxAtStartPar
1
&
\sphinxAtStartPar
\textendash{}
&
\sphinxAtStartPar
C1\_DT
&
\sphinxAtStartPar
clinical candidate in phase 1 \sphinxhyphen{} drug target
\\
\sphinxhline
\sphinxAtStartPar
yes
&
\sphinxAtStartPar
\textless{}1
&
\sphinxAtStartPar
\textendash{}
&
\sphinxAtStartPar
C0\_DT
&
\sphinxAtStartPar
compound in unknown phase %
\begin{footnote}[2]\sphinxAtStartFootnote
C0\_DT groups together all compounds with a max\_phase not between 1 and 4.
%
\end{footnote} \sphinxhyphen{} drug target
\\
\sphinxhline
\sphinxAtStartPar
no
&
\sphinxAtStartPar
\textendash{}
&
\sphinxAtStartPar
yes
&
\sphinxAtStartPar
DT
&
\sphinxAtStartPar
drug target
\\
\sphinxhline
\sphinxAtStartPar
no
&
\sphinxAtStartPar
\textendash{}
&
\sphinxAtStartPar
no
&
\sphinxAtStartPar
NDT
&
\sphinxAtStartPar
not drug target
\\
\sphinxbottomrule
\end{tabulary}
\sphinxtableafterendhook\par
\sphinxattableend\end{savenotes}

\sphinxAtStartPar
Since ChEMBL32 there are three possible annotations in ChEMBL
with a max\_phase value not between 1 and 4:
\begin{itemize}
\item {} 
\sphinxAtStartPar
0.5 = early phase 1 clinical trials

\item {} \begin{description}
\sphinxlineitem{\sphinxhyphen{}1 = clinical phase unknown for drug or clinical candidate drug,}
\sphinxAtStartPar
i.e., where ChEMBL cannot assign a clinical phase

\end{description}

\item {} 
\sphinxAtStartPar
NULL = preclinical compounds with bioactivity data

\end{itemize}

\sphinxAtStartPar
All three are grouped together into the annotation C0\_DT.

\sphinxAtStartPar
Compound\sphinxhyphen{}target pairs that were annotated with NDT,
i.e., compound\sphinxhyphen{}target pairs that are not in the drug\_mechanisms table
and for which the target was also not in the drug\_mechanisms table
(not a comparator compound), are discarded.
\begin{quote}\begin{description}
\sphinxlineitem{Parameters}\begin{itemize}
\item {} 
\sphinxAtStartPar
\sphinxstyleliteralstrong{\sphinxupquote{df\_combined}} (\sphinxstyleliteralemphasis{\sphinxupquote{pd.DataFrame}}) \textendash{} Pandas DataFrame with compound\sphinxhyphen{}target pairs
based on activities AND drug\_mechanism table

\item {} 
\sphinxAtStartPar
\sphinxstyleliteralstrong{\sphinxupquote{drug\_mechanism\_pairs\_set}} (\sphinxstyleliteralemphasis{\sphinxupquote{set}}) \textendash{} set of compound\sphinxhyphen{}target pairs in the drug\_mechanism table

\item {} 
\sphinxAtStartPar
\sphinxstyleliteralstrong{\sphinxupquote{drug\_mechanism\_targets\_set}} (\sphinxstyleliteralemphasis{\sphinxupquote{set}}) \textendash{} set of targets in the drug\_mechanism table

\end{itemize}

\sphinxlineitem{Returns}
\sphinxAtStartPar
Pandas DataFrame with all compound\sphinxhyphen{}target pairs and their DTI annotations.

\sphinxlineitem{Return type}
\sphinxAtStartPar
pd.DataFrame

\end{description}\end{quote}

\end{fulllineitems}


\sphinxstepscope


\section{add\_rdkit\_compound\_descriptors module}
\label{\detokenize{add_rdkit_compound_descriptors:module-add_rdkit_compound_descriptors}}\label{\detokenize{add_rdkit_compound_descriptors:add-rdkit-compound-descriptors-module}}\label{\detokenize{add_rdkit_compound_descriptors::doc}}\index{module@\spxentry{module}!add\_rdkit\_compound\_descriptors@\spxentry{add\_rdkit\_compound\_descriptors}}\index{add\_rdkit\_compound\_descriptors@\spxentry{add\_rdkit\_compound\_descriptors}!module@\spxentry{module}}\index{add\_aromaticity\_descriptors() (in module add\_rdkit\_compound\_descriptors)@\spxentry{add\_aromaticity\_descriptors()}\spxextra{in module add\_rdkit\_compound\_descriptors}}

\begin{fulllineitems}
\phantomsection\label{\detokenize{add_rdkit_compound_descriptors:add_rdkit_compound_descriptors.add_aromaticity_descriptors}}
\pysigstartsignatures
\pysiglinewithargsret{\sphinxcode{\sphinxupquote{add\_rdkit\_compound\_descriptors.}}\sphinxbfcode{\sphinxupquote{add\_aromaticity\_descriptors}}}{\sphinxparam{\DUrole{n}{df\_combined}\DUrole{p}{:}\DUrole{w}{ }\DUrole{n}{DataFrame}}}{{ $\rightarrow$ DataFrame}}
\pysigstopsignatures
\sphinxAtStartPar
Add number of aromatic atoms in a compounds, specifically:
\begin{itemize}
\item {} 
\sphinxAtStartPar
total \# aromatics atoms (aromatic\_atoms)

\item {} 
\sphinxAtStartPar
\# aromatic carbon atoms (aromatic\_c)

\item {} 
\sphinxAtStartPar
\# aromatic nitrogen atoms (aromatic\_n)

\item {} 
\sphinxAtStartPar
\# aromatic hetero atoms (aromatic\_hetero)

\end{itemize}
\begin{quote}\begin{description}
\sphinxlineitem{Parameters}
\sphinxAtStartPar
\sphinxstyleliteralstrong{\sphinxupquote{df\_combined}} (\sphinxstyleliteralemphasis{\sphinxupquote{pd.DataFrame}}) \textendash{} Pandas DataFrame with compound\sphinxhyphen{}target pairs

\sphinxlineitem{Returns}
\sphinxAtStartPar
Pandas DataFrame with added counts of aromatic atoms

\sphinxlineitem{Return type}
\sphinxAtStartPar
pd.DataFrame

\end{description}\end{quote}

\end{fulllineitems}

\index{add\_built\_in\_descriptors() (in module add\_rdkit\_compound\_descriptors)@\spxentry{add\_built\_in\_descriptors()}\spxextra{in module add\_rdkit\_compound\_descriptors}}

\begin{fulllineitems}
\phantomsection\label{\detokenize{add_rdkit_compound_descriptors:add_rdkit_compound_descriptors.add_built_in_descriptors}}
\pysigstartsignatures
\pysiglinewithargsret{\sphinxcode{\sphinxupquote{add\_rdkit\_compound\_descriptors.}}\sphinxbfcode{\sphinxupquote{add\_built\_in\_descriptors}}}{\sphinxparam{\DUrole{n}{df\_combined}\DUrole{p}{:}\DUrole{w}{ }\DUrole{n}{DataFrame}}}{{ $\rightarrow$ DataFrame}}
\pysigstopsignatures
\sphinxAtStartPar
Add RDKit built\sphinxhyphen{}in compound descriptors.
\begin{quote}\begin{description}
\sphinxlineitem{Parameters}
\sphinxAtStartPar
\sphinxstyleliteralstrong{\sphinxupquote{df\_combined}} (\sphinxstyleliteralemphasis{\sphinxupquote{pd.DataFrame}}) \textendash{} Pandas DataFrame with compound\sphinxhyphen{}target pairs

\sphinxlineitem{Returns}
\sphinxAtStartPar
Pandas DataFrame with added built\sphinxhyphen{}in RDKit compound descriptors

\sphinxlineitem{Return type}
\sphinxAtStartPar
pd.DataFrame

\end{description}\end{quote}

\end{fulllineitems}

\index{add\_rdkit\_compound\_descriptors() (in module add\_rdkit\_compound\_descriptors)@\spxentry{add\_rdkit\_compound\_descriptors()}\spxextra{in module add\_rdkit\_compound\_descriptors}}

\begin{fulllineitems}
\phantomsection\label{\detokenize{add_rdkit_compound_descriptors:add_rdkit_compound_descriptors.add_rdkit_compound_descriptors}}
\pysigstartsignatures
\pysiglinewithargsret{\sphinxcode{\sphinxupquote{add\_rdkit\_compound\_descriptors.}}\sphinxbfcode{\sphinxupquote{add\_rdkit\_compound\_descriptors}}}{\sphinxparam{\DUrole{n}{df\_combined}\DUrole{p}{:}\DUrole{w}{ }\DUrole{n}{DataFrame}}}{{ $\rightarrow$ DataFrame}}
\pysigstopsignatures
\sphinxAtStartPar
Add RDKit\sphinxhyphen{}based compound descriptors (built\sphinxhyphen{}in and numbers of aromatic atoms).
\begin{quote}\begin{description}
\sphinxlineitem{Parameters}
\sphinxAtStartPar
\sphinxstyleliteralstrong{\sphinxupquote{df\_combined}} (\sphinxstyleliteralemphasis{\sphinxupquote{pd.DataFrame}}) \textendash{} Pandas DataFrame with compound\sphinxhyphen{}target pairs

\sphinxlineitem{Returns}
\sphinxAtStartPar
Pandas DataFrame with added built\sphinxhyphen{}in RDKit compound descriptors
and numbers of aromatic atoms

\sphinxlineitem{Return type}
\sphinxAtStartPar
pd.DataFrame

\end{description}\end{quote}

\end{fulllineitems}

\index{calculate\_aromatic\_atoms() (in module add\_rdkit\_compound\_descriptors)@\spxentry{calculate\_aromatic\_atoms()}\spxextra{in module add\_rdkit\_compound\_descriptors}}

\begin{fulllineitems}
\phantomsection\label{\detokenize{add_rdkit_compound_descriptors:add_rdkit_compound_descriptors.calculate_aromatic_atoms}}
\pysigstartsignatures
\pysiglinewithargsret{\sphinxcode{\sphinxupquote{add\_rdkit\_compound\_descriptors.}}\sphinxbfcode{\sphinxupquote{calculate\_aromatic\_atoms}}}{\sphinxparam{\DUrole{n}{smiles\_set}\DUrole{p}{:}\DUrole{w}{ }\DUrole{n}{set\DUrole{p}{{[}}str\DUrole{p}{{]}}}}}{{ $\rightarrow$ tuple\DUrole{p}{{[}}dict\DUrole{p}{{[}}str\DUrole{p}{,}\DUrole{w}{ }int\DUrole{p}{{]}}\DUrole{p}{,}\DUrole{w}{ }dict\DUrole{p}{{[}}str\DUrole{p}{,}\DUrole{w}{ }int\DUrole{p}{{]}}\DUrole{p}{,}\DUrole{w}{ }dict\DUrole{p}{{[}}str\DUrole{p}{,}\DUrole{w}{ }int\DUrole{p}{{]}}\DUrole{p}{,}\DUrole{w}{ }dict\DUrole{p}{{[}}str\DUrole{p}{,}\DUrole{w}{ }int\DUrole{p}{{]}}\DUrole{p}{{]}}}}
\pysigstopsignatures
\sphinxAtStartPar
Get dictionaries with number of aromatic atoms for each smiles.
\begin{quote}\begin{description}
\sphinxlineitem{Parameters}
\sphinxAtStartPar
\sphinxstyleliteralstrong{\sphinxupquote{smiles\_set}} (\sphinxstyleliteralemphasis{\sphinxupquote{set}}\sphinxstyleliteralemphasis{\sphinxupquote{{[}}}\sphinxstyleliteralemphasis{\sphinxupquote{str}}\sphinxstyleliteralemphasis{\sphinxupquote{{]}}}) \textendash{} Set of smiles to calculate the number of aromatic atoms for

\sphinxlineitem{Returns}
\sphinxAtStartPar

\sphinxAtStartPar
Dictionaries with:
\begin{itemize}
\item {} 
\sphinxAtStartPar
SMILES \sphinxhyphen{}\textgreater{} \# aromatics atoms

\item {} 
\sphinxAtStartPar
SMILES \sphinxhyphen{}\textgreater{} \# aromatic carbon atoms

\item {} 
\sphinxAtStartPar
SMILES \sphinxhyphen{}\textgreater{} \# aromatic nitrogen atoms

\item {} 
\sphinxAtStartPar
SMILES \sphinxhyphen{}\textgreater{} \# aromatic hetero atoms

\end{itemize}


\sphinxlineitem{Return type}
\sphinxAtStartPar
(dict{[}str, int{]}, dict{[}str, int{]}, dict{[}str, int{]}, dict{[}str, int{]})

\end{description}\end{quote}

\end{fulllineitems}


\sphinxstepscope


\section{clean\_dataset module}
\label{\detokenize{clean_dataset:module-clean_dataset}}\label{\detokenize{clean_dataset:clean-dataset-module}}\label{\detokenize{clean_dataset::doc}}\index{module@\spxentry{module}!clean\_dataset@\spxentry{clean\_dataset}}\index{clean\_dataset@\spxentry{clean\_dataset}!module@\spxentry{module}}\index{clean\_dataset() (in module clean\_dataset)@\spxentry{clean\_dataset()}\spxextra{in module clean\_dataset}}

\begin{fulllineitems}
\phantomsection\label{\detokenize{clean_dataset:clean_dataset.clean_dataset}}
\pysigstartsignatures
\pysiglinewithargsret{\sphinxcode{\sphinxupquote{clean\_dataset.}}\sphinxbfcode{\sphinxupquote{clean\_dataset}}}{\sphinxparam{\DUrole{n}{df\_combined}\DUrole{p}{:}\DUrole{w}{ }\DUrole{n}{DataFrame}}\sphinxparamcomma \sphinxparam{\DUrole{n}{calculate\_rdkit}\DUrole{p}{:}\DUrole{w}{ }\DUrole{n}{bool}}}{{ $\rightarrow$ DataFrame}}
\pysigstopsignatures
\sphinxAtStartPar
Clean the dataset by
\begin{itemize}
\item {} 
\sphinxAtStartPar
changing nan values and empty strings to None

\item {} 
\sphinxAtStartPar
setting the type of relevant columns to Int64

\item {} 
\sphinxAtStartPar
rounding floats to 4 decimal places (with the exception of max\_phase which is not rounded)

\item {} 
\sphinxAtStartPar
reordering columns

\item {} 
\sphinxAtStartPar
sorting rows by cpd\_target\_pair\_mutation

\end{itemize}
\begin{quote}\begin{description}
\sphinxlineitem{Parameters}\begin{itemize}
\item {} 
\sphinxAtStartPar
\sphinxstyleliteralstrong{\sphinxupquote{df\_combined}} (\sphinxstyleliteralemphasis{\sphinxupquote{pd.DataFrame}}) \textendash{} Pandas DataFrame with compound\sphinxhyphen{}target pairs

\item {} 
\sphinxAtStartPar
\sphinxstyleliteralstrong{\sphinxupquote{calculate\_rdkit}} (\sphinxstyleliteralemphasis{\sphinxupquote{bool}}) \textendash{} True if the DataFrame contains RDKit\sphinxhyphen{}based compound properties

\end{itemize}

\sphinxlineitem{Returns}
\sphinxAtStartPar
Cleaned pandas DataFrame with compound\sphinxhyphen{}target pairs

\sphinxlineitem{Return type}
\sphinxAtStartPar
pd.DataFrame

\end{description}\end{quote}

\end{fulllineitems}

\index{clean\_none\_values() (in module clean\_dataset)@\spxentry{clean\_none\_values()}\spxextra{in module clean\_dataset}}

\begin{fulllineitems}
\phantomsection\label{\detokenize{clean_dataset:clean_dataset.clean_none_values}}
\pysigstartsignatures
\pysiglinewithargsret{\sphinxcode{\sphinxupquote{clean\_dataset.}}\sphinxbfcode{\sphinxupquote{clean\_none\_values}}}{\sphinxparam{\DUrole{n}{df\_combined}}}{}
\pysigstopsignatures
\sphinxAtStartPar
Change nan values and empty strings to None for consistency.

\end{fulllineitems}

\index{remove\_compounds\_without\_smiles\_and\_mixtures() (in module clean\_dataset)@\spxentry{remove\_compounds\_without\_smiles\_and\_mixtures()}\spxextra{in module clean\_dataset}}

\begin{fulllineitems}
\phantomsection\label{\detokenize{clean_dataset:clean_dataset.remove_compounds_without_smiles_and_mixtures}}
\pysigstartsignatures
\pysiglinewithargsret{\sphinxcode{\sphinxupquote{clean\_dataset.}}\sphinxbfcode{\sphinxupquote{remove\_compounds\_without\_smiles\_and\_mixtures}}}{\sphinxparam{\DUrole{n}{df\_combined}\DUrole{p}{:}\DUrole{w}{ }\DUrole{n}{DataFrame}}\sphinxparamcomma \sphinxparam{\DUrole{n}{chembl\_con}\DUrole{p}{:}\DUrole{w}{ }\DUrole{n}{Connection}}}{{ $\rightarrow$ DataFrame}}
\pysigstopsignatures
\sphinxAtStartPar
Remove
\begin{itemize}
\item {} 
\sphinxAtStartPar
compounds without a smiles

\item {} 
\sphinxAtStartPar
compounds with smiles containing a dot (mixtures and salts).

\end{itemize}

\sphinxAtStartPar
Since compound information is aggregated for the parents of salts,
the number of smiles with a dot is relatively low.
\begin{quote}\begin{description}
\sphinxlineitem{Parameters}\begin{itemize}
\item {} 
\sphinxAtStartPar
\sphinxstyleliteralstrong{\sphinxupquote{df\_combined}} (\sphinxstyleliteralemphasis{\sphinxupquote{pd.DataFrame}}) \textendash{} Pandas DataFrame with compound\sphinxhyphen{}target pairs

\item {} 
\sphinxAtStartPar
\sphinxstyleliteralstrong{\sphinxupquote{chembl\_con}} (\sphinxstyleliteralemphasis{\sphinxupquote{sqlite3.Connection}}) \textendash{} Sqlite3 connection to ChEMBL database.

\end{itemize}

\sphinxlineitem{Returns}
\sphinxAtStartPar
Pandas DataFrame with compound\sphinxhyphen{}target pairs with a smiles that does not contain a ‘.’

\sphinxlineitem{Return type}
\sphinxAtStartPar
pd.DataFrame

\end{description}\end{quote}

\end{fulllineitems}

\index{reorder\_columns() (in module clean\_dataset)@\spxentry{reorder\_columns()}\spxextra{in module clean\_dataset}}

\begin{fulllineitems}
\phantomsection\label{\detokenize{clean_dataset:clean_dataset.reorder_columns}}
\pysigstartsignatures
\pysiglinewithargsret{\sphinxcode{\sphinxupquote{clean\_dataset.}}\sphinxbfcode{\sphinxupquote{reorder\_columns}}}{\sphinxparam{\DUrole{n}{df\_combined}}\sphinxparamcomma \sphinxparam{\DUrole{n}{calculate\_rdkit}}}{}
\pysigstopsignatures
\sphinxAtStartPar
Reorder the columns in the DataFrame.

\end{fulllineitems}

\index{round\_floats() (in module clean\_dataset)@\spxentry{round\_floats()}\spxextra{in module clean\_dataset}}

\begin{fulllineitems}
\phantomsection\label{\detokenize{clean_dataset:clean_dataset.round_floats}}
\pysigstartsignatures
\pysiglinewithargsret{\sphinxcode{\sphinxupquote{clean\_dataset.}}\sphinxbfcode{\sphinxupquote{round\_floats}}}{\sphinxparam{\DUrole{n}{df\_combined}}\sphinxparamcomma \sphinxparam{\DUrole{n}{decimal\_places}\DUrole{o}{=}\DUrole{default_value}{4}}}{}
\pysigstopsignatures
\sphinxAtStartPar
Round float columns to \textless{}decimal\_places\textgreater{} decimal places.
This does not apply to max\_phase.

\end{fulllineitems}

\index{set\_types\_to\_int() (in module clean\_dataset)@\spxentry{set\_types\_to\_int()}\spxextra{in module clean\_dataset}}

\begin{fulllineitems}
\phantomsection\label{\detokenize{clean_dataset:clean_dataset.set_types_to_int}}
\pysigstartsignatures
\pysiglinewithargsret{\sphinxcode{\sphinxupquote{clean\_dataset.}}\sphinxbfcode{\sphinxupquote{set\_types\_to\_int}}}{\sphinxparam{\DUrole{n}{df\_combined}}\sphinxparamcomma \sphinxparam{\DUrole{n}{calculate\_rdkit}}}{}
\pysigstopsignatures
\sphinxAtStartPar
Set the type of relevant columns to Int64.

\end{fulllineitems}


\sphinxstepscope


\section{get\_activity\_ct\_pairs module}
\label{\detokenize{get_activity_ct_pairs:module-get_activity_ct_pairs}}\label{\detokenize{get_activity_ct_pairs:get-activity-ct-pairs-module}}\label{\detokenize{get_activity_ct_pairs::doc}}\index{module@\spxentry{module}!get\_activity\_ct\_pairs@\spxentry{get\_activity\_ct\_pairs}}\index{get\_activity\_ct\_pairs@\spxentry{get\_activity\_ct\_pairs}!module@\spxentry{module}}\index{get\_aggregated\_activity\_ct\_pairs() (in module get\_activity\_ct\_pairs)@\spxentry{get\_aggregated\_activity\_ct\_pairs()}\spxextra{in module get\_activity\_ct\_pairs}}

\begin{fulllineitems}
\phantomsection\label{\detokenize{get_activity_ct_pairs:get_activity_ct_pairs.get_aggregated_activity_ct_pairs}}
\pysigstartsignatures
\pysiglinewithargsret{\sphinxcode{\sphinxupquote{get\_activity\_ct\_pairs.}}\sphinxbfcode{\sphinxupquote{get\_aggregated\_activity\_ct\_pairs}}}{\sphinxparam{\DUrole{n}{chembl\_con}\DUrole{p}{:}\DUrole{w}{ }\DUrole{n}{Connection}}\sphinxparamcomma \sphinxparam{\DUrole{n}{limit\_to\_literature}\DUrole{p}{:}\DUrole{w}{ }\DUrole{n}{bool}}\sphinxparamcomma \sphinxparam{\DUrole{n}{df\_sizes}\DUrole{p}{:}\DUrole{w}{ }\DUrole{n}{list\DUrole{p}{{[}}list\DUrole{p}{{[}}int\DUrole{p}{{]}}\DUrole{p}{,}\DUrole{w}{ }list\DUrole{p}{{[}}int\DUrole{p}{{]}}\DUrole{p}{{]}}}}}{{ $\rightarrow$ DataFrame}}
\pysigstopsignatures
\sphinxAtStartPar
Get dataset of compound target\sphinxhyphen{}pairs with an associated pchembl value
with pchembl and publication dates aggregated into one entry per pair.

\sphinxAtStartPar
Values are aggregated for
\begin{itemize}
\item {} 
\sphinxAtStartPar
a subset of the initial dataset based on binding and functional assays (suffix ‘\_BF’) and

\item {} 
\sphinxAtStartPar
a subset of the initial dataset set on only binding assays (suffix ‘\_B’).

\end{itemize}

\sphinxAtStartPar
Therefore, there are two columns for pchembl\_value\_mean, \_max, \_median,
first\_publication\_cpd\_target\_pair and first\_publication\_cpd\_target\_pair\_w\_pchembl,
one with the suffix ‘\_BF’ based on binding + functional data
and one with the suffix ‘\_B’ based on only binding data.
\begin{quote}\begin{description}
\sphinxlineitem{Parameters}\begin{itemize}
\item {} 
\sphinxAtStartPar
\sphinxstyleliteralstrong{\sphinxupquote{chembl\_con}} (\sphinxstyleliteralemphasis{\sphinxupquote{sqlite3.Connection}}) \textendash{} Sqlite3 connection to ChEMBL database.

\item {} 
\sphinxAtStartPar
\sphinxstyleliteralstrong{\sphinxupquote{limit\_to\_literature}} (\sphinxstyleliteralemphasis{\sphinxupquote{bool}}) \textendash{} Include only literature sources if True.
Include all available sources otherwise.

\item {} 
\sphinxAtStartPar
\sphinxstyleliteralstrong{\sphinxupquote{df\_sizes}} (\sphinxstyleliteralemphasis{\sphinxupquote{list}}\sphinxstyleliteralemphasis{\sphinxupquote{{[}}}\sphinxstyleliteralemphasis{\sphinxupquote{list}}\sphinxstyleliteralemphasis{\sphinxupquote{{[}}}\sphinxstyleliteralemphasis{\sphinxupquote{int}}\sphinxstyleliteralemphasis{\sphinxupquote{{]}}}\sphinxstyleliteralemphasis{\sphinxupquote{, }}\sphinxstyleliteralemphasis{\sphinxupquote{list}}\sphinxstyleliteralemphasis{\sphinxupquote{{[}}}\sphinxstyleliteralemphasis{\sphinxupquote{int}}\sphinxstyleliteralemphasis{\sphinxupquote{{]}}}\sphinxstyleliteralemphasis{\sphinxupquote{{]}}}) \textendash{} List of intermediate sized of the dataset used for debugging.

\end{itemize}

\sphinxlineitem{Returns}
\sphinxAtStartPar
Pandas Dataframe with compound\sphinxhyphen{}target pairs based on ChEMBL activity data
aggregated into one entry per compound\sphinxhyphen{}target pair.

\sphinxlineitem{Return type}
\sphinxAtStartPar
pd.DataFrame

\end{description}\end{quote}

\end{fulllineitems}

\index{get\_average\_info() (in module get\_activity\_ct\_pairs)@\spxentry{get\_average\_info()}\spxextra{in module get\_activity\_ct\_pairs}}

\begin{fulllineitems}
\phantomsection\label{\detokenize{get_activity_ct_pairs:get_activity_ct_pairs.get_average_info}}
\pysigstartsignatures
\pysiglinewithargsret{\sphinxcode{\sphinxupquote{get\_activity\_ct\_pairs.}}\sphinxbfcode{\sphinxupquote{get\_average\_info}}}{\sphinxparam{\DUrole{n}{df}\DUrole{p}{:}\DUrole{w}{ }\DUrole{n}{DataFrame}}\sphinxparamcomma \sphinxparam{\DUrole{n}{suffix}\DUrole{p}{:}\DUrole{w}{ }\DUrole{n}{str}}}{{ $\rightarrow$ DataFrame}}
\pysigstopsignatures
\sphinxAtStartPar
Aggregate the information about compound\sphinxhyphen{}target pairs for which
there is more than one entry into one entry.
Compound\sphinxhyphen{}target pairs are considered equal if parent\_molregno (internal compound ID)
and tid\_mutation (target ID + mutation annotations) are equal.

\sphinxAtStartPar
The following values are aggregated:


\begin{savenotes}\sphinxattablestart
\sphinxthistablewithglobalstyle
\centering
\begin{tabulary}{\linewidth}[t]{TT}
\sphinxtoprule
\sphinxtableatstartofbodyhook
\sphinxAtStartPar
pchembl\_value\_mean
&
\sphinxAtStartPar
mean pchembl value for a compound\sphinxhyphen{}target pair
\\
\sphinxhline
\sphinxAtStartPar
pchembl\_value\_max
&
\sphinxAtStartPar
maximum pchembl value for a compound\sphinxhyphen{}target pair
\\
\sphinxhline
\sphinxAtStartPar
pchembl\_value\_median
&
\sphinxAtStartPar
median pchembl value for a compound\sphinxhyphen{}target pair
\\
\sphinxhline
\sphinxAtStartPar
first\_publication\_cpd\_target\_pair
&
\sphinxAtStartPar
first publication in ChEMBL with this compound\sphinxhyphen{}target pair
\\
\sphinxhline
\sphinxAtStartPar
first\_publication\_cpd\_target\_pair\_w\_pchembl
&
\sphinxAtStartPar
first publication in ChEMBL with this compound\sphinxhyphen{}target pair and an associated pchembl value
\\
\sphinxbottomrule
\end{tabulary}
\sphinxtableafterendhook\par
\sphinxattableend\end{savenotes}
\begin{quote}\begin{description}
\sphinxlineitem{Parameters}\begin{itemize}
\item {} 
\sphinxAtStartPar
\sphinxstyleliteralstrong{\sphinxupquote{df}} (\sphinxstyleliteralemphasis{\sphinxupquote{pd.DataFrame}}) \textendash{} Pandas DataFrame with compound\sphinxhyphen{}target pairs for which
the information should be aggregated.

\item {} 
\sphinxAtStartPar
\sphinxstyleliteralstrong{\sphinxupquote{suffix}} (\sphinxstyleliteralemphasis{\sphinxupquote{str}}) \textendash{} Suffix indicating the type of the given DataFrame,
e.g., \_B for binding assays, \_BF for binding+functional assays.

\end{itemize}

\sphinxlineitem{Returns}
\sphinxAtStartPar
Pandas DataFrame with ‘parent\_molregno’, ‘tid\_mutation’, and the aggregated columns.

\sphinxlineitem{Return type}
\sphinxAtStartPar
pd.DataFrame

\end{description}\end{quote}

\end{fulllineitems}

\index{get\_compound\_target\_pairs\_with\_pchembl() (in module get\_activity\_ct\_pairs)@\spxentry{get\_compound\_target\_pairs\_with\_pchembl()}\spxextra{in module get\_activity\_ct\_pairs}}

\begin{fulllineitems}
\phantomsection\label{\detokenize{get_activity_ct_pairs:get_activity_ct_pairs.get_compound_target_pairs_with_pchembl}}
\pysigstartsignatures
\pysiglinewithargsret{\sphinxcode{\sphinxupquote{get\_activity\_ct\_pairs.}}\sphinxbfcode{\sphinxupquote{get\_compound\_target\_pairs\_with\_pchembl}}}{\sphinxparam{\DUrole{n}{chembl\_con}\DUrole{p}{:}\DUrole{w}{ }\DUrole{n}{Connection}}\sphinxparamcomma \sphinxparam{\DUrole{n}{limit\_to\_literature}\DUrole{p}{:}\DUrole{w}{ }\DUrole{n}{bool}}\sphinxparamcomma \sphinxparam{\DUrole{n}{df\_sizes}\DUrole{p}{:}\DUrole{w}{ }\DUrole{n}{list\DUrole{p}{{[}}list\DUrole{p}{{[}}int\DUrole{p}{{]}}\DUrole{p}{,}\DUrole{w}{ }list\DUrole{p}{{[}}int\DUrole{p}{{]}}\DUrole{p}{{]}}}}}{{ $\rightarrow$ DataFrame}}
\pysigstopsignatures
\sphinxAtStartPar
Query ChEMBL activities and related assay for compound\sphinxhyphen{}target pairs
with an associated pchembl value.
Compound\sphinxhyphen{}target pairs are required to have a pchembl value.
Salt forms of compounds are mapped to their parent form.
If limit\_to\_literature is true, only literature sources will be considered.
Otherwise, all sources are included.
Includes information about targets, mutations and year of publication (based on docs).
\begin{quote}\begin{description}
\sphinxlineitem{Parameters}\begin{itemize}
\item {} 
\sphinxAtStartPar
\sphinxstyleliteralstrong{\sphinxupquote{chembl\_con}} (\sphinxstyleliteralemphasis{\sphinxupquote{sqlite3.Connection}}) \textendash{} Sqlite3 connection to ChEMBL database.

\item {} 
\sphinxAtStartPar
\sphinxstyleliteralstrong{\sphinxupquote{limit\_to\_literature}} (\sphinxstyleliteralemphasis{\sphinxupquote{bool}}) \textendash{} Include only literature sources if True.
Include all available sources otherwise.

\item {} 
\sphinxAtStartPar
\sphinxstyleliteralstrong{\sphinxupquote{df\_sizes}} (\sphinxstyleliteralemphasis{\sphinxupquote{list}}\sphinxstyleliteralemphasis{\sphinxupquote{{[}}}\sphinxstyleliteralemphasis{\sphinxupquote{list}}\sphinxstyleliteralemphasis{\sphinxupquote{{[}}}\sphinxstyleliteralemphasis{\sphinxupquote{int}}\sphinxstyleliteralemphasis{\sphinxupquote{{]}}}\sphinxstyleliteralemphasis{\sphinxupquote{, }}\sphinxstyleliteralemphasis{\sphinxupquote{list}}\sphinxstyleliteralemphasis{\sphinxupquote{{[}}}\sphinxstyleliteralemphasis{\sphinxupquote{int}}\sphinxstyleliteralemphasis{\sphinxupquote{{]}}}\sphinxstyleliteralemphasis{\sphinxupquote{{]}}}) \textendash{} List of intermediate sized of the dataset used for debugging.

\end{itemize}

\sphinxlineitem{Returns}
\sphinxAtStartPar
Pandas DataFrame with compound\sphinxhyphen{}target pairs with a pchembl value.

\sphinxlineitem{Return type}
\sphinxAtStartPar
pd.DataFrame

\end{description}\end{quote}

\end{fulllineitems}


\sphinxstepscope


\section{get\_dataset module}
\label{\detokenize{get_dataset:module-get_dataset}}\label{\detokenize{get_dataset:get-dataset-module}}\label{\detokenize{get_dataset::doc}}\index{module@\spxentry{module}!get\_dataset@\spxentry{get\_dataset}}\index{get\_dataset@\spxentry{get\_dataset}!module@\spxentry{module}}\index{get\_ct\_pair\_dataset() (in module get\_dataset)@\spxentry{get\_ct\_pair\_dataset()}\spxextra{in module get\_dataset}}

\begin{fulllineitems}
\phantomsection\label{\detokenize{get_dataset:get_dataset.get_ct_pair_dataset}}
\pysigstartsignatures
\pysiglinewithargsret{\sphinxcode{\sphinxupquote{get\_dataset.}}\sphinxbfcode{\sphinxupquote{get\_ct\_pair\_dataset}}}{\sphinxparam{\DUrole{n}{chembl\_con}\DUrole{p}{:}\DUrole{w}{ }\DUrole{n}{Connection}}\sphinxparamcomma \sphinxparam{\DUrole{n}{chembl\_version}\DUrole{p}{:}\DUrole{w}{ }\DUrole{n}{str}}\sphinxparamcomma \sphinxparam{\DUrole{n}{output\_path}\DUrole{p}{:}\DUrole{w}{ }\DUrole{n}{str}}\sphinxparamcomma \sphinxparam{\DUrole{n}{limit\_to\_literature}\DUrole{p}{:}\DUrole{w}{ }\DUrole{n}{bool}}\sphinxparamcomma \sphinxparam{\DUrole{n}{calculate\_rdkit}\DUrole{p}{:}\DUrole{w}{ }\DUrole{n}{bool}}\sphinxparamcomma \sphinxparam{\DUrole{n}{write\_to\_csv}\DUrole{p}{:}\DUrole{w}{ }\DUrole{n}{bool}}\sphinxparamcomma \sphinxparam{\DUrole{n}{write\_to\_excel}\DUrole{p}{:}\DUrole{w}{ }\DUrole{n}{bool}}\sphinxparamcomma \sphinxparam{\DUrole{n}{delimiter}\DUrole{p}{:}\DUrole{w}{ }\DUrole{n}{str}}\sphinxparamcomma \sphinxparam{\DUrole{n}{write\_full\_dataset}\DUrole{p}{:}\DUrole{w}{ }\DUrole{n}{bool}}\sphinxparamcomma \sphinxparam{\DUrole{n}{write\_bf}\DUrole{p}{:}\DUrole{w}{ }\DUrole{n}{bool}}\sphinxparamcomma \sphinxparam{\DUrole{n}{write\_b}\DUrole{p}{:}\DUrole{w}{ }\DUrole{n}{bool}}}{}
\pysigstopsignatures
\sphinxAtStartPar
Calculate and output the compound\sphinxhyphen{}target pair dataset.
\begin{quote}\begin{description}
\sphinxlineitem{Parameters}\begin{itemize}
\item {} 
\sphinxAtStartPar
\sphinxstyleliteralstrong{\sphinxupquote{chembl\_con}} (\sphinxstyleliteralemphasis{\sphinxupquote{sqlite3.Connection}}) \textendash{} Sqlite3 connection to ChEMBL database

\item {} 
\sphinxAtStartPar
\sphinxstyleliteralstrong{\sphinxupquote{chembl\_version}} (\sphinxstyleliteralemphasis{\sphinxupquote{str}}) \textendash{} Version of ChEMBL for output file names

\item {} 
\sphinxAtStartPar
\sphinxstyleliteralstrong{\sphinxupquote{output\_path}} (\sphinxstyleliteralemphasis{\sphinxupquote{str}}) \textendash{} Path to write output files to

\item {} 
\sphinxAtStartPar
\sphinxstyleliteralstrong{\sphinxupquote{limit\_to\_literature}} (\sphinxstyleliteralemphasis{\sphinxupquote{bool}}) \textendash{} Include only literature sources if True.
Include all available sources otherwise.

\item {} 
\sphinxAtStartPar
\sphinxstyleliteralstrong{\sphinxupquote{calculate\_rdkit}} (\sphinxstyleliteralemphasis{\sphinxupquote{bool}}) \textendash{} True if RDKit\sphinxhyphen{}based compound properties should be calculated

\item {} 
\sphinxAtStartPar
\sphinxstyleliteralstrong{\sphinxupquote{write\_to\_csv}} (\sphinxstyleliteralemphasis{\sphinxupquote{bool}}) \textendash{} True if output should be written to csv

\item {} 
\sphinxAtStartPar
\sphinxstyleliteralstrong{\sphinxupquote{write\_to\_excel}} (\sphinxstyleliteralemphasis{\sphinxupquote{bool}}) \textendash{} True if output should be written to excel

\item {} 
\sphinxAtStartPar
\sphinxstyleliteralstrong{\sphinxupquote{delimiter}} (\sphinxstyleliteralemphasis{\sphinxupquote{str}}) \textendash{} Delimiter in csv\sphinxhyphen{}output

\item {} 
\sphinxAtStartPar
\sphinxstyleliteralstrong{\sphinxupquote{write\_full\_dataset}} (\sphinxstyleliteralemphasis{\sphinxupquote{bool}}) \textendash{} True if the full dataset should be written to output

\item {} 
\sphinxAtStartPar
\sphinxstyleliteralstrong{\sphinxupquote{write\_bf}} (\sphinxstyleliteralemphasis{\sphinxupquote{bool}}) \textendash{} True if subsets based on binding+functional data should be written to output

\item {} 
\sphinxAtStartPar
\sphinxstyleliteralstrong{\sphinxupquote{write\_b}} (\sphinxstyleliteralemphasis{\sphinxupquote{bool}}) \textendash{} True if subsets based on binding data only should be written to output

\end{itemize}

\end{description}\end{quote}

\end{fulllineitems}


\sphinxstepscope


\section{get\_drug\_mechanism\_ct\_pairs module}
\label{\detokenize{get_drug_mechanism_ct_pairs:module-get_drug_mechanism_ct_pairs}}\label{\detokenize{get_drug_mechanism_ct_pairs:get-drug-mechanism-ct-pairs-module}}\label{\detokenize{get_drug_mechanism_ct_pairs::doc}}\index{module@\spxentry{module}!get\_drug\_mechanism\_ct\_pairs@\spxentry{get\_drug\_mechanism\_ct\_pairs}}\index{get\_drug\_mechanism\_ct\_pairs@\spxentry{get\_drug\_mechanism\_ct\_pairs}!module@\spxentry{module}}\index{add\_annotations\_to\_drug\_mechanisms\_cti() (in module get\_drug\_mechanism\_ct\_pairs)@\spxentry{add\_annotations\_to\_drug\_mechanisms\_cti()}\spxextra{in module get\_drug\_mechanism\_ct\_pairs}}

\begin{fulllineitems}
\phantomsection\label{\detokenize{get_drug_mechanism_ct_pairs:get_drug_mechanism_ct_pairs.add_annotations_to_drug_mechanisms_cti}}
\pysigstartsignatures
\pysiglinewithargsret{\sphinxcode{\sphinxupquote{get\_drug\_mechanism\_ct\_pairs.}}\sphinxbfcode{\sphinxupquote{add\_annotations\_to\_drug\_mechanisms\_cti}}}{\sphinxparam{\DUrole{n}{chembl\_con}\DUrole{p}{:}\DUrole{w}{ }\DUrole{n}{Connection}}\sphinxparamcomma \sphinxparam{\DUrole{n}{cpd\_target\_pairs}\DUrole{p}{:}\DUrole{w}{ }\DUrole{n}{DataFrame}}}{{ $\rightarrow$ DataFrame}}
\pysigstopsignatures
\sphinxAtStartPar
Add additional information to the compound\sphinxhyphen{}target pairs from the drug\_mechanisms table
to match the information that is present in the compound\sphinxhyphen{}target pairs table based on activities.
\begin{quote}\begin{description}
\sphinxlineitem{Parameters}\begin{itemize}
\item {} 
\sphinxAtStartPar
\sphinxstyleliteralstrong{\sphinxupquote{chembl\_con}} (\sphinxstyleliteralemphasis{\sphinxupquote{sqlite3.Connection}}) \textendash{} Sqlite3 connection to ChEMBL database.

\item {} 
\sphinxAtStartPar
\sphinxstyleliteralstrong{\sphinxupquote{cpd\_target\_pairs}} (\sphinxstyleliteralemphasis{\sphinxupquote{pd.DataFrame}}) \textendash{} Pandas DataFrame with compound\sphinxhyphen{}target pairs
from the drug\_mechanism table.

\end{itemize}

\sphinxlineitem{Returns}
\sphinxAtStartPar
Updated pandas DataFrame with the additional annotations.

\sphinxlineitem{Return type}
\sphinxAtStartPar
pd.DataFrame

\end{description}\end{quote}

\end{fulllineitems}

\index{add\_drug\_mechanism\_ct\_pairs() (in module get\_drug\_mechanism\_ct\_pairs)@\spxentry{add\_drug\_mechanism\_ct\_pairs()}\spxextra{in module get\_drug\_mechanism\_ct\_pairs}}

\begin{fulllineitems}
\phantomsection\label{\detokenize{get_drug_mechanism_ct_pairs:get_drug_mechanism_ct_pairs.add_drug_mechanism_ct_pairs}}
\pysigstartsignatures
\pysiglinewithargsret{\sphinxcode{\sphinxupquote{get\_drug\_mechanism\_ct\_pairs.}}\sphinxbfcode{\sphinxupquote{add\_drug\_mechanism\_ct\_pairs}}}{\sphinxparam{\DUrole{n}{df\_combined}\DUrole{p}{:}\DUrole{w}{ }\DUrole{n}{DataFrame}}\sphinxparamcomma \sphinxparam{\DUrole{n}{chembl\_con}\DUrole{p}{:}\DUrole{w}{ }\DUrole{n}{Connection}}}{{ $\rightarrow$ tuple\DUrole{p}{{[}}DataFrame\DUrole{p}{,}\DUrole{w}{ }set\DUrole{p}{,}\DUrole{w}{ }set\DUrole{p}{{]}}}}
\pysigstopsignatures
\sphinxAtStartPar
Add compound\sphinxhyphen{}target pairs from the drug\_mechanism table
that are not in the dataset based on the initial ChEMBL query.
These are compound\sphinxhyphen{}target pairs for which there is no associated pchembl value data.
Since the pairs are known interactions,
they are added to the dataset despite not having a pchembl value.
\begin{quote}\begin{description}
\sphinxlineitem{Parameters}\begin{itemize}
\item {} 
\sphinxAtStartPar
\sphinxstyleliteralstrong{\sphinxupquote{df\_combined}} (\sphinxstyleliteralemphasis{\sphinxupquote{pd.DataFrame}}) \textendash{} Pandas Dataframe with compound\sphinxhyphen{}target pairs based on ChEMBL activity data

\item {} 
\sphinxAtStartPar
\sphinxstyleliteralstrong{\sphinxupquote{chembl\_con}} (\sphinxstyleliteralemphasis{\sphinxupquote{sqlite3.Connection}}) \textendash{} Sqlite3 connection to ChEMBL database.

\end{itemize}

\sphinxlineitem{Returns}
\sphinxAtStartPar
\begin{itemize}
\item {} \begin{description}
\sphinxlineitem{Pandas DataFrame with compound\sphinxhyphen{}target pairs}
\sphinxAtStartPar
based on activities AND drug\_mechanism table 

\end{description}

\item {} 
\sphinxAtStartPar
set of compound\sphinxhyphen{}target pairs in the drug\_mechanism table 

\item {} 
\sphinxAtStartPar
set of targets in the drug\_mechanism table

\end{itemize}


\sphinxlineitem{Return type}
\sphinxAtStartPar
(pd.DataFrame, set, set)

\end{description}\end{quote}

\end{fulllineitems}

\index{get\_drug\_mechanism\_ct\_pairs() (in module get\_drug\_mechanism\_ct\_pairs)@\spxentry{get\_drug\_mechanism\_ct\_pairs()}\spxextra{in module get\_drug\_mechanism\_ct\_pairs}}

\begin{fulllineitems}
\phantomsection\label{\detokenize{get_drug_mechanism_ct_pairs:get_drug_mechanism_ct_pairs.get_drug_mechanism_ct_pairs}}
\pysigstartsignatures
\pysiglinewithargsret{\sphinxcode{\sphinxupquote{get\_drug\_mechanism\_ct\_pairs.}}\sphinxbfcode{\sphinxupquote{get\_drug\_mechanism\_ct\_pairs}}}{\sphinxparam{\DUrole{n}{chembl\_con}\DUrole{p}{:}\DUrole{w}{ }\DUrole{n}{Connection}}}{{ $\rightarrow$ DataFrame}}
\pysigstopsignatures
\sphinxAtStartPar
Get compound\sphinxhyphen{}target pairs from the drug\_mechanism table
with all the columns that are present in the compound\sphinxhyphen{}target pairs based on activities.
Relevant mappings of target ids to related target ids are taken into account.
\begin{quote}\begin{description}
\sphinxlineitem{Parameters}
\sphinxAtStartPar
\sphinxstyleliteralstrong{\sphinxupquote{chembl\_con}} (\sphinxstyleliteralemphasis{\sphinxupquote{sqlite3.Connection}}) \textendash{} Sqlite3 connection to ChEMBL database.

\sphinxlineitem{Returns}
\sphinxAtStartPar
Pandas DataFrame with compound\sphinxhyphen{}target interactions from the drug\_mechanism table.

\sphinxlineitem{Return type}
\sphinxAtStartPar
pd.DataFrame

\end{description}\end{quote}

\end{fulllineitems}

\index{get\_drug\_mechanisms\_interactions() (in module get\_drug\_mechanism\_ct\_pairs)@\spxentry{get\_drug\_mechanisms\_interactions()}\spxextra{in module get\_drug\_mechanism\_ct\_pairs}}

\begin{fulllineitems}
\phantomsection\label{\detokenize{get_drug_mechanism_ct_pairs:get_drug_mechanism_ct_pairs.get_drug_mechanisms_interactions}}
\pysigstartsignatures
\pysiglinewithargsret{\sphinxcode{\sphinxupquote{get\_drug\_mechanism\_ct\_pairs.}}\sphinxbfcode{\sphinxupquote{get\_drug\_mechanisms\_interactions}}}{\sphinxparam{\DUrole{n}{chembl\_con}\DUrole{p}{:}\DUrole{w}{ }\DUrole{n}{Connection}}}{{ $\rightarrow$ DataFrame}}
\pysigstopsignatures
\sphinxAtStartPar
Extract the known compound\sphinxhyphen{}target interactions from the ChEMBL drug\_mechanisms table.
Note: While the interactions are mostly between drugs and targets,
the table also includes some known interactions between
compounds with a max\_phase \textless{} 4 and their targets.

\sphinxAtStartPar
Only entries with a disease\_efficacy of 1 are taken into account,
i.e., the target is believed to play a role in the efficacy of the drug.

\sphinxAtStartPar
\sphinxstyleemphasis{disease\_efficacy: Flag to show whether the target assigned is believed
to play a role in the efficacy of the drug in the indication(s)
for which it is approved (1 = yes, 0 = no).}
\begin{quote}\begin{description}
\sphinxlineitem{Parameters}
\sphinxAtStartPar
\sphinxstyleliteralstrong{\sphinxupquote{chembl\_con}} (\sphinxstyleliteralemphasis{\sphinxupquote{sqlite3.Connection}}) \textendash{} Sqlite3 connection to ChEMBL database.

\sphinxlineitem{Returns}
\sphinxAtStartPar
Pandas DataFrame with compound\sphinxhyphen{}target pairs
from the drug\_mechanism table with disease relevance.

\sphinxlineitem{Return type}
\sphinxAtStartPar
pd.DataFrame

\end{description}\end{quote}

\end{fulllineitems}

\index{get\_relevant\_tid\_mappings() (in module get\_drug\_mechanism\_ct\_pairs)@\spxentry{get\_relevant\_tid\_mappings()}\spxextra{in module get\_drug\_mechanism\_ct\_pairs}}

\begin{fulllineitems}
\phantomsection\label{\detokenize{get_drug_mechanism_ct_pairs:get_drug_mechanism_ct_pairs.get_relevant_tid_mappings}}
\pysigstartsignatures
\pysiglinewithargsret{\sphinxcode{\sphinxupquote{get\_drug\_mechanism\_ct\_pairs.}}\sphinxbfcode{\sphinxupquote{get\_relevant\_tid\_mappings}}}{\sphinxparam{\DUrole{n}{chembl\_con}\DUrole{p}{:}\DUrole{w}{ }\DUrole{n}{Connection}}}{{ $\rightarrow$ DataFrame}}
\pysigstopsignatures
\sphinxAtStartPar
Get DataFrame with mappings from target id to their related target ids
based on the target\_relations table.
The following mappings are considered:


\begin{savenotes}\sphinxattablestart
\sphinxthistablewithglobalstyle
\centering
\begin{tabulary}{\linewidth}[t]{TTT}
\sphinxtoprule
\sphinxtableatstartofbodyhook
\sphinxAtStartPar
protein family
&
\sphinxAtStartPar
\sphinxhyphen{}{[}superset of{]}\sphinxhyphen{}\textgreater{}
&
\sphinxAtStartPar
single protein
\\
\sphinxhline
\sphinxAtStartPar
protein complex
&
\sphinxAtStartPar
\sphinxhyphen{}{[}superset of{]}\sphinxhyphen{}\textgreater{}
&
\sphinxAtStartPar
single protein
\\
\sphinxhline
\sphinxAtStartPar
protein complex group
&
\sphinxAtStartPar
\sphinxhyphen{}{[}superset of{]}\sphinxhyphen{}\textgreater{}
&
\sphinxAtStartPar
single protein
\\
\sphinxhline
\sphinxAtStartPar
single protein
&
\sphinxAtStartPar
\sphinxhyphen{}{[}equivalent to{]}\sphinxhyphen{}\textgreater{}
&
\sphinxAtStartPar
single protein
\\
\sphinxhline
\sphinxAtStartPar
chimeric protein
&
\sphinxAtStartPar
\sphinxhyphen{}{[}superset of{]}\sphinxhyphen{}\textgreater{}
&
\sphinxAtStartPar
single protein
\\
\sphinxhline
\sphinxAtStartPar
protein\sphinxhyphen{}protein interaction
&
\sphinxAtStartPar
\sphinxhyphen{}{[}superset of{]}\sphinxhyphen{}\textgreater{}
&
\sphinxAtStartPar
single protein
\\
\sphinxbottomrule
\end{tabulary}
\sphinxtableafterendhook\par
\sphinxattableend\end{savenotes}

\sphinxAtStartPar
These mappings can be used to increase the number of target ids
for which there is data in the drug\_mechanisms table.
For example, for \sphinxstyleemphasis{protein family \sphinxhyphen{}{[}superset of{]}\sphinxhyphen{}\textgreater{} single protein} this means:
If there is a known relevant interaction between a compound and a protein family,
interactions between the compound and single proteins of that protein family
are considered to be known interactions as well.
\begin{quote}\begin{description}
\sphinxlineitem{Parameters}
\sphinxAtStartPar
\sphinxstyleliteralstrong{\sphinxupquote{chembl\_con}} (\sphinxstyleliteralemphasis{\sphinxupquote{sqlite3.Connection}}) \textendash{} Sqlite3 connection to ChEMBL database.

\sphinxlineitem{Returns}
\sphinxAtStartPar
Pandas DataFrame with mappings from tid to related tid
for the defined subset of target relations.

\sphinxlineitem{Return type}
\sphinxAtStartPar
pd.DataFrame

\end{description}\end{quote}

\end{fulllineitems}


\sphinxstepscope


\section{get\_stats module}
\label{\detokenize{get_stats:module-get_stats}}\label{\detokenize{get_stats:get-stats-module}}\label{\detokenize{get_stats::doc}}\index{module@\spxentry{module}!get\_stats@\spxentry{get\_stats}}\index{get\_stats@\spxentry{get\_stats}!module@\spxentry{module}}\index{add\_dataset\_sizes() (in module get\_stats)@\spxentry{add\_dataset\_sizes()}\spxextra{in module get\_stats}}

\begin{fulllineitems}
\phantomsection\label{\detokenize{get_stats:get_stats.add_dataset_sizes}}
\pysigstartsignatures
\pysiglinewithargsret{\sphinxcode{\sphinxupquote{get\_stats.}}\sphinxbfcode{\sphinxupquote{add\_dataset\_sizes}}}{\sphinxparam{\DUrole{n}{df}\DUrole{p}{:}\DUrole{w}{ }\DUrole{n}{DataFrame}}\sphinxparamcomma \sphinxparam{\DUrole{n}{label}\DUrole{p}{:}\DUrole{w}{ }\DUrole{n}{str}}\sphinxparamcomma \sphinxparam{\DUrole{n}{df\_sizes}\DUrole{p}{:}\DUrole{w}{ }\DUrole{n}{list\DUrole{p}{{[}}list\DUrole{p}{{[}}int\DUrole{p}{{]}}\DUrole{p}{,}\DUrole{w}{ }list\DUrole{p}{{[}}int\DUrole{p}{{]}}\DUrole{p}{{]}}}}}{}
\pysigstopsignatures
\sphinxAtStartPar
Count and add representative counts of df to the list df\_sizes used for debugging.
\begin{quote}\begin{description}
\sphinxlineitem{Parameters}\begin{itemize}
\item {} 
\sphinxAtStartPar
\sphinxstyleliteralstrong{\sphinxupquote{df}} (\sphinxstyleliteralemphasis{\sphinxupquote{pd.DataFrame}}) \textendash{} Pandas DataFrame with current compound\sphinxhyphen{}target pairs

\item {} 
\sphinxAtStartPar
\sphinxstyleliteralstrong{\sphinxupquote{label}} (\sphinxstyleliteralemphasis{\sphinxupquote{str}}) \textendash{} Description of pipeline step (e.g., initial query).

\item {} 
\sphinxAtStartPar
\sphinxstyleliteralstrong{\sphinxupquote{df\_sizes}} (\sphinxstyleliteralemphasis{\sphinxupquote{list}}\sphinxstyleliteralemphasis{\sphinxupquote{{[}}}\sphinxstyleliteralemphasis{\sphinxupquote{list}}\sphinxstyleliteralemphasis{\sphinxupquote{{[}}}\sphinxstyleliteralemphasis{\sphinxupquote{int}}\sphinxstyleliteralemphasis{\sphinxupquote{{]}}}\sphinxstyleliteralemphasis{\sphinxupquote{, }}\sphinxstyleliteralemphasis{\sphinxupquote{list}}\sphinxstyleliteralemphasis{\sphinxupquote{{[}}}\sphinxstyleliteralemphasis{\sphinxupquote{int}}\sphinxstyleliteralemphasis{\sphinxupquote{{]}}}\sphinxstyleliteralemphasis{\sphinxupquote{{]}}}) \textendash{} List of intermediate sized of the dataset used for debugging.

\end{itemize}

\end{description}\end{quote}

\end{fulllineitems}

\index{calculate\_dataset\_sizes() (in module get\_stats)@\spxentry{calculate\_dataset\_sizes()}\spxextra{in module get\_stats}}

\begin{fulllineitems}
\phantomsection\label{\detokenize{get_stats:get_stats.calculate_dataset_sizes}}
\pysigstartsignatures
\pysiglinewithargsret{\sphinxcode{\sphinxupquote{get\_stats.}}\sphinxbfcode{\sphinxupquote{calculate\_dataset\_sizes}}}{\sphinxparam{\DUrole{n}{df}\DUrole{p}{:}\DUrole{w}{ }\DUrole{n}{DataFrame}}}{{ $\rightarrow$ list\DUrole{p}{{[}}int\DUrole{p}{{]}}}}
\pysigstopsignatures
\sphinxAtStartPar
Calculate the number of unique compounds, targets and pairs
for df and df limited to drugs.
\begin{quote}\begin{description}
\sphinxlineitem{Parameters}
\sphinxAtStartPar
\sphinxstyleliteralstrong{\sphinxupquote{df}} (\sphinxstyleliteralemphasis{\sphinxupquote{pd.DataFrame}}) \textendash{} Pandas DataFrame for which the dataset sizes should be calculated.

\sphinxlineitem{Returns}
\sphinxAtStartPar
List of calculated unique counts.

\sphinxlineitem{Return type}
\sphinxAtStartPar
list{[}int{]}

\end{description}\end{quote}

\end{fulllineitems}

\index{get\_stats\_for\_column() (in module get\_stats)@\spxentry{get\_stats\_for\_column()}\spxextra{in module get\_stats}}

\begin{fulllineitems}
\phantomsection\label{\detokenize{get_stats:get_stats.get_stats_for_column}}
\pysigstartsignatures
\pysiglinewithargsret{\sphinxcode{\sphinxupquote{get\_stats.}}\sphinxbfcode{\sphinxupquote{get\_stats\_for\_column}}}{\sphinxparam{\DUrole{n}{df}\DUrole{p}{:}\DUrole{w}{ }\DUrole{n}{DataFrame}}\sphinxparamcomma \sphinxparam{\DUrole{n}{column}\DUrole{p}{:}\DUrole{w}{ }\DUrole{n}{str}}\sphinxparamcomma \sphinxparam{\DUrole{n}{columns\_desc}\DUrole{p}{:}\DUrole{w}{ }\DUrole{n}{str}}}{{ $\rightarrow$ list\DUrole{p}{{[}}list\DUrole{p}{{[}}str\DUrole{p}{,}\DUrole{w}{ }str\DUrole{p}{,}\DUrole{w}{ }int\DUrole{p}{{]}}\DUrole{p}{{]}}}}
\pysigstopsignatures
\sphinxAtStartPar
Calculate the number of unique values in df{[}column{]} and various subsets of df.
\begin{quote}\begin{description}
\sphinxlineitem{Parameters}\begin{itemize}
\item {} 
\sphinxAtStartPar
\sphinxstyleliteralstrong{\sphinxupquote{df}} (\sphinxstyleliteralemphasis{\sphinxupquote{pd.DataFrame}}) \textendash{} Pandas Dataframe for which the number of unique values should be calculated

\item {} 
\sphinxAtStartPar
\sphinxstyleliteralstrong{\sphinxupquote{column}} (\sphinxstyleliteralemphasis{\sphinxupquote{str}}) \textendash{} Column of df that the values should be calculated for

\item {} 
\sphinxAtStartPar
\sphinxstyleliteralstrong{\sphinxupquote{columns\_desc}} (\sphinxstyleliteralemphasis{\sphinxupquote{str}}) \textendash{} Description of the column

\end{itemize}

\sphinxlineitem{Returns}
\sphinxAtStartPar
List of results in the format {[}column\_name, subset\_type, size{]}

\sphinxlineitem{Return type}
\sphinxAtStartPar
list{[}list{[}str, str, int{]}{]}

\end{description}\end{quote}

\end{fulllineitems}


\sphinxstepscope


\section{main module}
\label{\detokenize{main:module-main}}\label{\detokenize{main:main-module}}\label{\detokenize{main::doc}}\index{module@\spxentry{module}!main@\spxentry{main}}\index{main@\spxentry{main}!module@\spxentry{module}}\index{main() (in module main)@\spxentry{main()}\spxextra{in module main}}

\begin{fulllineitems}
\phantomsection\label{\detokenize{main:main.main}}
\pysigstartsignatures
\pysiglinewithargsret{\sphinxcode{\sphinxupquote{main.}}\sphinxbfcode{\sphinxupquote{main}}}{}{}
\pysigstopsignatures
\sphinxAtStartPar
Call get\_ct\_pair\_dataset to get the compound\sphinxhyphen{}target dataset using the given arguments.

\end{fulllineitems}

\index{parse\_args() (in module main)@\spxentry{parse\_args()}\spxextra{in module main}}

\begin{fulllineitems}
\phantomsection\label{\detokenize{main:main.parse_args}}
\pysigstartsignatures
\pysiglinewithargsret{\sphinxcode{\sphinxupquote{main.}}\sphinxbfcode{\sphinxupquote{parse\_args}}}{}{{ $\rightarrow$ Namespace}}
\pysigstopsignatures
\sphinxAtStartPar
Get arguments with argparse.
\begin{quote}\begin{description}
\sphinxlineitem{Returns}
\sphinxAtStartPar
Populated argparse.Namespace

\sphinxlineitem{Return type}
\sphinxAtStartPar
argparse.Namespace

\end{description}\end{quote}

\end{fulllineitems}


\sphinxstepscope


\section{sanity\_checks module}
\label{\detokenize{sanity_checks:module-sanity_checks}}\label{\detokenize{sanity_checks:sanity-checks-module}}\label{\detokenize{sanity_checks::doc}}\index{module@\spxentry{module}!sanity\_checks@\spxentry{sanity\_checks}}\index{sanity\_checks@\spxentry{sanity\_checks}!module@\spxentry{module}}\index{check\_atc\_and\_target\_classes() (in module sanity\_checks)@\spxentry{check\_atc\_and\_target\_classes()}\spxextra{in module sanity\_checks}}

\begin{fulllineitems}
\phantomsection\label{\detokenize{sanity_checks:sanity_checks.check_atc_and_target_classes}}
\pysigstartsignatures
\pysiglinewithargsret{\sphinxcode{\sphinxupquote{sanity\_checks.}}\sphinxbfcode{\sphinxupquote{check\_atc\_and\_target\_classes}}}{\sphinxparam{\DUrole{n}{df\_combined}\DUrole{p}{:}\DUrole{w}{ }\DUrole{n}{DataFrame}}\sphinxparamcomma \sphinxparam{\DUrole{n}{atc\_levels}\DUrole{p}{:}\DUrole{w}{ }\DUrole{n}{DataFrame}}\sphinxparamcomma \sphinxparam{\DUrole{n}{target\_classes\_level1}\DUrole{p}{:}\DUrole{w}{ }\DUrole{n}{DataFrame}}\sphinxparamcomma \sphinxparam{\DUrole{n}{target\_classes\_level2}\DUrole{p}{:}\DUrole{w}{ }\DUrole{n}{DataFrame}}}{}
\pysigstopsignatures
\sphinxAtStartPar
Check that atc\_level1 and target class information is only null
if the parent\_molregno / target id is not in the respective table.

\end{fulllineitems}

\index{check\_compound\_props() (in module sanity\_checks)@\spxentry{check\_compound\_props()}\spxextra{in module sanity\_checks}}

\begin{fulllineitems}
\phantomsection\label{\detokenize{sanity_checks:sanity_checks.check_compound_props}}
\pysigstartsignatures
\pysiglinewithargsret{\sphinxcode{\sphinxupquote{sanity\_checks.}}\sphinxbfcode{\sphinxupquote{check\_compound\_props}}}{\sphinxparam{\DUrole{n}{df\_combined}\DUrole{p}{:}\DUrole{w}{ }\DUrole{n}{DataFrame}}\sphinxparamcomma \sphinxparam{\DUrole{n}{df\_cpd\_props}\DUrole{p}{:}\DUrole{w}{ }\DUrole{n}{DataFrame}}}{}
\pysigstopsignatures
\sphinxAtStartPar
Check that compound props are only null if
\begin{itemize}
\item {} 
\sphinxAtStartPar
the property in the parent\_molregno is not in df\_cpd\_props

\item {} 
\sphinxAtStartPar
or if the value in the compound props table is null.

\end{itemize}

\end{fulllineitems}

\index{check\_for\_mixed\_types() (in module sanity\_checks)@\spxentry{check\_for\_mixed\_types()}\spxextra{in module sanity\_checks}}

\begin{fulllineitems}
\phantomsection\label{\detokenize{sanity_checks:sanity_checks.check_for_mixed_types}}
\pysigstartsignatures
\pysiglinewithargsret{\sphinxcode{\sphinxupquote{sanity\_checks.}}\sphinxbfcode{\sphinxupquote{check\_for\_mixed\_types}}}{\sphinxparam{\DUrole{n}{df\_combined}\DUrole{p}{:}\DUrole{w}{ }\DUrole{n}{DataFrame}}}{}
\pysigstopsignatures
\sphinxAtStartPar
Check that there are no mixed types in columns with dtype=object.

\end{fulllineitems}

\index{check\_ligand\_efficiency\_metrics() (in module sanity\_checks)@\spxentry{check\_ligand\_efficiency\_metrics()}\spxextra{in module sanity\_checks}}

\begin{fulllineitems}
\phantomsection\label{\detokenize{sanity_checks:sanity_checks.check_ligand_efficiency_metrics}}
\pysigstartsignatures
\pysiglinewithargsret{\sphinxcode{\sphinxupquote{sanity\_checks.}}\sphinxbfcode{\sphinxupquote{check\_ligand\_efficiency\_metrics}}}{\sphinxparam{\DUrole{n}{df\_combined}\DUrole{p}{:}\DUrole{w}{ }\DUrole{n}{DataFrame}}}{}
\pysigstopsignatures
\sphinxAtStartPar
Check that ligand efficiency metrics are only null
when at least one of the values used to calculate them is null.
Ligand efficiency metrics are only null when at least
one of the values used to calculate them is null.

\end{fulllineitems}

\index{check\_null\_values() (in module sanity\_checks)@\spxentry{check\_null\_values()}\spxextra{in module sanity\_checks}}

\begin{fulllineitems}
\phantomsection\label{\detokenize{sanity_checks:sanity_checks.check_null_values}}
\pysigstartsignatures
\pysiglinewithargsret{\sphinxcode{\sphinxupquote{sanity\_checks.}}\sphinxbfcode{\sphinxupquote{check\_null\_values}}}{\sphinxparam{\DUrole{n}{df\_combined}\DUrole{p}{:}\DUrole{w}{ }\DUrole{n}{DataFrame}}}{}
\pysigstopsignatures
\sphinxAtStartPar
Check if any columns contain nan or null which aren’t recognised as null values.

\end{fulllineitems}

\index{check\_pairs\_without\_pchembl\_are\_in\_drug\_mechanisms() (in module sanity\_checks)@\spxentry{check\_pairs\_without\_pchembl\_are\_in\_drug\_mechanisms()}\spxextra{in module sanity\_checks}}

\begin{fulllineitems}
\phantomsection\label{\detokenize{sanity_checks:sanity_checks.check_pairs_without_pchembl_are_in_drug_mechanisms}}
\pysigstartsignatures
\pysiglinewithargsret{\sphinxcode{\sphinxupquote{sanity\_checks.}}\sphinxbfcode{\sphinxupquote{check\_pairs\_without\_pchembl\_are\_in\_drug\_mechanisms}}}{\sphinxparam{\DUrole{n}{df\_combined}\DUrole{p}{:}\DUrole{w}{ }\DUrole{n}{DataFrame}}}{}
\pysigstopsignatures
\sphinxAtStartPar
Check that rows without a pchembl value based on binding+functional assays (pchembl\_x\_BF)
are in the drug\_mechanism table.
Note that this is not true for the pchembl\_x\_B columns which are based on binding data only.
They may be in the table because there is data based on functional assays
but no data based on binding assays.
All pchembl\_value\_x\_BF columns without a pchembl should be in the dm table.

\end{fulllineitems}

\index{check\_rdkit\_props() (in module sanity\_checks)@\spxentry{check\_rdkit\_props()}\spxextra{in module sanity\_checks}}

\begin{fulllineitems}
\phantomsection\label{\detokenize{sanity_checks:sanity_checks.check_rdkit_props}}
\pysigstartsignatures
\pysiglinewithargsret{\sphinxcode{\sphinxupquote{sanity\_checks.}}\sphinxbfcode{\sphinxupquote{check\_rdkit\_props}}}{\sphinxparam{\DUrole{n}{df\_combined}\DUrole{p}{:}\DUrole{w}{ }\DUrole{n}{DataFrame}}}{}
\pysigstopsignatures
\sphinxAtStartPar
Check that columns set by the RDKit are only null
if there is no canonical SMILES for the molecule.
Scaffolds are excluded from this test because
they can be None if the molecule is acyclic.

\end{fulllineitems}

\index{sanity\_checks() (in module sanity\_checks)@\spxentry{sanity\_checks()}\spxextra{in module sanity\_checks}}

\begin{fulllineitems}
\phantomsection\label{\detokenize{sanity_checks:sanity_checks.sanity_checks}}
\pysigstartsignatures
\pysiglinewithargsret{\sphinxcode{\sphinxupquote{sanity\_checks.}}\sphinxbfcode{\sphinxupquote{sanity\_checks}}}{\sphinxparam{\DUrole{n}{df\_combined}\DUrole{p}{:}\DUrole{w}{ }\DUrole{n}{DataFrame}}\sphinxparamcomma \sphinxparam{\DUrole{n}{df\_cpd\_props}\DUrole{p}{:}\DUrole{w}{ }\DUrole{n}{DataFrame}}\sphinxparamcomma \sphinxparam{\DUrole{n}{atc\_levels}\DUrole{p}{:}\DUrole{w}{ }\DUrole{n}{DataFrame}}\sphinxparamcomma \sphinxparam{\DUrole{n}{target\_classes\_level1}\DUrole{p}{:}\DUrole{w}{ }\DUrole{n}{DataFrame}}\sphinxparamcomma \sphinxparam{\DUrole{n}{target\_classes\_level2}\DUrole{p}{:}\DUrole{w}{ }\DUrole{n}{DataFrame}}\sphinxparamcomma \sphinxparam{\DUrole{n}{calculate\_rdkit}\DUrole{p}{:}\DUrole{w}{ }\DUrole{n}{bool}}}{}
\pysigstopsignatures
\sphinxAtStartPar
Check basic assumptions about the finished dataset, specifically:
\begin{itemize}
\item {} 
\sphinxAtStartPar
no columns contain nan or null values which aren’t recognised as null values

\item {} 
\sphinxAtStartPar
there are no mixed types in columns with dtype=object

\item {} \begin{description}
\sphinxlineitem{rows without a pchembl value based on binding+functional assays (pchembl\_x\_BF)}
\sphinxAtStartPar
are in the drug\_mechanism table

\end{description}

\item {} \begin{description}
\sphinxlineitem{ligand efficiency metrics are only null when at least one of the values}
\sphinxAtStartPar
used to calculate them is null

\end{description}

\item {} \begin{description}
\sphinxlineitem{compound props are only null if the compound is not in df\_cpd\_props}
\sphinxAtStartPar
or the value in that table is null

\end{description}

\item {} \begin{description}
\sphinxlineitem{atc\_level1 and target class information is only null if}
\sphinxAtStartPar
the parent\_molregno / target id is not in the respective table

\end{description}

\item {} \begin{description}
\sphinxlineitem{columns set by the RDKit are only null if there is no canonical SMILES}
\sphinxAtStartPar
for the molecule (excluding scaffolds)

\end{description}

\end{itemize}
\begin{quote}\begin{description}
\sphinxlineitem{Parameters}\begin{itemize}
\item {} 
\sphinxAtStartPar
\sphinxstyleliteralstrong{\sphinxupquote{df\_combined}} (\sphinxstyleliteralemphasis{\sphinxupquote{pd.DataFrame}}) \textendash{} Pandas DataFrame with compound\sphinxhyphen{}target pairs

\item {} 
\sphinxAtStartPar
\sphinxstyleliteralstrong{\sphinxupquote{df\_cpd\_props}} (\sphinxstyleliteralemphasis{\sphinxupquote{pd.DataFrame}}) \textendash{} Pandas DataFrame with compound properties
and structures for all compound ids in ChEMBL.

\item {} 
\sphinxAtStartPar
\sphinxstyleliteralstrong{\sphinxupquote{atc\_levels}} (\sphinxstyleliteralemphasis{\sphinxupquote{pd.DataFrame}}) \textendash{} Pandas DataFrame with ATC annotations in ChEMBL

\item {} 
\sphinxAtStartPar
\sphinxstyleliteralstrong{\sphinxupquote{target\_classes\_level1}} (\sphinxstyleliteralemphasis{\sphinxupquote{pd.DataFrame}}) \textendash{} Pandas DataFrame with mapping
from target id to level 1 target class

\item {} 
\sphinxAtStartPar
\sphinxstyleliteralstrong{\sphinxupquote{target\_classes\_level2}} (\sphinxstyleliteralemphasis{\sphinxupquote{pd.DataFrame}}) \textendash{} Pandas DataFrame with mapping
from target id to level 2 target class

\item {} 
\sphinxAtStartPar
\sphinxstyleliteralstrong{\sphinxupquote{calculate\_rdkit}} (\sphinxstyleliteralemphasis{\sphinxupquote{bool}}) \textendash{} True if the DataFrame contains RDKit\sphinxhyphen{}based compound properties

\end{itemize}

\end{description}\end{quote}

\end{fulllineitems}

\index{test\_equality() (in module sanity\_checks)@\spxentry{test\_equality()}\spxextra{in module sanity\_checks}}

\begin{fulllineitems}
\phantomsection\label{\detokenize{sanity_checks:sanity_checks.test_equality}}
\pysigstartsignatures
\pysiglinewithargsret{\sphinxcode{\sphinxupquote{sanity\_checks.}}\sphinxbfcode{\sphinxupquote{test\_equality}}}{\sphinxparam{\DUrole{n}{current\_df}\DUrole{p}{:}\DUrole{w}{ }\DUrole{n}{DataFrame}}\sphinxparamcomma \sphinxparam{\DUrole{n}{read\_file\_name}\DUrole{p}{:}\DUrole{w}{ }\DUrole{n}{str}}\sphinxparamcomma \sphinxparam{\DUrole{n}{assay\_type}\DUrole{p}{:}\DUrole{w}{ }\DUrole{n}{str}}\sphinxparamcomma \sphinxparam{\DUrole{n}{file\_type\_list}\DUrole{p}{:}\DUrole{w}{ }\DUrole{n}{list\DUrole{p}{{[}}str\DUrole{p}{{]}}}}\sphinxparamcomma \sphinxparam{\DUrole{n}{calculate\_rdkit}\DUrole{p}{:}\DUrole{w}{ }\DUrole{n}{bool}}}{}
\pysigstopsignatures
\sphinxAtStartPar
Check that the file that was written to \textless{}read\_file\_name\textgreater{}
is identical to the DataFrame \textless{}current\_df\textgreater{} it was based on.
\begin{quote}\begin{description}
\sphinxlineitem{Parameters}\begin{itemize}
\item {} 
\sphinxAtStartPar
\sphinxstyleliteralstrong{\sphinxupquote{current\_df}} (\sphinxstyleliteralemphasis{\sphinxupquote{pd.DataFrame}}) \textendash{} Pandas DataFrame that was written to read\_file\_name

\item {} 
\sphinxAtStartPar
\sphinxstyleliteralstrong{\sphinxupquote{read\_file\_name}} (\sphinxstyleliteralemphasis{\sphinxupquote{str}}) \textendash{} Name of the file current\_df was written to

\item {} 
\sphinxAtStartPar
\sphinxstyleliteralstrong{\sphinxupquote{assay\_type}} (\sphinxstyleliteralemphasis{\sphinxupquote{str}}) \textendash{} Types of assays current\_df contains information about.         Options:    “BF” (binding+functional),
“B” (binding),
“all” (contains both BF and B information)

\item {} 
\sphinxAtStartPar
\sphinxstyleliteralstrong{\sphinxupquote{file\_type\_list}} (\sphinxstyleliteralemphasis{\sphinxupquote{list}}\sphinxstyleliteralemphasis{\sphinxupquote{{[}}}\sphinxstyleliteralemphasis{\sphinxupquote{str}}\sphinxstyleliteralemphasis{\sphinxupquote{{]}}}) \textendash{} List of file extensions used with read\_file\_name. Options: csv, xlsx

\item {} 
\sphinxAtStartPar
\sphinxstyleliteralstrong{\sphinxupquote{calculate\_rdkit}} (\sphinxstyleliteralemphasis{\sphinxupquote{bool}}) \textendash{} If True, current\_df contains RDKit\sphinxhyphen{}based columns

\end{itemize}

\end{description}\end{quote}

\end{fulllineitems}


\sphinxstepscope


\section{write\_subsets module}
\label{\detokenize{write_subsets:module-write_subsets}}\label{\detokenize{write_subsets:write-subsets-module}}\label{\detokenize{write_subsets::doc}}\index{module@\spxentry{module}!write\_subsets@\spxentry{write\_subsets}}\index{write\_subsets@\spxentry{write\_subsets}!module@\spxentry{module}}\index{get\_data\_subsets() (in module write\_subsets)@\spxentry{get\_data\_subsets()}\spxextra{in module write\_subsets}}

\begin{fulllineitems}
\phantomsection\label{\detokenize{write_subsets:write_subsets.get_data_subsets}}
\pysigstartsignatures
\pysiglinewithargsret{\sphinxcode{\sphinxupquote{write\_subsets.}}\sphinxbfcode{\sphinxupquote{get\_data\_subsets}}}{\sphinxparam{\DUrole{n}{data}\DUrole{p}{:}\DUrole{w}{ }\DUrole{n}{DataFrame}}\sphinxparamcomma \sphinxparam{\DUrole{n}{min\_nof\_cpds}\DUrole{p}{:}\DUrole{w}{ }\DUrole{n}{int}}\sphinxparamcomma \sphinxparam{\DUrole{n}{desc}\DUrole{p}{:}\DUrole{w}{ }\DUrole{n}{str}}}{{ $\rightarrow$ tuple\DUrole{p}{{[}}DataFrame\DUrole{p}{,}\DUrole{w}{ }DataFrame\DUrole{p}{,}\DUrole{w}{ }DataFrame\DUrole{p}{,}\DUrole{w}{ }DataFrame\DUrole{p}{{]}}}}
\pysigstopsignatures
\sphinxAtStartPar
Calculate and return the different subsets of interest.
\begin{quote}\begin{description}
\sphinxlineitem{Parameters}\begin{itemize}
\item {} 
\sphinxAtStartPar
\sphinxstyleliteralstrong{\sphinxupquote{data}} (\sphinxstyleliteralemphasis{\sphinxupquote{pd.DataFrame}}) \textendash{} Pandas DataFrame with compound\sphinxhyphen{}target pairs

\item {} 
\sphinxAtStartPar
\sphinxstyleliteralstrong{\sphinxupquote{min\_nof\_cpds}} (\sphinxstyleliteralemphasis{\sphinxupquote{int}}) \textendash{} Miminum number of compounds per target

\item {} 
\sphinxAtStartPar
\sphinxstyleliteralstrong{\sphinxupquote{desc}} (\sphinxstyleliteralemphasis{\sphinxupquote{str}}) \textendash{} Types of assays current\_df contains information about.         Options: “BF” (binding+functional), “B” (binding)

\end{itemize}

\sphinxlineitem{Returns}
\sphinxAtStartPar
\begin{itemize}
\item {} \begin{description}
\sphinxlineitem{data: Pandas DataFrame with compound\sphinxhyphen{}target pairs}
\sphinxAtStartPar
without the annotations for the opposite desc,             e.g. if desc = “BF”, the average pchembl value based on
binding data only is dropped

\end{description}

\item {} \begin{description}
\sphinxlineitem{df\_enough\_cpds: Pandas DataFrame with targets}
\sphinxAtStartPar
with at least \textless{}min\_nof\_cpds\textgreater{} compounds with a pchembl value,

\end{description}

\item {} \begin{description}
\sphinxlineitem{df\_c\_dt\_d\_dt: As df\_enough\_cpds but with             at least one compound\sphinxhyphen{}target pair labelled as}
\sphinxAtStartPar
’D\_DT’, ‘C3\_DT’, ‘C2\_DT’, ‘C1\_DT’ or ‘C0\_DT’ (i.e., known interaction),

\end{description}

\item {} \begin{description}
\sphinxlineitem{df\_d\_dt: As df\_enough\_cpds but with             at least one compound\sphinxhyphen{}target pair labelled as}
\sphinxAtStartPar
’D\_DT’ (i.e., known drug\sphinxhyphen{}target interaction)

\end{description}

\end{itemize}


\sphinxlineitem{Return type}
\sphinxAtStartPar
(pd.DataFrame, pd.DataFrame, pd.DataFrame, pd.DataFrame)

\end{description}\end{quote}

\end{fulllineitems}

\index{output\_debug\_sizes() (in module write\_subsets)@\spxentry{output\_debug\_sizes()}\spxextra{in module write\_subsets}}

\begin{fulllineitems}
\phantomsection\label{\detokenize{write_subsets:write_subsets.output_debug_sizes}}
\pysigstartsignatures
\pysiglinewithargsret{\sphinxcode{\sphinxupquote{write\_subsets.}}\sphinxbfcode{\sphinxupquote{output\_debug\_sizes}}}{\sphinxparam{\DUrole{n}{df\_sizes}\DUrole{p}{:}\DUrole{w}{ }\DUrole{n}{list\DUrole{p}{{[}}list\DUrole{p}{{[}}int\DUrole{p}{{]}}\DUrole{p}{,}\DUrole{w}{ }list\DUrole{p}{{[}}int\DUrole{p}{{]}}\DUrole{p}{{]}}}}\sphinxparamcomma \sphinxparam{\DUrole{n}{output\_path}\DUrole{p}{:}\DUrole{w}{ }\DUrole{n}{str}}\sphinxparamcomma \sphinxparam{\DUrole{n}{write\_to\_csv}\DUrole{p}{:}\DUrole{w}{ }\DUrole{n}{bool}}\sphinxparamcomma \sphinxparam{\DUrole{n}{write\_to\_excel}\DUrole{p}{:}\DUrole{w}{ }\DUrole{n}{bool}}\sphinxparamcomma \sphinxparam{\DUrole{n}{delimiter}\DUrole{p}{:}\DUrole{w}{ }\DUrole{n}{str}}}{}
\pysigstopsignatures
\sphinxAtStartPar
Output counts at various points during calculating the final dataset for debugging.
\begin{quote}\begin{description}
\sphinxlineitem{Parameters}\begin{itemize}
\item {} 
\sphinxAtStartPar
\sphinxstyleliteralstrong{\sphinxupquote{df\_sizes}} (\sphinxstyleliteralemphasis{\sphinxupquote{list}}\sphinxstyleliteralemphasis{\sphinxupquote{{[}}}\sphinxstyleliteralemphasis{\sphinxupquote{list}}\sphinxstyleliteralemphasis{\sphinxupquote{{[}}}\sphinxstyleliteralemphasis{\sphinxupquote{int}}\sphinxstyleliteralemphasis{\sphinxupquote{{]}}}\sphinxstyleliteralemphasis{\sphinxupquote{, }}\sphinxstyleliteralemphasis{\sphinxupquote{list}}\sphinxstyleliteralemphasis{\sphinxupquote{{[}}}\sphinxstyleliteralemphasis{\sphinxupquote{int}}\sphinxstyleliteralemphasis{\sphinxupquote{{]}}}\sphinxstyleliteralemphasis{\sphinxupquote{{]}}}) \textendash{} List of intermediate sized of the dataset used for debugging.

\item {} 
\sphinxAtStartPar
\sphinxstyleliteralstrong{\sphinxupquote{output\_path}} (\sphinxstyleliteralemphasis{\sphinxupquote{str}}) \textendash{} Path to write the dataset counts to

\item {} 
\sphinxAtStartPar
\sphinxstyleliteralstrong{\sphinxupquote{write\_to\_csv}} (\sphinxstyleliteralemphasis{\sphinxupquote{bool}}) \textendash{} True if counts should be written to csv

\item {} 
\sphinxAtStartPar
\sphinxstyleliteralstrong{\sphinxupquote{write\_to\_excel}} (\sphinxstyleliteralemphasis{\sphinxupquote{bool}}) \textendash{} True if counts should be written to excel

\item {} 
\sphinxAtStartPar
\sphinxstyleliteralstrong{\sphinxupquote{delimiter}} (\sphinxstyleliteralemphasis{\sphinxupquote{str}}) \textendash{} Delimiter in csv\sphinxhyphen{}output

\end{itemize}

\end{description}\end{quote}

\end{fulllineitems}

\index{output\_stats() (in module write\_subsets)@\spxentry{output\_stats()}\spxextra{in module write\_subsets}}

\begin{fulllineitems}
\phantomsection\label{\detokenize{write_subsets:write_subsets.output_stats}}
\pysigstartsignatures
\pysiglinewithargsret{\sphinxcode{\sphinxupquote{write\_subsets.}}\sphinxbfcode{\sphinxupquote{output\_stats}}}{\sphinxparam{\DUrole{n}{df}\DUrole{p}{:}\DUrole{w}{ }\DUrole{n}{DataFrame}}\sphinxparamcomma \sphinxparam{\DUrole{n}{output\_file}\DUrole{p}{:}\DUrole{w}{ }\DUrole{n}{str}}\sphinxparamcomma \sphinxparam{\DUrole{n}{write\_to\_csv}\DUrole{p}{:}\DUrole{w}{ }\DUrole{n}{bool}}\sphinxparamcomma \sphinxparam{\DUrole{n}{write\_to\_excel}\DUrole{p}{:}\DUrole{w}{ }\DUrole{n}{bool}}\sphinxparamcomma \sphinxparam{\DUrole{n}{delimiter}\DUrole{p}{:}\DUrole{w}{ }\DUrole{n}{str}}}{}
\pysigstopsignatures
\sphinxAtStartPar
Summarise and output the number of unique values in the following columns:
\begin{itemize}
\item {} 
\sphinxAtStartPar
parent\_molregno (compound ID)

\item {} 
\sphinxAtStartPar
tid (target ID)

\item {} 
\sphinxAtStartPar
tid\_mutation (target ID + mutation annotations)

\item {} 
\sphinxAtStartPar
cpd\_target\_pair (compound\sphinxhyphen{}target pairs)

\item {} 
\sphinxAtStartPar
cpd\_target\_pair\_mutation (compound\sphinxhyphen{}target pairs including mutation annotations)

\end{itemize}
\begin{quote}\begin{description}
\sphinxlineitem{Parameters}\begin{itemize}
\item {} 
\sphinxAtStartPar
\sphinxstyleliteralstrong{\sphinxupquote{df}} (\sphinxstyleliteralemphasis{\sphinxupquote{pd.DataFrame}}) \textendash{} Pandas Dataframe for which the stats should be calculated

\item {} 
\sphinxAtStartPar
\sphinxstyleliteralstrong{\sphinxupquote{output\_file}} (\sphinxstyleliteralemphasis{\sphinxupquote{str}}) \textendash{} Path and filename to write the dataset stats to

\item {} 
\sphinxAtStartPar
\sphinxstyleliteralstrong{\sphinxupquote{write\_to\_csv}} (\sphinxstyleliteralemphasis{\sphinxupquote{bool}}) \textendash{} True if stats should be written to csv

\item {} 
\sphinxAtStartPar
\sphinxstyleliteralstrong{\sphinxupquote{write\_to\_excel}} (\sphinxstyleliteralemphasis{\sphinxupquote{bool}}) \textendash{} True if stats should be written to excel

\item {} 
\sphinxAtStartPar
\sphinxstyleliteralstrong{\sphinxupquote{delimiter}} (\sphinxstyleliteralemphasis{\sphinxupquote{str}}) \textendash{} Delimiter in csv\sphinxhyphen{}output

\end{itemize}

\end{description}\end{quote}

\end{fulllineitems}

\index{write\_and\_check\_output() (in module write\_subsets)@\spxentry{write\_and\_check\_output()}\spxextra{in module write\_subsets}}

\begin{fulllineitems}
\phantomsection\label{\detokenize{write_subsets:write_subsets.write_and_check_output}}
\pysigstartsignatures
\pysiglinewithargsret{\sphinxcode{\sphinxupquote{write\_subsets.}}\sphinxbfcode{\sphinxupquote{write\_and\_check\_output}}}{\sphinxparam{\DUrole{n}{df}\DUrole{p}{:}\DUrole{w}{ }\DUrole{n}{DataFrame}}\sphinxparamcomma \sphinxparam{\DUrole{n}{filename}\DUrole{p}{:}\DUrole{w}{ }\DUrole{n}{str}}\sphinxparamcomma \sphinxparam{\DUrole{n}{write\_to\_csv}\DUrole{p}{:}\DUrole{w}{ }\DUrole{n}{bool}}\sphinxparamcomma \sphinxparam{\DUrole{n}{write\_to\_excel}\DUrole{p}{:}\DUrole{w}{ }\DUrole{n}{bool}}\sphinxparamcomma \sphinxparam{\DUrole{n}{delimiter}\DUrole{p}{:}\DUrole{w}{ }\DUrole{n}{str}}\sphinxparamcomma \sphinxparam{\DUrole{n}{assay\_type}\DUrole{p}{:}\DUrole{w}{ }\DUrole{n}{str}}\sphinxparamcomma \sphinxparam{\DUrole{n}{calculate\_rdkit}\DUrole{p}{:}\DUrole{w}{ }\DUrole{n}{bool}}}{}
\pysigstopsignatures
\sphinxAtStartPar
Write df to file and check that writing was successful.
\begin{quote}\begin{description}
\sphinxlineitem{Parameters}\begin{itemize}
\item {} 
\sphinxAtStartPar
\sphinxstyleliteralstrong{\sphinxupquote{df}} (\sphinxstyleliteralemphasis{\sphinxupquote{pd.DataFrame}}) \textendash{} Pandas Dataframe to write to output file.

\item {} 
\sphinxAtStartPar
\sphinxstyleliteralstrong{\sphinxupquote{filename}} (\sphinxstyleliteralemphasis{\sphinxupquote{bool}}) \textendash{} Filename to write the output to

\item {} 
\sphinxAtStartPar
\sphinxstyleliteralstrong{\sphinxupquote{write\_to\_csv}} (\sphinxstyleliteralemphasis{\sphinxupquote{bool}}) \textendash{} True if output should be written to csv

\item {} 
\sphinxAtStartPar
\sphinxstyleliteralstrong{\sphinxupquote{write\_to\_excel}} (\sphinxstyleliteralemphasis{\sphinxupquote{bool}}) \textendash{} True if output should be written to excel

\item {} 
\sphinxAtStartPar
\sphinxstyleliteralstrong{\sphinxupquote{delimiter}} (\sphinxstyleliteralemphasis{\sphinxupquote{str}}) \textendash{} Delimiter in csv\sphinxhyphen{}output

\item {} 
\sphinxAtStartPar
\sphinxstyleliteralstrong{\sphinxupquote{assay\_type}} (\sphinxstyleliteralemphasis{\sphinxupquote{str}}) \textendash{} Types of assays current\_df contains information about.         Options: “BF” (binding+functional),
“B” (binding),
“all” (contains both BF and B information)

\item {} 
\sphinxAtStartPar
\sphinxstyleliteralstrong{\sphinxupquote{calculate\_rdkit}} (\sphinxstyleliteralemphasis{\sphinxupquote{bool}}) \textendash{} If True, current\_df contains RDKit\sphinxhyphen{}based columns

\end{itemize}

\end{description}\end{quote}

\end{fulllineitems}

\index{write\_b\_to\_file() (in module write\_subsets)@\spxentry{write\_b\_to\_file()}\spxextra{in module write\_subsets}}

\begin{fulllineitems}
\phantomsection\label{\detokenize{write_subsets:write_subsets.write_b_to_file}}
\pysigstartsignatures
\pysiglinewithargsret{\sphinxcode{\sphinxupquote{write\_subsets.}}\sphinxbfcode{\sphinxupquote{write\_b\_to\_file}}}{\sphinxparam{\DUrole{n}{df\_combined}\DUrole{p}{:}\DUrole{w}{ }\DUrole{n}{DataFrame}}\sphinxparamcomma \sphinxparam{\DUrole{n}{df\_combined\_annotated}\DUrole{p}{:}\DUrole{w}{ }\DUrole{n}{DataFrame}}\sphinxparamcomma \sphinxparam{\DUrole{n}{chembl\_version}\DUrole{p}{:}\DUrole{w}{ }\DUrole{n}{str}}\sphinxparamcomma \sphinxparam{\DUrole{n}{min\_nof\_cpds\_b}\DUrole{p}{:}\DUrole{w}{ }\DUrole{n}{int}}\sphinxparamcomma \sphinxparam{\DUrole{n}{output\_path}\DUrole{p}{:}\DUrole{w}{ }\DUrole{n}{str}}\sphinxparamcomma \sphinxparam{\DUrole{n}{write\_b}\DUrole{p}{:}\DUrole{w}{ }\DUrole{n}{bool}}\sphinxparamcomma \sphinxparam{\DUrole{n}{write\_to\_csv}\DUrole{p}{:}\DUrole{w}{ }\DUrole{n}{bool}}\sphinxparamcomma \sphinxparam{\DUrole{n}{write\_to\_excel}\DUrole{p}{:}\DUrole{w}{ }\DUrole{n}{bool}}\sphinxparamcomma \sphinxparam{\DUrole{n}{delimiter}\DUrole{p}{:}\DUrole{w}{ }\DUrole{n}{str}}\sphinxparamcomma \sphinxparam{\DUrole{n}{limited\_flag}\DUrole{p}{:}\DUrole{w}{ }\DUrole{n}{str}}\sphinxparamcomma \sphinxparam{\DUrole{n}{calculate\_rdkit}\DUrole{p}{:}\DUrole{w}{ }\DUrole{n}{bool}}\sphinxparamcomma \sphinxparam{\DUrole{n}{df\_sizes}\DUrole{p}{:}\DUrole{w}{ }\DUrole{n}{list\DUrole{p}{{[}}list\DUrole{p}{{[}}int\DUrole{p}{{]}}\DUrole{p}{,}\DUrole{w}{ }list\DUrole{p}{{[}}int\DUrole{p}{{]}}\DUrole{p}{{]}}}}}{{ $\rightarrow$ DataFrame}}
\pysigstopsignatures
\sphinxAtStartPar
Calculate relevant subsets for the portion of df\_combined that is based on binding data.
If write\_b the subsets are written to output\_path.
Independent of write\_b, filtering columns for B are added to df\_combined\_annotated.
\begin{quote}\begin{description}
\sphinxlineitem{Parameters}\begin{itemize}
\item {} 
\sphinxAtStartPar
\sphinxstyleliteralstrong{\sphinxupquote{df\_combined}} (\sphinxstyleliteralemphasis{\sphinxupquote{pd.DataFrame}}) \textendash{} Pandas DataFrame with compound\sphinxhyphen{}target pairs

\item {} 
\sphinxAtStartPar
\sphinxstyleliteralstrong{\sphinxupquote{df\_combined\_annotated}} (\sphinxstyleliteralemphasis{\sphinxupquote{pd.DataFrame}}) \textendash{} Pandas DataFrame with additional filtering columns

\item {} 
\sphinxAtStartPar
\sphinxstyleliteralstrong{\sphinxupquote{chembl\_version}} (\sphinxstyleliteralemphasis{\sphinxupquote{str}}) \textendash{} Version of ChEMBL for output files

\item {} 
\sphinxAtStartPar
\sphinxstyleliteralstrong{\sphinxupquote{min\_nof\_cpds\_b}} (\sphinxstyleliteralemphasis{\sphinxupquote{int}}) \textendash{} Miminum number of compounds per target

\item {} 
\sphinxAtStartPar
\sphinxstyleliteralstrong{\sphinxupquote{output\_path}} (\sphinxstyleliteralemphasis{\sphinxupquote{str}}) \textendash{} Path to write the output to

\item {} 
\sphinxAtStartPar
\sphinxstyleliteralstrong{\sphinxupquote{write\_b}} (\sphinxstyleliteralemphasis{\sphinxupquote{bool}}) \textendash{} Should the subsets be written to files?

\item {} 
\sphinxAtStartPar
\sphinxstyleliteralstrong{\sphinxupquote{write\_to\_csv}} (\sphinxstyleliteralemphasis{\sphinxupquote{bool}}) \textendash{} Should the subsets be written to csv?

\item {} 
\sphinxAtStartPar
\sphinxstyleliteralstrong{\sphinxupquote{write\_to\_excel}} (\sphinxstyleliteralemphasis{\sphinxupquote{bool}}) \textendash{} Should the subsets be written to excel?

\item {} 
\sphinxAtStartPar
\sphinxstyleliteralstrong{\sphinxupquote{delimiter}} (\sphinxstyleliteralemphasis{\sphinxupquote{str}}) \textendash{} Delimiter for csv output

\item {} 
\sphinxAtStartPar
\sphinxstyleliteralstrong{\sphinxupquote{limited\_flag}} (\sphinxstyleliteralemphasis{\sphinxupquote{str}}) \textendash{} Document suffix indicating
whether the dataset was limited to literature sources

\item {} 
\sphinxAtStartPar
\sphinxstyleliteralstrong{\sphinxupquote{calculate\_rdkit}} (\sphinxstyleliteralemphasis{\sphinxupquote{bool}}) \textendash{} Does df\_combined include RDKit\sphinxhyphen{}based columns?

\item {} 
\sphinxAtStartPar
\sphinxstyleliteralstrong{\sphinxupquote{df\_sizes}} (\sphinxstyleliteralemphasis{\sphinxupquote{list}}\sphinxstyleliteralemphasis{\sphinxupquote{{[}}}\sphinxstyleliteralemphasis{\sphinxupquote{list}}\sphinxstyleliteralemphasis{\sphinxupquote{{[}}}\sphinxstyleliteralemphasis{\sphinxupquote{int}}\sphinxstyleliteralemphasis{\sphinxupquote{{]}}}\sphinxstyleliteralemphasis{\sphinxupquote{, }}\sphinxstyleliteralemphasis{\sphinxupquote{list}}\sphinxstyleliteralemphasis{\sphinxupquote{{[}}}\sphinxstyleliteralemphasis{\sphinxupquote{int}}\sphinxstyleliteralemphasis{\sphinxupquote{{]}}}\sphinxstyleliteralemphasis{\sphinxupquote{{]}}}) \textendash{} List of intermediate sized of the dataset used for debugging.

\end{itemize}

\sphinxlineitem{Returns}
\sphinxAtStartPar
Pandas DataFrame with additional filtering columns for B subsets

\sphinxlineitem{Return type}
\sphinxAtStartPar
pd.Dataframe

\end{description}\end{quote}

\end{fulllineitems}

\index{write\_bf\_to\_file() (in module write\_subsets)@\spxentry{write\_bf\_to\_file()}\spxextra{in module write\_subsets}}

\begin{fulllineitems}
\phantomsection\label{\detokenize{write_subsets:write_subsets.write_bf_to_file}}
\pysigstartsignatures
\pysiglinewithargsret{\sphinxcode{\sphinxupquote{write\_subsets.}}\sphinxbfcode{\sphinxupquote{write\_bf\_to\_file}}}{\sphinxparam{\DUrole{n}{df\_combined}\DUrole{p}{:}\DUrole{w}{ }\DUrole{n}{DataFrame}}\sphinxparamcomma \sphinxparam{\DUrole{n}{chembl\_version}\DUrole{p}{:}\DUrole{w}{ }\DUrole{n}{str}}\sphinxparamcomma \sphinxparam{\DUrole{n}{min\_nof\_cpds\_bf}\DUrole{p}{:}\DUrole{w}{ }\DUrole{n}{int}}\sphinxparamcomma \sphinxparam{\DUrole{n}{output\_path}\DUrole{p}{:}\DUrole{w}{ }\DUrole{n}{str}}\sphinxparamcomma \sphinxparam{\DUrole{n}{write\_bf}\DUrole{p}{:}\DUrole{w}{ }\DUrole{n}{bool}}\sphinxparamcomma \sphinxparam{\DUrole{n}{write\_to\_csv}\DUrole{p}{:}\DUrole{w}{ }\DUrole{n}{bool}}\sphinxparamcomma \sphinxparam{\DUrole{n}{write\_to\_excel}\DUrole{p}{:}\DUrole{w}{ }\DUrole{n}{bool}}\sphinxparamcomma \sphinxparam{\DUrole{n}{delimiter}\DUrole{p}{:}\DUrole{w}{ }\DUrole{n}{str}}\sphinxparamcomma \sphinxparam{\DUrole{n}{limited\_flag}\DUrole{p}{:}\DUrole{w}{ }\DUrole{n}{str}}\sphinxparamcomma \sphinxparam{\DUrole{n}{calculate\_rdkit}\DUrole{p}{:}\DUrole{w}{ }\DUrole{n}{bool}}\sphinxparamcomma \sphinxparam{\DUrole{n}{df\_sizes}\DUrole{p}{:}\DUrole{w}{ }\DUrole{n}{list\DUrole{p}{{[}}list\DUrole{p}{{[}}int\DUrole{p}{{]}}\DUrole{p}{,}\DUrole{w}{ }list\DUrole{p}{{[}}int\DUrole{p}{{]}}\DUrole{p}{{]}}}}}{{ $\rightarrow$ DataFrame}}
\pysigstopsignatures
\sphinxAtStartPar
Calculate relevant subsets for the portion of df\_combined
that is based on binding+functional data.
If write\_bf the subsets are written to output\_path.
Independent of write\_bf, filtering columns for BF are added to df\_combined and returned.
\begin{quote}\begin{description}
\sphinxlineitem{Parameters}\begin{itemize}
\item {} 
\sphinxAtStartPar
\sphinxstyleliteralstrong{\sphinxupquote{df\_combined}} (\sphinxstyleliteralemphasis{\sphinxupquote{pd.DataFrame}}) \textendash{} Pandas DataFrame with compound\sphinxhyphen{}target pairs

\item {} 
\sphinxAtStartPar
\sphinxstyleliteralstrong{\sphinxupquote{chembl\_version}} (\sphinxstyleliteralemphasis{\sphinxupquote{str}}) \textendash{} Version of ChEMBL for output files

\item {} 
\sphinxAtStartPar
\sphinxstyleliteralstrong{\sphinxupquote{min\_nof\_cpds\_bf}} (\sphinxstyleliteralemphasis{\sphinxupquote{int}}) \textendash{} Miminum number of compounds per target

\item {} 
\sphinxAtStartPar
\sphinxstyleliteralstrong{\sphinxupquote{output\_path}} (\sphinxstyleliteralemphasis{\sphinxupquote{str}}) \textendash{} Path to write the output to

\item {} 
\sphinxAtStartPar
\sphinxstyleliteralstrong{\sphinxupquote{write\_bf}} (\sphinxstyleliteralemphasis{\sphinxupquote{bool}}) \textendash{} Should the subsets be written to files?

\item {} 
\sphinxAtStartPar
\sphinxstyleliteralstrong{\sphinxupquote{write\_to\_csv}} (\sphinxstyleliteralemphasis{\sphinxupquote{bool}}) \textendash{} Should the subsets be written to csv?

\item {} 
\sphinxAtStartPar
\sphinxstyleliteralstrong{\sphinxupquote{write\_to\_excel}} (\sphinxstyleliteralemphasis{\sphinxupquote{bool}}) \textendash{} Should the subsets be written to excel?

\item {} 
\sphinxAtStartPar
\sphinxstyleliteralstrong{\sphinxupquote{delimiter}} (\sphinxstyleliteralemphasis{\sphinxupquote{str}}) \textendash{} Delimiter for csv output

\item {} 
\sphinxAtStartPar
\sphinxstyleliteralstrong{\sphinxupquote{limited\_flag}} (\sphinxstyleliteralemphasis{\sphinxupquote{str}}) \textendash{} Document suffix indicating
whether the dataset was limited to literature sources

\item {} 
\sphinxAtStartPar
\sphinxstyleliteralstrong{\sphinxupquote{calculate\_rdkit}} (\sphinxstyleliteralemphasis{\sphinxupquote{bool}}) \textendash{} Does df\_combined include RDKit\sphinxhyphen{}based columns?

\item {} 
\sphinxAtStartPar
\sphinxstyleliteralstrong{\sphinxupquote{df\_sizes}} (\sphinxstyleliteralemphasis{\sphinxupquote{list}}\sphinxstyleliteralemphasis{\sphinxupquote{{[}}}\sphinxstyleliteralemphasis{\sphinxupquote{list}}\sphinxstyleliteralemphasis{\sphinxupquote{{[}}}\sphinxstyleliteralemphasis{\sphinxupquote{int}}\sphinxstyleliteralemphasis{\sphinxupquote{{]}}}\sphinxstyleliteralemphasis{\sphinxupquote{, }}\sphinxstyleliteralemphasis{\sphinxupquote{list}}\sphinxstyleliteralemphasis{\sphinxupquote{{[}}}\sphinxstyleliteralemphasis{\sphinxupquote{int}}\sphinxstyleliteralemphasis{\sphinxupquote{{]}}}\sphinxstyleliteralemphasis{\sphinxupquote{{]}}}) \textendash{} List of intermediate sized of the dataset used for debugging.

\end{itemize}

\sphinxlineitem{Returns}
\sphinxAtStartPar
Pandas DataFrame with additional filtering columns for BF subsets

\sphinxlineitem{Return type}
\sphinxAtStartPar
pd.Dataframe

\end{description}\end{quote}

\end{fulllineitems}

\index{write\_full\_dataset\_to\_file() (in module write\_subsets)@\spxentry{write\_full\_dataset\_to\_file()}\spxextra{in module write\_subsets}}

\begin{fulllineitems}
\phantomsection\label{\detokenize{write_subsets:write_subsets.write_full_dataset_to_file}}
\pysigstartsignatures
\pysiglinewithargsret{\sphinxcode{\sphinxupquote{write\_subsets.}}\sphinxbfcode{\sphinxupquote{write\_full\_dataset\_to\_file}}}{\sphinxparam{\DUrole{n}{df\_combined}\DUrole{p}{:}\DUrole{w}{ }\DUrole{n}{DataFrame}}\sphinxparamcomma \sphinxparam{\DUrole{n}{chembl\_version}\DUrole{p}{:}\DUrole{w}{ }\DUrole{n}{str}}\sphinxparamcomma \sphinxparam{\DUrole{n}{output\_path}\DUrole{p}{:}\DUrole{w}{ }\DUrole{n}{str}}\sphinxparamcomma \sphinxparam{\DUrole{n}{write\_full\_dataset}\DUrole{p}{:}\DUrole{w}{ }\DUrole{n}{bool}}\sphinxparamcomma \sphinxparam{\DUrole{n}{write\_to\_csv}\DUrole{p}{:}\DUrole{w}{ }\DUrole{n}{bool}}\sphinxparamcomma \sphinxparam{\DUrole{n}{write\_to\_excel}\DUrole{p}{:}\DUrole{w}{ }\DUrole{n}{bool}}\sphinxparamcomma \sphinxparam{\DUrole{n}{delimiter}\DUrole{p}{:}\DUrole{w}{ }\DUrole{n}{str}}\sphinxparamcomma \sphinxparam{\DUrole{n}{limited\_flag}\DUrole{p}{:}\DUrole{w}{ }\DUrole{n}{str}}\sphinxparamcomma \sphinxparam{\DUrole{n}{calculate\_rdkit}\DUrole{p}{:}\DUrole{w}{ }\DUrole{n}{bool}}}{}
\pysigstopsignatures
\sphinxAtStartPar
If write\_full\_dataset, write df\_combined with filtering columns to output\_path.
\begin{quote}\begin{description}
\sphinxlineitem{Parameters}\begin{itemize}
\item {} 
\sphinxAtStartPar
\sphinxstyleliteralstrong{\sphinxupquote{df\_combined}} (\sphinxstyleliteralemphasis{\sphinxupquote{pd.DataFrame}}) \textendash{} Pandas DataFrame with compound\sphinxhyphen{}target pairs and filtering columns

\item {} 
\sphinxAtStartPar
\sphinxstyleliteralstrong{\sphinxupquote{chembl\_version}} (\sphinxstyleliteralemphasis{\sphinxupquote{str}}) \textendash{} Version of ChEMBL for output files

\item {} 
\sphinxAtStartPar
\sphinxstyleliteralstrong{\sphinxupquote{output\_path}} (\sphinxstyleliteralemphasis{\sphinxupquote{str}}) \textendash{} Path to write the output to

\item {} 
\sphinxAtStartPar
\sphinxstyleliteralstrong{\sphinxupquote{write\_full\_dataset}} (\sphinxstyleliteralemphasis{\sphinxupquote{bool}}) \textendash{} Should the subsets be written to files?

\item {} 
\sphinxAtStartPar
\sphinxstyleliteralstrong{\sphinxupquote{write\_to\_csv}} (\sphinxstyleliteralemphasis{\sphinxupquote{bool}}) \textendash{} Should the subsets be written to csv?

\item {} 
\sphinxAtStartPar
\sphinxstyleliteralstrong{\sphinxupquote{write\_to\_excel}} (\sphinxstyleliteralemphasis{\sphinxupquote{bool}}) \textendash{} Should the subsets be written to excel?

\item {} 
\sphinxAtStartPar
\sphinxstyleliteralstrong{\sphinxupquote{delimiter}} (\sphinxstyleliteralemphasis{\sphinxupquote{str}}) \textendash{} Delimiter for csv output

\item {} 
\sphinxAtStartPar
\sphinxstyleliteralstrong{\sphinxupquote{limited\_flag}} (\sphinxstyleliteralemphasis{\sphinxupquote{str}}) \textendash{} Document suffix indicating
whether the dataset was limited to literature sources

\item {} 
\sphinxAtStartPar
\sphinxstyleliteralstrong{\sphinxupquote{calculate\_rdkit}} (\sphinxstyleliteralemphasis{\sphinxupquote{bool}}) \textendash{} Does df\_combined include RDKit\sphinxhyphen{}based columns?

\end{itemize}

\end{description}\end{quote}

\end{fulllineitems}

\index{write\_output() (in module write\_subsets)@\spxentry{write\_output()}\spxextra{in module write\_subsets}}

\begin{fulllineitems}
\phantomsection\label{\detokenize{write_subsets:write_subsets.write_output}}
\pysigstartsignatures
\pysiglinewithargsret{\sphinxcode{\sphinxupquote{write\_subsets.}}\sphinxbfcode{\sphinxupquote{write\_output}}}{\sphinxparam{\DUrole{n}{df}\DUrole{p}{:}\DUrole{w}{ }\DUrole{n}{DataFrame}}\sphinxparamcomma \sphinxparam{\DUrole{n}{filename}\DUrole{p}{:}\DUrole{w}{ }\DUrole{n}{str}}\sphinxparamcomma \sphinxparam{\DUrole{n}{write\_to\_csv}\DUrole{p}{:}\DUrole{w}{ }\DUrole{n}{bool}}\sphinxparamcomma \sphinxparam{\DUrole{n}{write\_to\_excel}\DUrole{p}{:}\DUrole{w}{ }\DUrole{n}{bool}}\sphinxparamcomma \sphinxparam{\DUrole{n}{delimiter}\DUrole{p}{:}\DUrole{w}{ }\DUrole{n}{str}}}{{ $\rightarrow$ list\DUrole{p}{{[}}str\DUrole{p}{{]}}}}
\pysigstopsignatures
\sphinxAtStartPar
Write DataFrame df to output file named \textless{}filename\textgreater{}.
\begin{quote}\begin{description}
\sphinxlineitem{Parameters}\begin{itemize}
\item {} 
\sphinxAtStartPar
\sphinxstyleliteralstrong{\sphinxupquote{df}} (\sphinxstyleliteralemphasis{\sphinxupquote{pd.DataFrame}}) \textendash{} Pandas Dataframe to write to output file.

\item {} 
\sphinxAtStartPar
\sphinxstyleliteralstrong{\sphinxupquote{filename}} (\sphinxstyleliteralemphasis{\sphinxupquote{bool}}) \textendash{} Filename to write the output to

\item {} 
\sphinxAtStartPar
\sphinxstyleliteralstrong{\sphinxupquote{write\_to\_csv}} (\sphinxstyleliteralemphasis{\sphinxupquote{bool}}) \textendash{} True if output should be written to csv

\item {} 
\sphinxAtStartPar
\sphinxstyleliteralstrong{\sphinxupquote{write\_to\_excel}} (\sphinxstyleliteralemphasis{\sphinxupquote{bool}}) \textendash{} True if output should be written to excel

\item {} 
\sphinxAtStartPar
\sphinxstyleliteralstrong{\sphinxupquote{delimiter}} (\sphinxstyleliteralemphasis{\sphinxupquote{str}}) \textendash{} Delimiter in csv\sphinxhyphen{}output

\end{itemize}

\sphinxlineitem{Returns}
\sphinxAtStartPar
Returns list of types of files that was written to (csv and/or xlsx)

\sphinxlineitem{Return type}
\sphinxAtStartPar
list{[}str{]}

\end{description}\end{quote}

\end{fulllineitems}



\chapter{Indices and tables}
\label{\detokenize{index:indices-and-tables}}\begin{itemize}
\item {} 
\sphinxAtStartPar
\DUrole{xref,std,std-ref}{genindex}

\item {} 
\sphinxAtStartPar
\DUrole{xref,std,std-ref}{modindex}

\item {} 
\sphinxAtStartPar
\DUrole{xref,std,std-ref}{search}

\end{itemize}

\begin{sphinxthebibliography}{Zdrazil2}
\bibitem[Zdrazil2023]{introduction:zdrazil2023}
\sphinxAtStartPar
Zdrazil et al., “The ChEMBL Database in 2023: a drug discovery platform spanning multiple bioactivity data types and time periods”,
Nucleic Acids Research, gkad1004, 2023, \sphinxurl{https://doi.org/10.1093/nar/gkad1004}
\bibitem[Leeson2021]{introduction:leeson2021}
\sphinxAtStartPar
Leeson et al., “Target\sphinxhyphen{}Based Evaluation of “Drug\sphinxhyphen{}Like” Properties and Ligand Efficiencies”,
Journal of Medicinal Chemistry, 64(11), 7210\sphinxhyphen{}7230, 2021, \sphinxurl{https://doi.org/10.1021/acs.jmedchem.1c00416}
\end{sphinxthebibliography}


\renewcommand{\indexname}{Python Module Index}
\begin{sphinxtheindex}
\let\bigletter\sphinxstyleindexlettergroup
\bigletter{a}
\item\relax\sphinxstyleindexentry{add\_chembl\_compound\_properties}\sphinxstyleindexpageref{add_chembl_compound_properties:\detokenize{module-add_chembl_compound_properties}}
\item\relax\sphinxstyleindexentry{add\_chembl\_target\_class\_annotations}\sphinxstyleindexpageref{add_chembl_target_class_annotations:\detokenize{module-add_chembl_target_class_annotations}}
\item\relax\sphinxstyleindexentry{add\_dti\_annotations}\sphinxstyleindexpageref{add_dti_annotations:\detokenize{module-add_dti_annotations}}
\item\relax\sphinxstyleindexentry{add\_rdkit\_compound\_descriptors}\sphinxstyleindexpageref{add_rdkit_compound_descriptors:\detokenize{module-add_rdkit_compound_descriptors}}
\indexspace
\bigletter{c}
\item\relax\sphinxstyleindexentry{clean\_dataset}\sphinxstyleindexpageref{clean_dataset:\detokenize{module-clean_dataset}}
\indexspace
\bigletter{g}
\item\relax\sphinxstyleindexentry{get\_activity\_ct\_pairs}\sphinxstyleindexpageref{get_activity_ct_pairs:\detokenize{module-get_activity_ct_pairs}}
\item\relax\sphinxstyleindexentry{get\_dataset}\sphinxstyleindexpageref{get_dataset:\detokenize{module-get_dataset}}
\item\relax\sphinxstyleindexentry{get\_drug\_mechanism\_ct\_pairs}\sphinxstyleindexpageref{get_drug_mechanism_ct_pairs:\detokenize{module-get_drug_mechanism_ct_pairs}}
\item\relax\sphinxstyleindexentry{get\_stats}\sphinxstyleindexpageref{get_stats:\detokenize{module-get_stats}}
\indexspace
\bigletter{m}
\item\relax\sphinxstyleindexentry{main}\sphinxstyleindexpageref{main:\detokenize{module-main}}
\indexspace
\bigletter{s}
\item\relax\sphinxstyleindexentry{sanity\_checks}\sphinxstyleindexpageref{sanity_checks:\detokenize{module-sanity_checks}}
\indexspace
\bigletter{w}
\item\relax\sphinxstyleindexentry{write\_subsets}\sphinxstyleindexpageref{write_subsets:\detokenize{module-write_subsets}}
\end{sphinxtheindex}

\renewcommand{\indexname}{Index}
\printindex
\end{document}